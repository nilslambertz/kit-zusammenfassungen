\documentclass[12pt,A4]{extarticle}	

\newcommand{\lectureTitle}{Symmetrische Verschlüsselungsverfahren}
\newcommand{\semester}{Sommersemester 2023}

\newcommand{\titleSize}{\LARGE}

\usepackage[a4paper,left=0.9cm,right=1cm,top=1.37cm,bottom=2.5cm]{geometry}
\usepackage[utf8]{inputenc}
\usepackage{xifthen}
\usepackage{cmbright}
\usepackage{fontawesome}
\usepackage[T1]{fontenc}
\usepackage{lastpage,lipsum}
\usepackage{hyperref}
\usepackage{transparent}
\usepackage{color}
\usepackage{fancyhdr}

\renewcommand*\familydefault{\sfdefault}
\setlength{\parindent}{0mm}

\usepackage{transparent}
\usepackage{color}
\usepackage{fancyhdr}

\definecolor{headerBg}{RGB}{11, 67, 158}
\definecolor{headerGrayColor}{RGB}{210, 210, 210}

\pagestyle{fancy}
\fancyhead[C]{
  \fcolorbox{headerBg}{headerBg}{
    \hspace{0.6cm}\begin{minipage}[c][50pt][c]{\paperwidth}
      \begin{minipage}[c]{.45\textwidth}
        \huge{\textcolor{white}{Vorlesung}}\normalsize\\
        \textcolor{headerGrayColor}{\small{Wintersemester 2022/23}}
      \end{minipage}%
      \begin{minipage}[c]{.45\textwidth}
        \raggedleft
        \textcolor{white}{
          \small{\href{https://nilslambertz.de/}{nilslambertz.de}}\\
          \href{https://github.com/nilslambertz/}{\textcolor{white}{\faicon{github}} \small{nilslambertz}}}
      \end{minipage}
    \end{minipage}}
}
\renewcommand{\headrulewidth}{0pt}
\setlength{\headheight}{40pt}

\newlength{\oddmarginwidth}
\setlength{\oddmarginwidth}{1in+\hoffset+\oddsidemargin}
\newlength{\evenmarginwidth}
\setlength{\evenmarginwidth}{\evensidemargin+1in}
\fancyhfoffset[LO,RE]{\oddmarginwidth}
\fancyhfoffset[LE,RO]{\evenmarginwidth}
\cfoot{\thepage\ $/$ \pageref*{LastPage}}


\definecolor{highlightColor}{RGB}{66, 135, 245}
\newcommand{\highlight}[1]{\textcolor{highlightColor}{\textbf{#1}}}

\def\contentsname{\empty}

\begin{document}

\disclaimer

\tableofcontents
\clearpage

\section{Einführung}
\subsection{Ausgangspunkte für Angriffe}
Angriffe können nach den zur Verfügung stehenden Informationen unterteilt werden:
\begin{itemize}
  \item{\highlight{Ciphertext-Only-Attack}: Nur das \textit{Chiffre}, also die verschlüsselte Nachricht, ist bekannt}
  \item{\highlight{Known-Plaintext-Attack}: Es gibt bekannte Klartext-Chiffre-Paare. Hilfreich sind bekannte Anfangs- und Endphrasen, die in mehreren Nachrichten vorkommen.}
  \item{\highlight{Chosen-Plaintext-Attack}: Es besteht die Möglichkeit, beliebige Texte zu verschlüsseln und somit Klartext-Chiffre-Paare zu erzeugen.}
\end{itemize}

\subsection{Angriffsarten}
\begin{itemize}
  \item{Brute-Force (z.B. alle Schlüssel ausprobieren)}
  \item{Statistische Methoden (z.B. Häufigkeitsanalysen von Buchstaben)}
  \item{Strukturelle Angriffe (z.B. Lineare Kryptoanalyse)}
\end{itemize}

\subsection{Historische Verschlüsselungsverfahren}
Historisch wurden zur Verschlüsselung zwei grundlegende Operationen verwendet:
\begin{itemize}
  \item{\highlight{Substitution}}
  \item{\highlight{Permutation}}
\end{itemize}
Alleine sind beide Verfahren meistens nicht sicher, jedoch verwenden moderne Verschlüsselungsverfahren eine Kombination beider Operationen.

\end{document}