\documentclass[12pt,A4]{extarticle}	

\usepackage{amsfonts}
\usepackage{amsmath}
\usepackage{amssymb}
\usepackage{graphicx,wrapfig,lipsum}
\usepackage[german]{babel}

\newcommand{\lectureTitle}{Authentisierung und Verschlüsselung [WIP]}
\newcommand{\semester}{Sommersemester 2023}

\newcommand{\titleSize}{\LARGE}

\usepackage[a4paper,left=0.9cm,right=1cm,top=1.37cm,bottom=2.5cm]{geometry}
\usepackage[utf8]{inputenc}
\usepackage{xifthen}
\usepackage{cmbright}
\usepackage{fontawesome}
\usepackage[T1]{fontenc}
\usepackage{lastpage,lipsum}
\usepackage{hyperref}
\usepackage{transparent}
\usepackage{color}
\usepackage{fancyhdr}

\renewcommand*\familydefault{\sfdefault}
\setlength{\parindent}{0mm}

\usepackage{transparent}
\usepackage{color}
\usepackage{fancyhdr}

\definecolor{headerBg}{RGB}{11, 67, 158}
\definecolor{headerGrayColor}{RGB}{210, 210, 210}

\pagestyle{fancy}
\fancyhead[C]{
  \fcolorbox{headerBg}{headerBg}{
    \hspace{0.6cm}\begin{minipage}[c][50pt][c]{\paperwidth}
      \begin{minipage}[c]{.45\textwidth}
        \huge{\textcolor{white}{Vorlesung}}\normalsize\\
        \textcolor{headerGrayColor}{\small{Wintersemester 2022/23}}
      \end{minipage}%
      \begin{minipage}[c]{.45\textwidth}
        \raggedleft
        \textcolor{white}{
          \small{\href{https://nilslambertz.de/}{nilslambertz.de}}\\
          \href{https://github.com/nilslambertz/}{\textcolor{white}{\faicon{github}} \small{nilslambertz}}}
      \end{minipage}
    \end{minipage}}
}
\renewcommand{\headrulewidth}{0pt}
\setlength{\headheight}{40pt}

\newlength{\oddmarginwidth}
\setlength{\oddmarginwidth}{1in+\hoffset+\oddsidemargin}
\newlength{\evenmarginwidth}
\setlength{\evenmarginwidth}{\evensidemargin+1in}
\fancyhfoffset[LO,RE]{\oddmarginwidth}
\fancyhfoffset[LE,RO]{\evenmarginwidth}
\cfoot{\thepage\ $/$ \pageref*{LastPage}}

\definecolor{highlightColor}{RGB}{66, 135, 245}
\newcommand{\highlight}[1]{\textcolor{highlightColor}{\textbf{#1}}}

\def\contentsname{\empty}
\addto\captionsgerman{
  \renewcommand{\contentsname}{\empty}
}

\begin{document}

\disclaimer

\tableofcontents
\clearpage

\section{Einführung}
\subsection{Ziel von Kryptographischen Verfahren}
Kryptographische Verfahren sollen \highlight{Authentizität} (Dokument wurde von einer bestimmten Person signiert) und \highlight{Integrität} (Dokument wurde nicht verändert) sicherstellen.

\subsection{Informelle Definition von Signaturen}
\begin{itemize}
  \item{\textbf{asymmetrische} Verfahren}
  \item{Schlüsselpaar $(pk, sk)$}
  \item{Nachricht $m$ wird mit $sk$ signiert und erzeugt Signatur $\sigma$}
  \item{Mit $pk$ kann überprüft werden, ob eine Signatur $\sigma$ gültig für eine Nachricht $m$ ist}
\end{itemize}

\subsection{Digitale Signaturen}
\subsubsection{Definition}
Ein digitales Signaturverfahren für einen Nachrichtenraum $\mathcal{M}$ ist ein Tupel $\Sigma = (Gen, Sign, Vfy)$ von probabilistischen Polyzeit (PPT) Algorithmen:
\begin{itemize}
  \item{$Gen(1^k) \rightarrow (pk, sk)$}
  \item{$Sign(sk, m) \rightarrow \sigma$, $m \in \mathcal{M}$}
  \item{$Vfy(pk, m, \sigma) \in \{0, 1\}$}
\end{itemize}

\subsubsection{Correctness}
\highlight{Correctness} (``Das Verfahren funktioniert''): $\forall (pk, sk) \leftarrow Gen(1^k) \forall m \in \mathcal{M}: Vfy(pk, m, Sign(sk, m)) = 1$

\subsection{Sicherheitsdefinitionen}
Sicherheit besteht aus einem \highlight{Angreifermodell} (was kann der Angreifer tun, welche Angriffsmöglichkeiten stehen zur Verfügung) und einem \highlight{Angreiferziel} (was muss der Angreifer tun, um das Verfahren zu brechen).

\subsubsection{Angreifermodelle}
\begin{enumerate}
  \item{no-message attack (NMA)
              \begin{itemize}
                \item{Angreifer erhält nur $pk$}
              \end{itemize}
        }
  \item{\highlight{non-adaptive chosen-message attack (naCMA)}
              \begin{itemize}
                \item{Angreifer wählt $m_1, \dots, m_q$}
                \item{Angreifer erhält \textbf{danach} $pk$ und Signaturen $\sigma_1, \dots, \sigma_q$}
              \end{itemize}
        }
  \item{\highlight{(adaptive) chosen-message attack (CMA)}
              \begin{itemize}
                \item{Angreifer erhält $pk$}
                \item{Angreifer wählt dann (adaptiv) $m_1, \dots, m_q$ und erhält Signaturen $\sigma_1, \dots, \sigma_q$}
                \item{Adaptiv: Angreifer darf Wahl von $m_i$ abhängig von vorherigen $\sigma_j$ ($j < i$) und $pk$ machen}
              \end{itemize}
        }
\end{enumerate}

\subsubsection{Angreiferziele}
\begin{enumerate}
  \item{Universal Unforgeability (UUF)
              \begin{itemize}
                \item{Nachricht $m$ wird zufällig gewählt}
                \item{Angreifer muss $m$ signieren}
              \end{itemize}
        }
  \item{\highlight{Existential Unforgeablility (EUF)}
              \begin{itemize}
                \item{Angreifer kann Nachricht $m$ beliebig wählen und diese signieren}
              \end{itemize}
        }
\end{enumerate}

In den \textbf{Sicherheitsdefinitionen} werden \textbf{Angreiferziel} und \textbf{Angreifermodell} kombiniert, z.B.
\begin{itemize}
  \item{EUF-CMA}
  \item{EUF-naCMA}
\end{itemize}

\subsection{EUF-CMA-Sicherheitsexperiment}
Bei Sicherheitsexperimenten spielt ein Angreifer $\mathcal{A}$ gegen einen Challenger $\mathcal{C}$. $\mathcal{A}$ gewinnt, falls er die Sicherheit des Verfahrens bricht.\par
$\mathcal{A}$ muss dabei mit einer nicht vernachlässigbaren Wahrscheinlichkeit eine gültige Signatur erzeugen können, ohne den Schlüssel $sk$ zu kennen.

\subsubsection{Visualisierung: EUF-CMA-Sicherheitsexperiment}
\begin{tikzpicture}
  \node (A) at (0,6) {$\mathcal{C}_\text{EUF-CMA}$};
  \node (B) at (5,6) {$\mathcal{A}$};
  \node[label={[align=center]below:$Vfy(pk, m^*, \sigma^*) = 1$?\\ $\land$ \\ $m^* \notin \{m_1, \dots, m_q\}$?}] (C) at (0,0) {};
  \node (D) at (5,0) {};

  \draw[dashed] (A) -- (C);
  \draw[dashed] (B) -- (D);

  \node[label={left:$(pk, sk) \leftarrow Gen(1^k)$}] at (0,5) {};
  \node[label={left:$\sigma_i \leftarrow Sign(sk, m_i)$}]  at (0,3) {};

  \draw[decoration={calligraphic brace,amplitude=10pt}, decorate, line width=1.25pt] (5.2,4) -- (5.2,2)
  node[midway, right=0pt, font=\footnotesize] {\begin{minipage}{5cm}\begin{itemize}
        \item{Anfragen nacheinander}
        \item{$q = q(k)$ Anfragen}
        \item{$q$ Polynom}
      \end{itemize}\end{minipage}};

  \draw[->,shorten >=5pt, shorten <=5pt] (0,5) -- (5,4.5) node[midway, above, sloped] {$pk$};
  \draw[->,shorten >=5pt, shorten <=5pt] (5,4) -- (0,3.5) node[midway, above, sloped] {$m_i$};

  \draw[->,shorten >=5pt, shorten <=5pt] (0,2.5) -- (5,2) node[midway, above, sloped] {$\sigma_i$};
  \draw[->,shorten >=5pt, shorten <=5pt] (5,1.5) -- (0,1) node[midway, above, sloped] {$m^*, \sigma^*$};
\end{tikzpicture}

$\mathcal{A}$ gewinnt, falls $Vfy(pk, m^*, \sigma^*) = 1$ \textbf{und} $m^* \notin \{m_1, \dots, m_q\}$

\subsubsection{Definition: Vernachlässigbarkeit}
Eine Funktion $negl: \mathbb{N} \rightarrow [0, 1]$ ist \textit{vernachlässigbar}, wenn
\begin{flalign*}
  \forall c \in \mathbb{N} \exists k_0 \in \mathbb{N} \forall k \geq k_0: negl(k) < \frac{1}{k^c}
\end{flalign*}

\subsubsection{Definition: EUF-CMA}
Ein digitales Signaturverfahren $\Sigma = (Gen, Sign, Vfy)$ ist \textit{EUF-CMA-sicher}, wenn für alle PPT $\mathcal{A}$ gilt, dass
\begin{flalign*}
   & \Pr[\mathcal{A} \text{ gewinnt EUF-CMA-Experiment}]                                                                                    \\
   & = \Pr[\mathcal{A}^{\mathcal{C}_\text{EUF-CMA}}(pk) = (m^*, \sigma^*): Vfy(pk, m^*, \sigma^*) = 1 \land m^* \notin \{m_1, \dots, m_q\}] \\
   & \leq negl(k)
\end{flalign*}
für eine im Sicherheitsparameter $k$ vernachlässigbare Funktion $negl$.

\subsection{EUF-naCMA-Sicherheitsexperiment}
\subsubsection{Visualisierung: EUF-naCMA-Sicherheitsexperiment}
\begin{tikzpicture}
  \node (A) at (0,5) {$\mathcal{C}_\text{EUF-naCMA}$};
  \node (B) at (5,5) {$\mathcal{A}$};
  \node[label={[align=center]below:$Vfy(pk, m^*, \sigma^*) = 1$?\\ $\land$ \\ $m^* \notin \{m_1, \dots, m_q\}$?}] (C) at (0,0) {};
  \node (D) at (5,0) {};

  \draw[dashed] (A) -- (C);
  \draw[dashed] (B) -- (D);

  \node[label={right:\begin{minipage}{5cm}\footnotesize\begin{itemize}
            \item{$q = q(k)$ Nachrichten}
            \item{$q$ Polynom}
          \end{itemize}\end{minipage}}] at (5,4) {};
  \node[label={[align=left]left:$(pk, sk) \leftarrow Gen(1^k)$\\$\forall i: \sigma_i \leftarrow Sign(sk, m_i)$}] at (0,3) {};

  \draw[->,shorten >=5pt, shorten <=5pt] (5,4) -- (0,3.5) node[midway, above, sloped] {$m_1, \dots, m_q$};

  \draw[->,shorten >=5pt, shorten <=5pt] (0,2.5) -- (5,2) node[midway, above, sloped] {$pk, \sigma_1, \dots, \sigma_q$};
  \draw[->,shorten >=5pt, shorten <=5pt] (5,1.5) -- (0,1) node[midway, above, sloped] {$m^*, \sigma^*$};
\end{tikzpicture}

$\mathcal{A}$ gewinnt, falls $Vfy(pk, m^*, \sigma^*) = 1$ \textbf{und} $m^* \notin \{m_1, \dots, m_q\}$


\subsubsection{Definition: EUF-naCMA}
Ein digitales Signaturverfahren $\Sigma = (Gen, Sign, Vfy)$ ist \textit{EUF-naCMA-sicher}, wenn für alle PPT $\mathcal{A}$ gilt, dass
\begin{flalign*}
   & \Pr[\mathcal{A} \text{ gewinnt EUF-naCMA-Experiment}]                                                                                \\
   & = \Pr[\mathcal{A}^{\mathcal{C}_\text{EUF-naCMA}} = (m^*, \sigma^*): Vfy(pk, m^*, \sigma^*) = 1 \land m^* \notin \{m_1, \dots, m_q\}] \\
   & \leq negl(k)
\end{flalign*}
für eine im Sicherheitsparameter $k$ vernachlässigbare Funktion $negl$.

\end{document}