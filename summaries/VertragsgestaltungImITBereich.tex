\documentclass[12pt,A4]{extarticle}	
\usepackage[table]{xcolor}

\newcommand{\lectureTitle}{Vertragsgestaltung im IT-Bereich}
\newcommand{\semester}{Wintersemester 2024/25}

\usepackage[a4paper,left=0.9cm,right=1cm,top=1.37cm,bottom=2.5cm]{geometry}
\usepackage[utf8]{inputenc}
\usepackage{xifthen}
\usepackage{cmbright}
\usepackage{fontawesome}
\usepackage[T1]{fontenc}
\usepackage{lastpage,lipsum}
\usepackage{hyperref}
\usepackage{transparent}
\usepackage{color}
\usepackage{fancyhdr}

\renewcommand*\familydefault{\sfdefault}
\setlength{\parindent}{0mm}

\definecolor{headerBg}{RGB}{11, 67, 158}
\definecolor{headerGrayColor}{RGB}{210, 210, 210}

\newcommand{\printTitle}{\textcolor{white}{\lectureTitle}\normalsize}
\newcommand{\printSubtitle}{
  \ifdefined\lectureSubtitle
    \textcolor{white}{\small{\lectureSubtitle}}\\
  \fi
}

\fancyhf{}
\pagestyle{fancy}
\fancyhead[C]{
  \fcolorbox{headerBg}{headerBg}{
    \hspace{0.6cm}\begin{minipage}[c][50pt][c]{\paperwidth}
      \begin{minipage}[c]{.7\textwidth}
        \ifdefined\titleSize
          \titleSize \printTitle\\
        \else
          \huge\printTitle\\
        \fi
        \printSubtitle
        \textcolor{headerGrayColor}{\small{\semester}}
      \end{minipage}%
      \begin{minipage}[c]{.2\textwidth}
        \raggedleft
        \textcolor{white}{
          \small{\href{mailto:mail@nilslambertz.de}{\textcolor{white}{\faicon{envelope}} mail@nilslambertz.de}}\\
          \href{https://github.com/nilslambertz/kit-zusammenfassungen}{\textcolor{white}{\faicon{github}} \small{nilslambertz}}}
      \end{minipage}
    \end{minipage}}
}
\renewcommand{\headrulewidth}{0pt}
\setlength{\headheight}{40pt}

\newlength{\oddmarginwidth}
\setlength{\oddmarginwidth}{1in+\hoffset+\oddsidemargin}
\newlength{\evenmarginwidth}
\setlength{\evenmarginwidth}{\evensidemargin+1in}
\fancyhfoffset[LO,RE]{\oddmarginwidth}
\fancyhfoffset[LE,RO]{\evenmarginwidth}
\cfoot{\thepage\ $/$ \pageref*{LastPage}}

\definecolor{highlightColor}{RGB}{66, 135, 245}
\newcommand{\highlight}[1]{\textcolor{highlightColor}{\textbf{#1}}}
\definecolor{gesetzLink}{RGB}{194, 74, 14}
\newcommand{\bgb}[2][]{\textbf{\textcolor{gesetzLink}{\href{https://www.gesetze-im-internet.de/bgb/__#2.html}{§ #2 \ifthenelse{\equal{#1}{}}{}{#1 }BGB}}}}
\newcommand{\bgbb}[2][]{\textbf{\textcolor{gesetzLink}{\href{https://www.gesetze-im-internet.de/bgb/__#2.html}{§§ #1 BGB}}}}
\newcommand{\prodHaftG}[2][]{\textbf{\textcolor{gesetzLink}{\href{https://www.gesetze-im-internet.de/prodhaftg/__#2.html}{§ #2 \ifthenelse{\equal{#1}{}}{}{#1 }ProdHaftG}}}}
\newcommand{\hgb}[2][]{\textbf{\textcolor{gesetzLink}{\href{https://www.gesetze-im-internet.de/hgb/__#2.html}{§ #2 \ifthenelse{\equal{#1}{}}{}{#1 }HGB}}}}
\newcommand{\urhg}[2][]{\textbf{\textcolor{gesetzLink}{\href{https://www.gesetze-im-internet.de/urhg/__#2.html}{§ #2 \ifthenelse{\equal{#1}{}}{}{#1 }UrhG}}}}

\def\contentsname{\empty}

\begin{document}

\disclaimer

\tableofcontents
\clearpage

\section{Vertragsschluss und Vertragstypen}
\subsection{Vertragstypen}
Um den Vertragstyp eines Vertrages zu bestimmen, muss folgendes geprüft werden:

\begin{enumerate}
  \item{Was sind die \textit{essentialia negotii} (wesentlichen Vertragsbestandteile)}
  \item{Welche Form muss der Vertrag haben?}
  \item{Wie endet der Vertrag?}
\end{enumerate}

\subsubsection{Kaufvertrag}
Bei einem \highlight{Kaufvertrag} gemäß \bgbb[433 ff.]{433} einigen sich die Parteien über die Übereignung eines Kaufgegenstandes und die Höhe eines Kaufpreises. 
Bei einer Leistungsstörung (z.B. Mangelhafte Leistung gemäß \bgb{437}) hat der Käufer gewisse Rechte.

\subsubsection{Werkvertrag}
Der \highlight{Werkvertrag} gemäß \bgbb[631 ff.]{631} wird für die \textbf{Herstellung eines Werkes} geschlossen, hierbei \textbf{ist der Erfolg geschuldet}. 
Bei mangelhafter Leistung hat der Besteller ebenfalls gemäß \bgb{634} gewisse Rechte.

\subsubsection{Dienstvertrag}
Bei einem \highlight{Dienstvertrag} gemäß \bgbb[611 ff.]{611} verpflichtet sich eine Partei zur Leistung von Diensten gegen Vergütung. Hierbei ist \textbf{kein Erfolg geschuldet}.
Es gibt keine besonderen Regelungen bei Leistungsstörungen.

\subsubsection{Mietvertrag}
Der \highlight{Mietvertrag} gemäß \bgbb[535 ff.]{535} regelt die entgeltliche Überlassung von Sachen. 
Bei mangelhafter Leistung hat der Mieter gemäß \bgb{536} gewisse Rechte.

\subsubsection{Werklieferungsvertrag}
Der \highlight{Werklieferungsvertrag} gemäß \bgb{650} wird zur Lieferung herzustellender oder zu erzeugender beweglicher Sachen geschlossen. Gemäß \bgb[Abs. 1]{650} wird das \textbf{Kaufvertragsrecht} angewendet.

\subsection{Vertragstypen im IT-Recht}
IT-Verträge können unter anderem über Standardsoftware, Individualsoftware, Pflege von Hard- und Software, Software as a Service, Open Source Software und Lizenzen geschlossen werden.
Für die unterschiedlichen Vertragstypen ergeben sich Unterschiede, z.B. im Hinblick auf Gewährleistungsansprüche. Daher stellt sich die Frage nach der Einordnung in die existierenden Vertragstypen.

\subsubsection{Grundfrage: Was ist Software?}
Bei \highlight{Software} handelt es sich nach geltender Rechtssprechung um das \textbf{Programm} sowie die zugehörige \textbf{Dokumentation}.
Für die verschiedenen Vertragsverhältnisse kommen folgende Vertragstypen in Betracht:

\begin{itemize}
  \item{\textbf{Standartsoftware} (Software, die nicht speziell für die Bedürfnisse des Kunden hergestellt wurde, kann aber für ihn angepasst worden sein):
    \begin{itemize}
      \item{typischerweise: \textbf{Kaufvertrag} (bei Überlassung auf Dauer gegen Entgelt)}
      \item{wenn weitere Anpassungsleistungen vereinbart wurden, kann es sich um einen \textbf{Werkvertrag} handeln}
    \end{itemize}
  }
  \item{\textbf{Individualsoftware} (Speziell auf die Bedürfnisse des Kunden zugeschnittene Software):
    \begin{itemize}
      \item{typischerweise: \textbf{Werkvertrag} (Schwerpunkt liegt auf dem individuell vereinbarten Werk)}
    \end{itemize}
  }
  \item{\textbf{Pflegeverträge} (Anpassung und Fehlerbeseitigung an einer Software):
    \begin{itemize}
      \item{typischerweise: \textbf{Dienstvertrag} (kein Erfolg geschuldet)}
      \item{für Updates und Upgrades: \textbf{Kaufvertrag}}
      \item{bei Instandsetzung: \textbf{Werkvertrag}}
    \end{itemize}
  }
  \item{\textbf{Software as a Service} (Software wird online auf Zeit bereitgestellt):
    \begin{itemize}
      \item{typischerweise: \textbf{Mietvertrag}}
    \end{itemize}
  }
  \item{\textbf{Open Source Software} (Software, die kostenfrei unter einer Open Source Lizenz veröffentlicht wurde):
    \begin{itemize}
      \item{typischerweise: \textbf{Schenkungsvertrag} (die meisten Gewährleistungsansprüche entfallen)}
      \item{bei Einbindung in kostenpflichtiges Gesamtprodukt: \textbf{Kaufvertrag}}
    \end{itemize}
  }
\end{itemize}

\section{Leistungsstörungsrecht, Haftung}

\subsection{Leistungsstörungen}

\subsubsection{Mangel}
Ein \highlight{Mangel} (\bgb{434}) liegt vor, wenn die Ist-Beschaffenheit von der Soll-Beschaffenheit abweicht:

\begin{itemize}
  \item{\textbf{Subjektiv} (\bgb[Abs. 2]{434}): Wenn sie sich für die \textbf{im Vertrag vorausgesetzte Verwendung} eignet}
  \item{\textbf{Objektiv} (\bgb[Abs. 3]{434}): Wenn sie sich für die \textbf{gewöhnliche Verwendung} eignet}
\end{itemize}

\subsubsection{Verzug}
Ein \highlight{Verzug} (\bgb{286}) liegt vor, wenn eine Leistung trotz Fälligkeit nicht erbracht wird.
Der Gläubiger hat Anspruch auf Schadensersatz, der durch den Verzug entstanden ist.

\subsubsection{Unmöglichkeit}
Eine \highlight{Unmöglichkeit} (\bgb{275}) liegt vor, wenn für den Schuldner die Leistungserbringung tatsächlich, praktisch oder persönlich unmöglich ist:

\begin{itemize}
  \item{Tatsächliche Unmöglichkeit (\bgb[Abs. 1]{275})}
  \item{Praktische (faktische) Unmöglichkeit (\bgb[Abs. 2]{275}): Aufwand im groben Missverhältnis zur Leistung}
  \item{Persönliche Unmöglichkeit (\bgb[Abs. 3]{275}): Bei Unzumutbarkeit}
\end{itemize}

\subsection{Haftung, Gewährleistung und Garantie}
\subsubsection{Gewährleistung vs Garantie}
Die \highlight{Gewährleistung} (\bgbb[437 ff.]{437}) sichert einem Käufer gesetzliche Ansprüche zu, wenn der \textbf{Zustand bei Übergabe} mangelhaft ist. Die Verjährungsfristen betragen in den meisten Fällen gemäß \bgb[Abs. 1]{438} zwei Jahre.\par
Die \highlight{Garantie} ist ein \textbf{freiwilliges Angebot} des Herstellers (das über die Gewährleistung hinaus geht), die Dauer und der Umfang sind nicht gesetzlich vorgeschrieben und können frei bestimmt werden.

\subsubsection{Produzentenhaftung}
Die Verpflichtung, ein Produkt so zu produzieren, dass fremde Rechtsgüter dadurch nicht verletzt werden, ergibt sich aus \bgb{823}. Bei Verletzung entsteht ein Schadensersatzanspruch.

\subsubsection{Produkthaftung}
Die Haftung für ein Produkt, das nicht die Sicherheit bietet, die unter Berücksichtigung aller Umstände erwartet werden kann, ist im \textbf{Produkthaftungsgesetz} geregelt. Folgende Anspruchsvoraussetzungen sind zu beachten:

\begin{enumerate}
  \item{Es muss ein Fehler gemäß \prodHaftG{3} vorliegen}
  \item{Die Verletzungshandlung muss gemäß \prodHaftG[Abs. 1]{1} in Form einer Tötung, einer Körper-/Gesundheitsverletzung oder einer Sachbeschädigung \textbf{an einer anderen Sache} erfolgt sein (Vermögensschäden zählen nicht!)}
  \item{Der Schaden muss auf den Produktfehler zurückzuführen sein}
\end{enumerate}

\section{Allgemeine Geschäftsbedingungen}
\highlight{Allgemeine Geschäftsbedingungen} (AGB) sind gemäß \bgbb[305 ff.]{305} alle

\begin{itemize}
  \item{für eine Vielzahl von Verträgen}
  \item{vorformulierte Vertragsbedingungen}
  \item{die eine Vertragspartei der anderen Vertragspartei bei Abschluss eines Vertrags stellt}
\end{itemize}

\subsection{Einbeziehung von AGB in den Vertrag}
AGB werden gemäß \bgb[Abs. 2]{305} nur dann Teil des Vertrags, wenn bei Vertragsschluss

\begin{itemize}
  \item{ausdrücklich auf sie hingewiesen wurde}
  \item{eine zumutbare Möglichkeit zur Kenntnisnahme bestand}
  \item{die andere Partei (ausdrücklich oder konkludent) einverstanden ist}
\end{itemize}

\subsection{Unwirksamkeit von AGB-Klauseln}
\subsubsection{Inhaltskontrolle}
Gemäß \bgb[Abs. 1]{307} (\textbf{Generalklausel}) sind AGB-Klauseln unwirksam, wenn sie einen Vertragspartner \textbf{unangemessen benachteiligen}. Dies gilt auch, wenn die Bestimmungen \textbf{nicht klar und verständlich} sind.

\subsubsection{Klauselverbote}
\bgb{309} beschreibt \textbf{Klauselverbote ohne Wertungsmöglichkeit}, diese sind immer unwirksam. Bei \textbf{Klauselverboten mit Wertungsmöglichkeit} (\bgb{308}) ist eine Interessenabwägung notwendig.

\subsubsection{Überraschende Klauseln}
Überraschende Klauseln werden gemäß \bgb{305c} nicht Vertragsbestandteil.

\subsection{Besonderheiten zwischen Unternehmern}
Für Verträge zwischen Unternehmern ergeben sich besondere Regelungen. Gemäß \bgb[Abs. 1]{310} gelten die Klauselverbote nur eingeschränkt, auch die Regelungen für die Einbeziehung von AGB in den Vertrag finden keine Anwendung.

\section{Handels- und Gesellschaftsrecht}

\subsection{Handelsrecht}
\subsubsection{Definition: Kaufmann}
\textbf{Kaufmann} ist gemäß \hgb[Abs. 1]{1}, wer ein Handelsgewerbe betreibt.

\subsubsection{Definition: Gewerbe}

\begin{itemize}
  \item{\textbf{selbstständig} (in Abgrenzung zum Angestellten)}
  \item{auf Dauer angelegt}
  \item{it der \textbf{Absicht} zur Gewinnerzielung}
  \item{am Markt}
  \item{nicht: künstlerisch, wissenchaftlich oder freier Beruf}
\end{itemize}

\subsubsection{Definition: Firma}
Eine Firma ist gemäß \hgb[Abs. 1]{17} der \textbf{Name}, unter dem der Kaufmann seine Geschäfte betreibt.

\subsubsection{Besonderheiten}

\begin{itemize}
  \item{\textbf{Keine Herabsetzung von Vertragsstrafen} wenn unverhältnismäßig hoch}
  \item{\textbf{Verzicht auf Formvorschriften}, z.B. bei Schuldversprechen, Bürgschaft, etc.}
  \item{\textbf{Einschränkungen des Verbraucherschutzes} (aufgrund der Geschäftserfahrung)}
  \item{\textbf{Anwendung des AGB-Rechts eingeschränkt}}
\end{itemize}

\subsection{Gesellschaftsrecht}
Eine \highlight{Gesellschaft} ist gemäß \bgb{705} jeder vertragliche Zusammenschluss von zwei oder mehr Personen zur Förderung eines vereinbarten gemeinsamen Zwecks.

\section{Urheberrecht, Open Source Lizenzen}
\subsection{Urheberrecht}
\subsubsection{Schutzgegenstand}
Die \textbf{geschützten Werke} sind in \urhg{2} \textbf{nicht abschließend} aufgezählt, darunter Schriftwerke, Werke der Musik und Computerprogramme.\par
Es muss sich gemäß \urhg[Abs. 2]{2} um eine \highlight{persönliche geistige Schöpfung} handeln.

\subsubsection{Urherberpersönlichkeitsrechte}
Zu den \highlight{Urherberpersönlichkeitsrechten} gehört das

\begin{itemize}
  \item{\textbf{Veröffentlichungsrecht} gemäß \urhg{12}}
  \item{\textbf{Anerkennungsrecht} gemäß \urhg{13}}
  \item{\textbf{Entstellungsverbot} gemäß \urhg{14}}
\end{itemize}

Die Urheberpersönlichkeitsrechte können gemäß \urhg[Abs. 1]{29} \textbf{nicht übertragen} werden.

\subsubsection{Verwertungsrechte}
Zu den \highlight{Verwertungsrechten} gehört das

\begin{itemize}
  \item{\textbf{Vervielfältigungsrecht} gemäß \urhg{16}}
  \item{\textbf{Verbreitungsrecht} gemäß \urhg{17}}
  \item{\textbf{Bearbeitungsrecht} gemäß \urhg{23}}
  \item{sonstige \textbf{Verwertungsrecht} gemäß \urhg{15}}
\end{itemize}

\subsubsection{Einräumung von Nutzungsrechten}
Die \highlight{Einräumung von Nutzungsrechten} ist in \urhg{31} geregelt. Es gibt kann auf verschiedene Arten eingeräumt werden, z.B. als ausschließliches Nutzungsrecht oder räumlich, zeitlich oder inhaltlich beschränkt.

\subsubsection{Zweckübertragungsprinzip}
Das \highlight{Zweckübertragungsprinzip} gemäß \urhg[Abs. 5]{31} besagt, dass die Nutzungsarten sich nach dem zugrunde gelegten Vertragszweck richten, wenn sie nicht ausdrücklich einzeln im Vertrag geregelt wurden.\par
Daher wird im Falle von fehlenden Regelungen im Vertrag angenommen, was ``vernünftige Vertragspartner'' vereinbart hätten.

\subsubsection{Erschöpfungsgrundsatz}
Der \highlight{Erschöpfungsgrundsatz} gemäß \urhg[Abs. 2]{17} besagt, dass die Weiterverbreitung eines Werkes zulässig ist, wenn es mit Zustimmung des Rechteinhabers in der EU in Verkehr gebracht wurde.\par
Wenn das Werk vorher nicht in der EU in Verkehr gebracht wurde, kann der Handel eine Marken-/ oder Urheberrechtsverletzung darstellen.

\subsubsection{Regelungen bei mehreren Urhebern}
Wenn mehrere Personen ein Werk gemeinsam geschaffen haben, sind die gemäß \urhg[Abs. 1]{8} \highlight{Miturheber}. Sie erhalten gemäß \urhg{8} gewisse (Mitbestimmungs-)Rechte bei der Nutzung des Werks.

\subsubsection{Programmierer in Arbeits- und Dienstverhältnissen}
Gemäß \urhg{69b} ist \textbf{ausschließlich der Arbeitgeber} zur Ausübung aller vermögensrechtlichen Befugnisse an dem Computerprogramm berechtigt, das von einem Arbeitnehmer für ihn geschaffen wurde.

\subsubsection{Angemessenheit der Vergütung}
Gemäß \urhg{32} hat der Urheber Anspruch auf \textbf{angemessene Vergütung}.\par
Gemäß \urhg{32a} besteht dieser Anspruch besteht auch Jahre nach Veröffentlichung des Werkes noch, z.B. wenn es unerwartet hohe Bekanntheit erfährt.

\subsubsection{Urheberrecht bei Computerprogrammen}
Urheberrechtlich geschützt sind gemäß \urhg[Abs. 1]{69a} der \textbf{Programmcode und die Entwurfsmaterialien}.\par
Ideen und Grundsätze die dem Programm zugrunde liegen, sind gemäß \urhg[Abs. 2]{69a} nicht geschützt.\par
Handlungen wie die \textbf{Vervielfältigung} (auch im RAM), \textbf{Verbreitung} und \textbf{Bearbeitung} erfordern gemäß \urhg{69c} die Zustimmung des Urhebers.\par
Ausnahmen davon sind in \urhg{69d} geregelt, z.B. die bestimmungsgemäße Benutzung, Sicherheitskopien und Untersuchungen.

\subsubsection{Ansprüche bei Rechtsverletzung}
Bei Rechtsverletzung entsteht z.B. ein

\begin{itemize}
  \item{\textbf{Unterlassungsanspruch} (\urhg[Abs. 1]{97})}
  \item{\textbf{Schadensersatzanspruch} (\urhg[Abs. 2]{97})}
  \item{\textbf{Beseitigungsanspruch} (\urhg{98})}
  \item{\textbf{Entschädigungsanspruch} (\urhg{100})}
  \item{\textbf{Auskuftsanspruch} (\urhg{101})}
\end{itemize}

\subsection{Open Source Lizenzen}
Open Source Software zeichnet sich dadurch aus, dass typischerweise

\begin{itemize}
  \item{keine Lizenzgebühr anfällt}
  \item{die Verbreitung des Quellcodes und Binarcodes erlaubt ist}
  \item{die Modifikation und Herstellung abgeleiteter Werke erlaubt ist}
  \item{keine Beschränkung auf Nutzergruppen und Anwendungsgebiete besteht}
\end{itemize}

\subsubsection{Lizenzrechte}
Mögliche Lizenzrechte sind z.B. \textbf{Vervielfältigung}, \textbf{Bearbeitung} und \textbf{Weiterverbreitung}.

\subsubsection{Lizenzpflichten}
Mögliche Lizenzpflichten sind z.b. \textbf{Lizenz- oder Änderungshinweise}, \textbf{Acknowledgement} oder \textbf{Copyleft}.

\subsubsection{Einbeziehung von OSS-Lizenzen}
OSS-Lizenzen gelten als \textbf{AGB}. Die Rechte und Pflichten werden jedem Nutzer der Software eingeräumt, d.h. sowohl dem Unternehmen, das Software programmiert als auch dem Kunden, der die Software später nutzt.

\subsubsection{Rechtsfolgen bei Verstößen}
Bei Verstößen gegen die Lizenzbedingungen besteht ein \textbf{Unterlassungsanspruch}, \textbf{Auskunftsanspruch} und \textbf{Schadensersatzanspruch}.

\section{Projektvertrag}
Es gibt keinen einheitlichen ``IT-Vertrag'', sondern meist eine Mischung aus Werk-, Dienst-, Miet- und Kaufvertrag.

\subsection{Vertragsgegenstand und Vertragsgrundlagen}
Zur Definition der Anforderungen (also was geschuldet wird), kann ein Lasten- und Pflichtenheft erstellt werden:

\begin{itemize}
  \item{\highlight{Lastenheft}: Die vom \textbf{Auftraggeber} festgelegte Gesamtheit der Forderungen an die Lieferungen und Leistungen des Auftragnehmers}
  \item {\highlight{Pflichtenheft}: Die vom \textbf{Auftragnehmer} erarbeiteten Realisierungsvorhaben aufgrund der Umsetzung des vom Auftraggeber vorgegebenen Lastenhefts}
\end{itemize}

\subsection{Einräumung urheberrechtlicher Nutzungsbefugnisse}
Urheberrechtliche Fragen müssen geklärt werden, darunter

\begin{itemize}
  \item{(nicht) ausschließliche Nutzungsbefugnis des Auftraggebers}
  \item{(zeitlich/räumlich) (un)begrenzte Nutzungsbefugnis}
  \item{Begrenzung nach Nutzungsumfang möglich (z.B. Anzahl der Benutzer, Lizenzen)}
  \item{Regeln zu Sicherheitskopien, Weitergabe, Dekompilierung}
\end{itemize}

\subsection{Vergütung}

\begin{itemize}
  \item{Aufwandsentschädigung (z.B. pro Tag)}
  \item{Pauschalvergütung}
  \item{Kombination aus beidem}
\end{itemize}

\subsection{Service-Level-Agreement (SLA)}
SLAs regeln die Qualität der Leistungen gegenüber dem Servicenehmer, z.B. Verfügbarkeit, Reporting, etc. Es gibt keine gesetzliche Qualitätsdefinition, geschuldet wird gemäß \bgb[Abs. 1]{243} ``Leistung mittlerer Art und Güte''. Dazu kann geregelt werden:

\begin{itemize}
  \item{Genaue Definition der Service Level (z.B. ``98\% Verfügbarkeit 24h/7 Tage pro Woche'')}
  \item{Zielwerte, Messmethoden, Messpunkte}
  \item{Art des Berichts, automatisches Reporting}
  \item{Prozedur bei Support und Service Desk}
\end{itemize}

Rechtsfolgen bei Nichteinhaltung können Minderung der Vergütung, Vertragsstrafen, Schadenersatz, Kündigung beinhalten.

\subsection{Change-Request}
\textbf{Change Request}-Klauseln regeln das Vorgehen bei Änderungs- und Erweiterungswünschen, z.B. das Formale Vorgehen, Aufstellung der Kosten, Änderung des Zeitplans, etc.

\subsection{Regelungen zur Abnahme/Abnahmeverfahren}
Für die Abnahme können geregelt werden:

\begin{itemize}
  \item{Testphase, Fehlerkategorien, Teilnehmer}
  \item{Förmliche/Konkludente Abnahme}
  \item{Teilabnahme, Abnahmefiktion}
\end{itemize}

\subsection{Gewährleistung}
Zur Gewährleistung können geregelt werden:

\begin{itemize}
  \item{Unverzügliche Fehlerermittlung}
  \item{Genaue Fehlerbeschreibung}
  \item{Zurverfügungstellen von Testdaten}
  \item{Konsequenzen verspäteter Rüge}
\end{itemize}

\subsection{Haftung}
Gesetzliche Haftung gilt nur für für Vorsatz und Fahrlässigkeit. Gemäß \bgb[Nr. 7]{309} kann die Haftung \textbf{für grobe Fahrlässigkeit nicht beschränkt werden}.\par
Die Kardinalpflichten (wesentlichen Pflichten) des Vertrags können gemäß \bgb[Abs. 2 Nr. 2]{307} ebenfalls nicht erheblich beschränkt werden.

\subsection{Quellcodevereinbarung}
Vereinbarungen zum Quellcode sind ebenfalls notwendig:

\begin{itemize}
  \item{z.B. historische Versionen des Quellcodes}
  \item{Enwicklungsumgebungen und Werkzeuge}
  \item{Daten und Datenbanken}
\end{itemize}

\section{Projektmanagement}
\subsection{Wasserfallmodell}
Das Wasserfallmodell läuft sequentiell in den folgenden Stufen ab:

\begin{enumerate}
  \item{Leistungsbeschreibung: Lastenheft/Pflichtenheft}
  \item{Projektablauf: Change Requests}
  \item{Abnahme: Zwischenabnahme/Tests/Förmliche Abnahme}
  \item{Inbetriebnahme: Nutzerakzeptanz}
  \item{Produktivbetrieb: Fehlerbeseitigung/Weiterentwicklung}
\end{enumerate}

\subsection{Agile Verträge: SCRUM}
\subsubsection{Stakeholder}

\begin{itemize}
  \item{\textbf{Product Owner}:
    \begin{itemize}
      \item{verantwortlich für den \textbf{Product Backlog}}
      \item{definiert und priorisiert Anforderungen}
    \end{itemize}
  }
  \item{\textbf{Scrum Master}:
    \begin{itemize}
      \item{verantwortlich für die Einhaltung der \textbf{SCRUM-Regeln}}
      \item{unterstützt das Team und den Product Owner}
    \end{itemize}
  }
  \item{\textbf{Scrum Team}:
    \begin{itemize}
      \item{Entwickler, verantwortlich für die \textbf{Umsetzung der Anforderungen}}
      \item{organisieren sich selbst}
    \end{itemize}
  }
\end{itemize}

\subsubsection{Ablauf}
\begin{enumerate}
  \item{Product Owner erstellt und priorisiert den \textbf{Product Backlog}}
  \item{Scrum Team wählt Aufgaben aus, die im nächsten Sprint umgesetzt werden}
  \item{Das Team arbeitet an den Aufgaben}
  \item{Sprint-Review: Feedback und Bericht}
  \item{Prozess beginnt erneut}
\end{enumerate}

\end{document}