\documentclass[12pt,A4]{extarticle}	
\usepackage[table]{xcolor}

\newcommand{\lectureTitle}{Vertragsgestaltung im IT-Bereich [WIP]}
\newcommand{\semester}{Wintersemester 2024/25}

\usepackage[a4paper,left=0.9cm,right=1cm,top=1.37cm,bottom=2.5cm]{geometry}
\usepackage[utf8]{inputenc}
\usepackage{xifthen}
\usepackage{cmbright}
\usepackage{fontawesome}
\usepackage[T1]{fontenc}
\usepackage{lastpage,lipsum}
\usepackage{hyperref}
\usepackage{transparent}
\usepackage{color}
\usepackage{fancyhdr}

\renewcommand*\familydefault{\sfdefault}
\setlength{\parindent}{0mm}

\usepackage{transparent}
\usepackage{color}
\usepackage{fancyhdr}

\definecolor{headerBg}{RGB}{11, 67, 158}
\definecolor{headerGrayColor}{RGB}{210, 210, 210}

\pagestyle{fancy}
\fancyhead[C]{
  \fcolorbox{headerBg}{headerBg}{
    \hspace{0.6cm}\begin{minipage}[c][50pt][c]{\paperwidth}
      \begin{minipage}[c]{.45\textwidth}
        \huge{\textcolor{white}{Vorlesung}}\normalsize\\
        \textcolor{headerGrayColor}{\small{Wintersemester 2022/23}}
      \end{minipage}%
      \begin{minipage}[c]{.45\textwidth}
        \raggedleft
        \textcolor{white}{
          \small{\href{https://nilslambertz.de/}{nilslambertz.de}}\\
          \href{https://github.com/nilslambertz/}{\textcolor{white}{\faicon{github}} \small{nilslambertz}}}
      \end{minipage}
    \end{minipage}}
}
\renewcommand{\headrulewidth}{0pt}
\setlength{\headheight}{40pt}

\newlength{\oddmarginwidth}
\setlength{\oddmarginwidth}{1in+\hoffset+\oddsidemargin}
\newlength{\evenmarginwidth}
\setlength{\evenmarginwidth}{\evensidemargin+1in}
\fancyhfoffset[LO,RE]{\oddmarginwidth}
\fancyhfoffset[LE,RO]{\evenmarginwidth}
\cfoot{\thepage\ $/$ \pageref*{LastPage}}

\definecolor{highlightColor}{RGB}{66, 135, 245}
\newcommand{\highlight}[1]{\textcolor{highlightColor}{\textbf{#1}}}
\definecolor{gesetzLink}{RGB}{194, 74, 14}
\newcommand{\bgb}[2][]{\textbf{\textcolor{gesetzLink}{\href{https://www.gesetze-im-internet.de/bgb/__#2.html}{§ #2 \ifthenelse{\equal{#1}{}}{}{#1 }BGB}}}}
\newcommand{\bgbb}[2][]{\textbf{\textcolor{gesetzLink}{\href{https://www.gesetze-im-internet.de/bgb/__#2.html}{§§ #1 BGB}}}}
\newcommand{\prodHaftG}[2][]{\textbf{\textcolor{gesetzLink}{\href{https://www.gesetze-im-internet.de/prodhaftg/__#2.html}{§ #2 \ifthenelse{\equal{#1}{}}{}{#1 }ProdHaftG}}}}

\def\contentsname{\empty}

\begin{document}

\disclaimer

\tableofcontents
\clearpage

\section{Vertragsschluss und Vertragstypen}
\subsection{Vertragstypen}
Um den Vertragstyp eines Vertrages zu bestimmen, muss folgendes geprüft werden:

\begin{enumerate}
  \item{Was sind die \textit{essentialia negotii} (wesentlichen Vertragsbestandteile)}
  \item{Welche Form muss der Vertrag haben?}
  \item{Wie endet der Vertrag?}
\end{enumerate}

\subsubsection{Kaufvertrag}
Bei einem \highlight{Kaufvertrag} gemäß \bgbb[433 ff.]{433} einigen sich die Parteien über die Übereignung eines Kaufgegenstandes und die Höhe eines Kaufpreises. 
Bei einer Leistungsstörung (z.B. Mangelhafte Leistung gemäß \bgb{437}) hat der Käufer gewisse Rechte.

\subsubsection{Werkvertrag}
Der \highlight{Werkvertrag} gemäß \bgbb[631 ff.]{631} wird für die \textbf{Herstellung eines Werkes} geschlossen, hierbei \textbf{ist der Erfolg geschuldet}. 
Bei mangelhafter Leistung hat der Besteller ebenfalls gemäß \bgb{634} gewisse Rechte.

\subsubsection{Dienstvertrag}
Bei einem \highlight{Dienstvertrag} gemäß \bgbb[611 ff.]{611} verpflichtet sich eine Partei zur Leistung von Diensten gegen Vergütung. Hierbei ist \textbf{kein Erfolg geschuldet}.
Es gibt keine besonderen Regelungen bei Leistungsstörungen.

\subsubsection{Mietvertrag}
Der \highlight{Mietvertrag} gemäß \bgbb[535 ff.]{535} regelt die entgeltliche Überlassung von Sachen. 
Bei mangelhafter Leistung hat der Mieter gemäß \bgb{536} gewisse Rechte.

\subsubsection{Werklieferungsvertrag}
Der \highlight{Werklieferungsvertrag} gemäß \bgb{650} wird zur Lieferung herzustellender oder zu erzeugender beweglicher Sachen geschlossen. Gemäß \bgb[Abs. 1]{650} wird das \textbf{Kaufvertragsrecht} angewendet.

\subsection{Vertragstypen im IT-Recht}
IT-Verträge können unter anderem über Standardsoftware, Individualsoftware, Pflege von Hard- und Software, Software as a Service, Open Source Software und Lizenzen geschlossen werden.
Für die unterschiedlichen Vertragstypen ergeben sich Unterschiede, z.B. im Hinblick auf Gewährleistungsansprüche. Daher stellt sich die Frage nach der Einordnung in die existierenden Vertragstypen.

\subsubsection{Grundfrage: Was ist Software?}
Bei \highlight{Software} handelt es sich nach geltender Rechtssprechung um das \textbf{Programm} sowie die zugehörige \textbf{Dokumentation}.
Für die verschiedenen Vertragsverhältnisse kommen folgende Vertragstypen in Betracht:

\begin{itemize}
  \item{\textbf{Standartsoftware} (Software, die nicht speziell für die Bedürfnisse des Kunden hergestellt wurde, kann aber für ihn angepasst worden sein):
    \begin{itemize}
      \item{typischerweise: \textbf{Kaufvertrag} (bei Überlassung auf Dauer gegen Entgelt)}
      \item{wenn weitere Anpassungsleistungen vereinbart wurden, kann es sich um einen \textbf{Werkvertrag} handeln}
    \end{itemize}
  }
  \item{\textbf{Individualsoftware} (Speziell auf die Bedürfnisse des Kunden zugeschnittene Software):
    \begin{itemize}
      \item{typischerweise: \textbf{Werkvertrag} (Schwerpunkt liegt auf dem individuell vereinbarten Werk)}
    \end{itemize}
  }
  \item{\textbf{Pflegeverträge} (Anpassung und Fehlerbeseitigung an einer Software):
    \begin{itemize}
      \item{typischerweise: \textbf{Dienstvertrag} (kein Erfolg geschuldet)}
      \item{für Updates und Upgrades: \textbf{Kaufvertrag}}
      \item{bei Instandsetzung: \textbf{Werkvertrag}}
    \end{itemize}
  }
  \item{\textbf{Software as a Service} (Software wird online auf Zeit bereitgestellt):
    \begin{itemize}
      \item{typischerweise: \textbf{Mietvertrag}}
    \end{itemize}
  }
  \item{\textbf{Open Source Software} (Software, die kostenfrei unter einer Open Source Lizenz veröffentlicht wurde):
    \begin{itemize}
      \item{typischerweise: \textbf{Schenkungsvertrag} (die meisten Gewährleistungsansprüche entfallen)}
      \item{bei Einbindung in kostenpflichtiges Gesamtprodukt: \textbf{Kaufvertrag}}
    \end{itemize}
  }
\end{itemize}

\section{Leistungsstörungsrecht, Haftung}

\subsection{Leistungsstörungen}

\subsubsection{Mangel}
Ein \highlight{Mangel} (\bgb{434}) liegt vor, wenn die Ist-Beschaffenheit von der Soll-Beschaffenheit abweicht:

\begin{itemize}
  \item{\textbf{Subjektiv} (\bgb[Abs. 2]{434}): Wenn sie sich für die \textbf{im Vertrag vorausgesetzte Verwendung} eignet}
  \item{\textbf{Objektiv} (\bgb[Abs. 3]{434}): Wenn sie sich für die \textbf{gewöhnliche Verwendung} eignet}
\end{itemize}

\subsubsection{Verzug}
Ein \highlight{Verzug} (\bgb{286}) liegt vor, wenn eine Leistung trotz Fälligkeit nicht erbracht wird.
Der Gläubiger hat Anspruch auf Schadensersatz, der durch den Verzug entstanden ist.

\subsubsection{Unmöglichkeit}
Eine \highlight{Unmöglichkeit} (\bgb{275}) liegt vor, wenn für den Schuldner die Leistungserbringung tatsächlich, praktisch oder persönlich unmöglich ist:

\begin{itemize}
  \item{Tatsächliche Unmöglichkeit (\bgb[Abs. 1]{275})}
  \item{Praktische (faktische) Unmöglichkeit (\bgb[Abs. 2]{275}): Aufwand im groben Missverhältnis zur Leistung}
  \item{Persönliche Unmöglichkeit (\bgb[Abs. 3]{275}): Bei Unzumutbarkeit}
\end{itemize}

\subsection{Haftung, Gewährleistung und Garantie}
\subsubsection{Produzentenhaftung}
Die Verpflichtung, ein Produkt so zu produzieren, dass fremde Rechtsgüter dadurch nicht verletzt werden, ergibt sich aus \bgb{823}. Bei Verletzung entsteht ein Schadensersatzanspruch.

\subsubsection{Produkthaftung}
Die Haftung für ein Produkt, das nicht die Sicherheit bietet, die unter Berücksichtigung aller Umstände erwartet werden kann, ist im \textbf{Produkthaftungsgesetz} geregelt. Folgende Anspruchsvoraussetzungen sind zu beachten:

\begin{enumerate}
  \item{Es muss ein Fehler gemäß \prodHaftG{3} vorliegen}
  \item{Die Verletzungshandlung muss gemäß \prodHaftG[Abs. 1]{1} in Form einer Tötung, einer Körper-/Gesundheitsverletzung oder einer Sachbeschädigung \textbf{an einer anderen Sache} erfolgt sein (Vermögensschäden zählen nicht!)}
  \item{Der Schaden muss auf den Produktfehler zurückzuführen sein}
\end{enumerate}

\section{Allgemeine Geschäftsbedingungen}
\highlight{Allgemeine Geschäftsbedingungen} (AGB) sind gemäß \bgbb[305 ff.]{305} alle

\begin{itemize}
  \item{für eine Vielzahl von Verträgen}
  \item{vorformulierte Vertragsbedingungen}
  \item{die eine Vertragspartei der anderen Vertragspartei bei Abschluss eines Vertrags stellt}
\end{itemize}

\subsection{Einbeziehung von AGB in den Vertrag}
AGB werden gemäß \bgb[Abs. 2]{305} nur dann Teil des Vertrags, wenn bei Vertragsschluss

\begin{itemize}
  \item{ausdrücklich auf sie hingewiesen wurde}
  \item{eine zumutbare Möglichkeit zur Kenntnisnahme bestand}
  \item{die andere Partei (ausdrücklich oder konkludent) einverstanden ist}
\end{itemize}

\subsection{Unwirksamkeit von AGB-Klauseln}
\subsubsection{Inhaltskontrolle}
Gemäß \bgb[Abs. 1]{307} (\textbf{Generalklausel}) sind AGB-Klauseln unwirksam, wenn sie einen Vertragspartner \textbf{unangemessen benachteiligen}. Dies gilt auch, wenn die Bestimmungen \textbf{nicht klar und verständlich} sind.

\subsubsection{Klauselverbote}
\bgb{309} beschreibt \textbf{Klauselverbote ohne Wertungsmöglichkeit}, diese sind immer unwirksam. Bei \textbf{Klauselverboten mit Wertungsmöglichkeit} (\bgb{308}) ist eine Interessenabwägung notwendig.

\subsubsection{Überraschende Klauseln}
Überraschende Klauseln werden gemäß \bgb{305c} nicht Vertragsbestandteil.

\subsection{Besonderheiten zwischen Unternehmern}
Für Verträge zwischen Unternehmern ergeben sich besondere Regelungen. Gemäß \bgb[Abs. 1]{310} gelten die Klauselverbote nur eingeschränkt, auch die Regelungen für die Einbeziehung von AGB in den Vertrag finden keine Anwendung.

\end{document}