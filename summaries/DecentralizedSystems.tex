\documentclass[12pt,A4]{extarticle}	
\usepackage{filecontents}

\begin{filecontents}{\jobname.bib}
  @article{DistributedSystemVanSteenTanenbaum,
  author       = {Maarten van Steen and
                  Andrew S. Tanenbaum},
  title        = {A brief introduction to distributed systems},
  journal      = {Computing},
  volume       = {98},
  number       = {10},
  pages        = {967--1009},
  year         = {2016},
  url          = {https://doi.org/10.1007/s00607-016-0508-7},
  doi          = {10.1007/s00607-016-0508-7},
  timestamp    = {Thu, 14 Oct 2021 09:12:09 +0200},
  biburl       = {https://dblp.org/rec/journals/computing/SteenT16.bib},
  bibsource    = {dblp computer science bibliography, https://dblp.org}
}
@article{lamportTimeClocks,
author = {Lamport, Leslie},
title = {Time, clocks, and the ordering of events in a distributed system},
year = {1978},
issue_date = {July 1978},
publisher = {Association for Computing Machinery},
address = {New York, NY, USA},
volume = {21},
number = {7},
issn = {0001-0782},
url = {https://doi.org/10.1145/359545.359563},
doi = {10.1145/359545.359563},
abstract = {The concept of one event happening before another in a distributed system is examined, and is shown to define a partial ordering of the events. A distributed algorithm is given for synchronizing a system of logical clocks which can be used to totally order the events. The use of the total ordering is illustrated with a method for solving synchronization problems. The algorithm is then specialized for synchronizing physical clocks, and a bound is derived on how far out of synchrony the clocks can become.},
journal = {Commun. ACM},
month = {jul},
pages = {558–565},
numpages = {8},
keywords = {clock synchronization, computer networks, distributed systems, multiprocess systems}
}
\end{filecontents}

\newcommand{\lectureTitle}{Decentralized Systems [WIP]}
\newcommand{\lectureSubtitle}{Fundamentals, Modeling, and Applications}
\newcommand{\semester}{Sommersemester 2024}

\newcommand{\titleSize}{\LARGE}

\usepackage[a4paper,left=0.9cm,right=1cm,top=1.37cm,bottom=2.5cm]{geometry}
\usepackage[utf8]{inputenc}
\usepackage{xifthen}
\usepackage{cmbright}
\usepackage{fontawesome}
\usepackage[T1]{fontenc}
\usepackage{lastpage,lipsum}
\usepackage{hyperref}
\usepackage{transparent}
\usepackage{color}
\usepackage{fancyhdr}

\renewcommand*\familydefault{\sfdefault}
\setlength{\parindent}{0mm}

\definecolor{headerBg}{RGB}{11, 67, 158}
\definecolor{headerGrayColor}{RGB}{210, 210, 210}

\newcommand{\printTitle}{\textcolor{white}{\lectureTitle}\normalsize}
\newcommand{\printSubtitle}{
  \ifdefined\lectureSubtitle
    \textcolor{white}{\small{\lectureSubtitle}}\\
  \fi
}

\fancyhf{}
\pagestyle{fancy}
\fancyhead[C]{
  \fcolorbox{headerBg}{headerBg}{
    \hspace{0.6cm}\begin{minipage}[c][50pt][c]{\paperwidth}
      \begin{minipage}[c]{.7\textwidth}
        \ifdefined\titleSize
          \titleSize \printTitle\\
        \else
          \huge\printTitle\\
        \fi
        \printSubtitle
        \textcolor{headerGrayColor}{\small{\semester}}
      \end{minipage}%
      \begin{minipage}[c]{.2\textwidth}
        \raggedleft
        \textcolor{white}{
          \small{\href{mailto:mail@nilslambertz.de}{\textcolor{white}{\faicon{envelope}} mail@nilslambertz.de}}\\
          \href{https://github.com/nilslambertz/kit-zusammenfassungen}{\textcolor{white}{\faicon{github}} \small{nilslambertz}}}
      \end{minipage}
    \end{minipage}}
}
\renewcommand{\headrulewidth}{0pt}
\setlength{\headheight}{40pt}

\newlength{\oddmarginwidth}
\setlength{\oddmarginwidth}{1in+\hoffset+\oddsidemargin}
\newlength{\evenmarginwidth}
\setlength{\evenmarginwidth}{\evensidemargin+1in}
\fancyhfoffset[LO,RE]{\oddmarginwidth}
\fancyhfoffset[LE,RO]{\evenmarginwidth}
\cfoot{\thepage\ $/$ \pageref*{LastPage}}


\definecolor{highlightColor}{RGB}{66, 135, 245}
\newcommand{\highlight}[1]{\textcolor{highlightColor}{\textbf{#1}}}

\def\contentsname{\empty}

\begin{document}

\disclaimer

\tableofcontents
\clearpage

\section{Introduction}
\subsection{What is a distributed system?}
\subsubsection{Characteristics by van Steen and Tanenbaum}
``\textit{A distributed system is a collection of autonomous computing elements that appears to its users as a single coherent system.}'' (\cite{DistributedSystemVanSteenTanenbaum}), this results in two characteristics:
\begin{enumerate}
  \item{``\textbf{Collection of autonomous computing elements}'': ``\textit{In practice, nodes are programmed to achieve common goals, which are realized by exchanging messages with each other}'' \\(\cite{DistributedSystemVanSteenTanenbaum})}
  \item{``\textbf{Appears as a single coherent system}'': Appears as a single large system}
\end{enumerate}

\subsubsection{Aspects of characteristic 1}
\begin{itemize}
  \item{``\textit{we cannot assume that there is something like a global clock}'' (\cite{DistributedSystemVanSteenTanenbaum}), therefore the \textbf{synchronization and coordination} between participants must be worked out}
  \item{``\textit{The fact that we are dealing with a collection of nodes implies that we may also need to manage the membership and organization of that collection}'' (\cite{DistributedSystemVanSteenTanenbaum}), therefore we need to think about identities and possible access-restrictions}
\end{itemize}

\subsubsection{Aspects of characteristic 2}
\begin{itemize}
  \item{``\textit{To assist the development of distributed applications, distributed systems are often organized to have a separate layer of software that is logically placed on top of the respective operating systems of the computers that are part of the system [...] leading to what is known as middleware}'' (\cite{DistributedSystemVanSteenTanenbaum})}
\end{itemize}

\subsubsection{Observations}
\begin{itemize}
  \item{Distributing tasks and aggregating a result from them is not easy at all}
  \item{The coordination of those tasks is still \textbf{centralized}}
\end{itemize}

\subsection{What makes a distributed system a decentralized system?}
According to \textbf{ISO/TC 307}: ``\textit{distributed system wherein control is distributed among the persons or organizations participating in the operation of the system}''

\subsubsection{Three types of Decentralization (Vitalik Buterin)}
\href{https://medium.com/@VitalikButerin/the-meaning-of-decentralization-a0c92b76a274}{Vitalik Buterin defines three types of Decentralization}:
\begin{itemize}
  \item{\textbf{Architectural (de)centralization}: How many \textbf{physical computers} is a system made up of? How many of those computers can it tolerate breaking down at any single time?}
  \item{\textbf{Political (de)centralization}: How many \textbf{individuals or organizations} ultimately control the computers that the system is made up of?}
  \item{\textbf{Logical (de)centralization}: Does the \textbf{interface and data structures} that the system presents and maintains look more like a single monolithic object, or an amorphous swarm? One simple heuristic is: if you cut the system in half, including both providers and users, will both halves continue to fully operate as independent units?}
\end{itemize}

\subsubsection{Our definition of decentralized systems}
\begin{itemize}
  \item{A decentralized system has political decentralization, where multiples parties are making their own independent decisions (they can still coordinate with each other)}
  \item{If a system is architecturally but not politically decentralized, we call it a distributed system}
  \item{Decentralized systems can be \textbf{logically decentralized or centralized}}
  \item{Decentralized systems can be open systems (anybody can participate) or closed systems}
\end{itemize}

\subsection{Reasons for decentralization}
\subsubsection{Reasons for architectural decentralization}
\begin{itemize}
  \item{\textbf{Latency}}
  \item{\textbf{Scalability}: Scale number of machines running the system}
  \item{Increase \textbf{fault tolerance} and \textbf{availability}, remove single point of failures}
  \item{Increase \textbf{attack resistance}, because no central points exist}
\end{itemize}

\subsubsection{Reasons for political decentralization}
\begin{itemize}
  \item{\textbf{Collusion resistance}: It is harder for participants to collude in ways that benefit a small group at the expense of other participants}
  \item{\textbf{Power} can be distributed ``equally''}
\end{itemize}

\subsubsection{Reasons for logical decentralization}
Logical decentralization is not always possible or even wanted, especially in use cases we cover. Example: \textbf{Distributed Ledgers}, where the goal is to have one commonly agreed system state at any point in time.

\subsection{Challenges of decentralization}
Decentralized systems come with risks/challenges to avoid harm to the system or its participants:
\begin{itemize}
  \item{\textbf{Time and synchrony}: Do we have a global clock? Is the communication synchronous or asynchronous?}
  \item{\textbf{behavior} of nodes: Can we handle arbitrary behavior? How many faulty nodes can we tolerate?}
  \item{\textbf{Identity}: Do we have an open system? Are nodes (identities) known?}
\end{itemize}

\section{Fundamentals}
\subsection{How to model a distributed system?}
We define a \textbf{distributed system} as a set of identical processes (or processors) that execute a program. Coordination between the processors is needed. The combination of processes form an application.

\subsubsection{Processes (Processors) and Messages}
A distributed system or algorithm consists of $n$ \highlight{processors} (called nodes/agents/participants) $p_0, \dots, p_{n-1}$.\par
Each process runs a local process and the processors cooperate on some \underline{common} task. They can communicate with each other.\par
\highlight{Messages} are uniquely identified by the sender using a sequence number of a logical clock.

\subsubsection{Links}
A \highlight{link} (channel) $\{p_i, p_j\}$ connects processors $i$ and $j$. Links are always considered \textbf{bidirectional}.\par
The network is a collection of all channels, the tolopogy is a pattern of channels (e.g. mesh, star).

\subsubsection{Inter-Process Communication}
Processors are communicating by \textbf{passing messages} to \textbf{in-} and \textbf{outboxes}. The address is a set of processes. Processors have access to \highlight{shared memory}.

\subsubsection{Automata and Steps}
We can model a distributed algorithm as a distributed collection of \highlight{automata} (one per process). Each automata is a state machine with defined states (\textbf{configurations}) and state transitions (\textbf{step}) that are triggered by an \textbf{event}.

\subsubsection{Safety and Liveness}
\begin{itemize}
  \item{\highlight{Safety}: ``\textit{Nothing bad has happened, yet}'': If a safety property is violated, we can point to a specific point in time where the violation occurred, \textbf{the violation cannot be undone}.}
  \item{\highlight{Liveness}: ``\textit{Eventually something good happend}'': At any time, there is the chance that the property will be satisfied at a later point in time.}
\end{itemize}

\subsection{Assumptions}
\subsubsection{Why do distributed algorithms need assumptions?}
Distributed algorithms deal with a lot of uncertainty, therefore we need assumptions to describe the \textbf{uncertainty} and \textbf{guarantees} of the system. Typical are
\begin{itemize}
  \item{\textbf{Process assumptions}: Crash behavior, adherance to the protocol}
  \item{\textbf{Communication assumptions}: Topology, reliability, attackers}
  \item{\textbf{Timing assumptions}: Latency, synchrony}
  \item{\textbf{Cryptographic assumptions}: Cryptographic primitives (e.g. encryption, signatures)}
  \item{\textbf{Setup assumptions}: What information is available to the participants at the start}
\end{itemize}

\subsubsection{Uniform/Nonuniform}
\begin{itemize}
  \item{\textbf{Uniform}: Total number of processors $n$ is not known to the algorithm}
  \item{\textbf{Nonuniform}: Each processor knows the total number of processors $n$}
\end{itemize}

\subsubsection{Fault model}\label{sec:faultModel}
The \highlight{Fault model} abstracts faults in the processors and channels.
\begin{itemize}
  \item{\highlight{Crash fault}: Processor works correctly until it crashes and never recovers}
  \item{\highlight{Omission fault}: Processor fails to send/receive messages it is supposed to send/receive (e.g. due to buffer overflow)}
  \item{\highlight{Crashes with recoveries}: Either the process crashes and never recovers or the process keeps crashing and recovering infinitely often}
  \item{\highlight{Byzantine fault}: Arbitrary behavior, the process can deviate from the protocol in any way}
\end{itemize}

\subsubsection{Fault tolerance}
The \highlight{Fault tolerance} of a system is the number of faulty processes $f$ out of $n$ processes that the system can tolerate while still operating correctly.

\subsubsection{Communication: Fair-loss links}
\textbf{Fair-loss links} are defined by three properties:
\begin{enumerate}
  \item{\textit{Fair-loss}: If a correct process $p$ infinitely often sends a message $m$ to a correct process $q$, then $q$ delivers $m$ an infinite number of times}
  \item{\textit{Finite duplication}: If a correct process $p$ sends a message $m$ a finite number of times to a process $q$, then $m$ cannot be delivered an infinite number of times by $q$}
  \item{\textit{No creation}: If some process $q$ delivers a message $m$ with sender $p$, then $m$ was previously sent to $q$ by $p$}
\end{enumerate}

\subsubsection{Communication: Perfect links}\label{sec:perfectLinks}
\textbf{Perfect links} are also defined by three properties:
\begin{enumerate}
  \item{\textit{Reliable delivery}: If a correct process $p$ sends a message $m$ to a correct process $q$, then $q$ eventually delivers $m$}
  \item{\textit{No duplication}: No message is delivered by a process more than once}
  \item{\textit{No creation}: If some process $q$ delivers a message $m$ with sender $p$, then $m$ was previously sent to $q$ by $p$}
\end{enumerate}
\textbf{Authenticated perfect links} are an extension of perfect links.

\subsubsection{Timing Models}
The \highlight{Timing model} describes the timing assumptions of the communication and execution behavior:
\begin{itemize}
  \item{\highlight{Synchronous model}: There is a \textbf{known upper bound} on processing delays and on message transmission delays}
  \item{\highlight{Asynchronous model}: There is \textbf{no timing assumption at all}. The execution and message delivery happens at an arbitrary speed, \textbf{but messages arrive eventually}}
\end{itemize}

\subsection{Time in Asynchronous Systems}
\subsubsection{Logical clocks: Lamport Clocks}
\highlight{Lamport clocks} are used to \textbf{measure passage of time} in \textbf{asynchronous systems}.
\begin{itemize}
  \item{Each process $p_i$ has a \textbf{logical clock} $l_i$, initially set to $0$}
  \item{Upon an event (sending or receiving a message), $l_i$ is incremented by $1$}
  \item{When sending a message $m$, process $p_i$ adds a timestamp $t_m = l_i$ to the message}
  \item{When receiving a message $m$, process $p_j$ increases its timestamp to $\max(l_j, t_m) + 1$}
\end{itemize}
With this, a \textbf{happened-before relationship} between events is established. For any two events $e_1, e_2$: $e_1 \rightarrow e_2 \Rightarrow t(e_1) < t(e_2)$.\par
This defines a \textbf{partial order} on the events.

\subsubsection{Hybrid: Partial Synchrony}
A hybrid between synchrony and asynchrony is \textbf{partial synchrony}, which comes in two variants:
\begin{itemize}
  \item{\textbf{Eventually synchronous}/Global Stabilization Time (GST): An event GST occurs after some finite time, afterwards time bound $\Lambda$ holds}
  \item{\textbf{Unknown Latency (UL)}: The system is always synchronous, but the delay bound $\Lambda$ is unknown}
\end{itemize}
Algorithms for these models typically increment their estimation of the delay bound dynamically.

\subsection{Combining Abstractions for Assumptions}
\subsubsection{Fail-stop}
\begin{itemize}
  \item{\hyperref[sec:faultModel]{Crash faults}}
  \item{\hyperref[sec:perfectLinks]{Perfect links}}
  \item{Perfect failure detector}
\end{itemize}

\subsubsection{Fail-silent}
\begin{itemize}
  \item{\hyperref[sec:faultModel]{Crash faults}}
  \item{\hyperref[sec:perfectLinks]{Perfect links}}
  \item{No failure detector}
\end{itemize}

\subsubsection{Fail-arbitrary}
\begin{itemize}
  \item{\hyperref[sec:faultModel]{Byzantine faults}}
  \item{\hyperref[sec:perfectLinks]{Authenticated perfect links}}
\end{itemize}

\subsection{Problem Statement: Leader Election}
\subsubsection{Problem definition}
\begin{itemize}
  \item{Group of processors has to elect one of them as leader}
  \item{Exactly one processor enters an elected state, all others enter a non-elected state}
\end{itemize}

\subsubsection{Anonymous rings}
\begin{itemize}
  \item{Processors have no identifiers}
  \item{Each processor has the same deterministic state machine}
  \item{Each processor is connected to two other processors}
\end{itemize}

\subsubsection{Leader Election in Anonymous Rings}
``\textit{There is no nonuniform anonymous algorithm for leader election in synchronous rings.}''\par
So, even with these strong assumptions, \textbf{leader election is not possible with anonymous participants}.

\subsubsection{Leader Election in Asynchronous Rings}
Setup assumptions:
\begin{itemize}
  \item{Assign each processor $p_i$ a unique identifier $id_i$}
  \item{Label the two connected processors of $p_i$ as \textit{left} and \textit{right neighbor}}
\end{itemize}

\begin{algorithm}
  \caption{Algorithm for Leader Election in Asynchronous Rings}
  \begin{algorithmic}
    \State \textbf{Each processor sends its identifier to its left neighbour.}
    \State \textbf{When a processor receives a termination message, it forwards it to the left neighbour and terminates as non-leader.}\\
    \State \textbf{Upon receiving message $m$ from right neighbour}
    \State \hspace{\algorithmicindent} \textbf{if $m < id_i$ then}
    \State  \hspace{\algorithmicindent}\hspace{\algorithmicindent} \textbf{drop message}
    \State \hspace{\algorithmicindent} \textbf{else if $m > id_i$ then}
    \State  \hspace{\algorithmicindent}\hspace{\algorithmicindent} \textbf{forward $m$ to left neighbour}
    \State \hspace{\algorithmicindent} \textbf{else if $m == id_i$ then}
    \State  \hspace{\algorithmicindent}\hspace{\algorithmicindent} \textbf{Send termination message to left neighbour}
    \State  \hspace{\algorithmicindent}\hspace{\algorithmicindent} \textbf{Terminate as leader}
    \State \hspace{\algorithmicindent} \textbf{end if}
  \end{algorithmic}
\end{algorithm}

The algorithm sends not more than $O(n^2)$ messages. Nonuniforms algorithms for synchronous rings also exist.

\subsection{Problem Statement: Mutual Exclusion}
\subsubsection{Problem definition}
\textbf{Mutual exclusion} is a known problem from concurrent computing. We need to ensure than critical sections are only accessible by one processor at a time. The desired sequence is Entry $\Rightarrow$ Critical section $\Rightarrow$ Exit.\par
The following three objectives should be satisfied:
\begin{itemize}
  \item{\textbf{Mutual exclusion}: At most one process can be in the critical section at any time (\textbf{Safety})}
  \item{\textbf{No Deadlock}: If some processor enters an entry section, some processor will later enter a critical section.}
  \item{\textbf{No lockout (starvation)}: If some processor enters an entry section, \textbf{that same processor} will later enter a critical section.}
\end{itemize}

\subsubsection{Lamport Mutual Exclusion}
Lamport defined a mutual exclusion algorithm based on \textbf{logical clocks} \cite{lamportTimeClocks}, the algorithm is not covered in this summary.

\subsection{Formalization via Modules}
We can now combine abstractions using \highlight{Modules}. A module has a \textbf{name}, \textbf{events} and \textbf{safety \& liveness properties}.\par
An algorithm \textit{implements} a module (and can build on other modules).

\newpage
\subsubsection{Perfect Failure Detector}
\textbf{Events}:
\begin{itemize}
  \item{\textbf{Indication}: $\langle \mathcal{P}, \textit{Crash} \mid p \rangle$: Detects that process $p$ has crashed}
\end{itemize}
\textbf{Properties}:
\begin{itemize}
  \item{\textbf{PFD1:} \textit{Strong completeness}: Eventually, every process that crashes is permanently detected by every correct process}
  \item{\textbf{PFD2:} \textit{Strong accuracy}: If a process $p$ is detected by any process, then $p$ has crashed}
\end{itemize}
Perfect Failure Detector is implemented by ``Exclude on Timeout'' (not covered in this summary).

\subsubsection{Eventually Perfect Failure Detector}
\textbf{Events}:
\begin{itemize}
  \item{\textbf{Indication}: $\langle \diamond \mathcal{P}, \textit{Suspect} \mid p \rangle$: Notifies that process $p$ is suspected to have crashed}
  \item {\textbf{Indication}: $\langle \diamond \mathcal{P}, \textit{Restore} \mid p \rangle$: Notifies that process $p$ is not suspected anymore}
\end{itemize}

\textbf{Properties}:
\begin{itemize}
  \item{\textbf{EPFD1:} \textit{Strong completeness}: Eventually, every process that crashes is permanently suspected by every correct process}
  \item{\textbf{EPFD2:} \textit{Eventual strong accuracy}: Eventually, no correct process is suspected by any correct process}
\end{itemize}
Eventually Perfect Failure Detector is implemented by ``Increasing Timeout'' (not covered in this summary).

\subsubsection{Leader Election}
\textbf{Events}:
\begin{itemize}
  \item{\textbf{Indication}: $\langle le, \textit{Leader} \mid p \rangle$: Indicates that process $p$ is elected as leader}
\end{itemize}

\textbf{Properties}:
\begin{itemize}
  \item{\textbf{LE1:} \textit{Eventual detection}: Either there is no correct process, or some correct process is eventually elected as the leader}
  \item{\textbf{LE2:} \textit{Accuracy}: If a process is leader, then all previously elected leaders have crashed}
\end{itemize}
Leader Election is implemented by ``Monarchical Leader Election'' (not covered in this summary).

\subsubsection{Eventual Leader Detector}
\textbf{Events}:
\begin{itemize}
  \item{\textbf{Indication}: $\langle \Omega, \textit{Trust} \mid p \rangle$: Indicates that process $p$ is trusted to be leader}
\end{itemize}

\textbf{Properties}:
\begin{itemize}
  \item{\textbf{ELD1:} \textit{Eventual accuracy}: There is a time after which every correct process trusts some correct process}
  \item{\textbf{ELD2:} \textit{Eventual agreement}: There is a time after which no two correct processes trust different processes}
\end{itemize}
Eventual Leader Detector is implemented by ``Monarchical Eventual Leader Detection'' (not covered in this summary).

\subsection{Quorums}
A \highlight{Quorum} is a set of processes with special properties. They are used for fault-tolerant algorithms. Dealing with $N$ crash-fault processes, \textbf{a quorum is any majority of processes}.\par
Assumption: There is a quorum of non-faulty processes (number of faulty processes $f < N/2$).\par
\textbf{Properties}:
\begin{itemize}
  \item{Two quorums intersect in at least one process}
  \item{In every quorum is at least one correct (non-faulty) process}
\end{itemize}
Dealing with $N$ arbitrary-fault processes (byzantine): To maintaing the second property, a quorum needs to be a set of more than $\frac{N+f}{2}$ processes. We call a set of more than $\frac{N+f}{2}$ a \highlight{byzantine quorum}.\par
When the required property is that there exists a Byzantine quorum of correct processes, $3f < N$ needs to hold.

\newpage
\bibliographystyle{apalike}
\bibliography{\jobname}


\end{document}