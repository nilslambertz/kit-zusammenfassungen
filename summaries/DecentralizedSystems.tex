\documentclass[12pt,A4]{extarticle}	
\usepackage{filecontents}

\begin{filecontents}{bibfile.bib}
  @article{DistributedSystemVanSteenTanenbaum,
  author       = {Maarten van Steen and
                  Andrew S. Tanenbaum},
  title        = {A brief introduction to distributed systems},
  journal      = {Computing},
  volume       = {98},
  number       = {10},
  pages        = {967--1009},
  year         = {2016},
  url          = {https://doi.org/10.1007/s00607-016-0508-7},
  doi          = {10.1007/s00607-016-0508-7},
  timestamp    = {Thu, 14 Oct 2021 09:12:09 +0200},
  biburl       = {https://dblp.org/rec/journals/computing/SteenT16.bib},
  bibsource    = {dblp computer science bibliography, https://dblp.org}
}
\end{filecontents}

\newcommand{\lectureTitle}{Decentralized Systems [WIP]}
\newcommand{\lectureSubtitle}{Fundamentals, Modeling, and Applications}
\newcommand{\semester}{Sommersemester 2023}

\newcommand{\titleSize}{\LARGE}

\usepackage[a4paper,left=0.9cm,right=1cm,top=1.37cm,bottom=2.5cm]{geometry}
\usepackage[utf8]{inputenc}
\usepackage{xifthen}
\usepackage{cmbright}
\usepackage{fontawesome}
\usepackage[T1]{fontenc}
\usepackage{lastpage,lipsum}
\usepackage{hyperref}
\usepackage{transparent}
\usepackage{color}
\usepackage{fancyhdr}

\renewcommand*\familydefault{\sfdefault}
\setlength{\parindent}{0mm}

\usepackage{transparent}
\usepackage{color}
\usepackage{fancyhdr}

\definecolor{headerBg}{RGB}{11, 67, 158}
\definecolor{headerGrayColor}{RGB}{210, 210, 210}

\pagestyle{fancy}
\fancyhead[C]{
  \fcolorbox{headerBg}{headerBg}{
    \hspace{0.6cm}\begin{minipage}[c][50pt][c]{\paperwidth}
      \begin{minipage}[c]{.45\textwidth}
        \huge{\textcolor{white}{Vorlesung}}\normalsize\\
        \textcolor{headerGrayColor}{\small{Wintersemester 2022/23}}
      \end{minipage}%
      \begin{minipage}[c]{.45\textwidth}
        \raggedleft
        \textcolor{white}{
          \small{\href{https://nilslambertz.de/}{nilslambertz.de}}\\
          \href{https://github.com/nilslambertz/}{\textcolor{white}{\faicon{github}} \small{nilslambertz}}}
      \end{minipage}
    \end{minipage}}
}
\renewcommand{\headrulewidth}{0pt}
\setlength{\headheight}{40pt}

\newlength{\oddmarginwidth}
\setlength{\oddmarginwidth}{1in+\hoffset+\oddsidemargin}
\newlength{\evenmarginwidth}
\setlength{\evenmarginwidth}{\evensidemargin+1in}
\fancyhfoffset[LO,RE]{\oddmarginwidth}
\fancyhfoffset[LE,RO]{\evenmarginwidth}
\cfoot{\thepage\ $/$ \pageref*{LastPage}}


\definecolor{highlightColor}{RGB}{66, 135, 245}
\newcommand{\highlight}[1]{\textcolor{highlightColor}{\textbf{#1}}}

\def\contentsname{\empty}

\begin{document}

\disclaimer

\tableofcontents
\clearpage

\section{Introduction}
\subsection{What is a distributed system?}
\subsubsection{Characteristics by van Steen and Tanenbaum}
``\textit{A distributed system is a collection of autonomous computing elements that appears to its users as a single coherent system.}'' (\cite{DBLP:journals/computing/SteenT16}), daraus gehen zwei Charakteristiken hervor:
\begin{enumerate}
  \item{``\textbf{Collection of autonomous computing elements}'': ``\textit{In practice, nodes are programmed to achieve common goals, which are realized by exchanging messages with each other}'' \\(\cite{DBLP:journals/computing/SteenT16})}
  \item{``\textbf{Appears as a single coherent system}'': Appears as a single large system}
\end{enumerate}

\subsubsection{Consequences of characteristic 1}
\begin{itemize}
  \item{``\textit{we cannot assume that there is something like a global clock}'' (\cite{DBLP:journals/computing/SteenT16}), therefore the synchronization and coordination between must be worked out}
  \item{``\textit{The fact that we are dealing with a collection of nodes implies that we may also need to manage the membership and organization of that collection}'' (\cite{DBLP:journals/computing/SteenT16}), therefore we need to think about identities and possible access-restrictions}
\end{itemize}

\subsubsection{Consequences of characteristic 2}
\begin{itemize}
  \item{``\textit{To assist the development of distributed applications, distributed systems are often organized to have a separate layer of software that is logically placed on top of the respective operating systems of the computers that are part of the system [...] leading to what is known as middleware}'' (\cite{DBLP:journals/computing/SteenT16})}
\end{itemize}

\subsubsection{Observations}
\begin{itemize}
  \item{Distributing tasks and aggregating a result from them is not easy at all}
  \item{The coordination of those tasks is still \textbf{centralized}}
\end{itemize}

\subsection{What makes a distributed system a decentralized system?}
According to \textbf{ISO/TC 307}: ``\textit{distributed system wherein control is distributed among the persons or organizations participating in the operation of the system}''

\subsubsection{Three types of Decentralization (Vitalik Buterin)}
\href{https://medium.com/@VitalikButerin/the-meaning-of-decentralization-a0c92b76a274}{Vitalik Buterin defines three types of Decentralization}:
\begin{itemize}
  \item{\textbf{Architectural (de)centralization}: How many \textbf{physical computers} is a system made up of? How many of those computers can it tolerate breaking down at any single time?}
  \item{\textbf{Political (de)centralization}: How many \textbf{individuals or organizations} ultimately control the computers that the system is made up of?}
  \item{\textbf{Logical (de)centralization}: Does the \textbf{interface and data structures} that the system presents and maintains look more like a single monolithic object, or an amorphous swarm? One simple heuristic is: if you cut the system in half, including both providers and users, will both halves continue to fully operate as independent units?}
\end{itemize}

\subsubsection{Our definition of decentralized systems}
\begin{itemize}
  \item{A decentralized system has political decentralization, where multiples parties are making their own independent decisions}
  \item{If a system is architecturally but not politically decentralized, we call it a distributed system}
  \item{Decentralized systems can be \textbf{logically decentralized or centralized}}
  \item{Decentralized systems can be open systems (anybody can participate) or closed systems}
\end{itemize}

\subsection{Reasons for decentralization}
\subsubsection{Reasons for architectural decentralization}
\begin{itemize}
  \item{Reduce \textbf{latency}}
  \item{\textbf{Scale} the number of machines running the system}
  \item{Increase \textbf{fault tolerance} and \textbf{availability}, remove single point of failures}
  \item{Increase \textbf{attack resistance}, because no central points exist}
\end{itemize}

\subsubsection{Reasons for political decentralization}
\begin{itemize}
  \item{Increate \textbf{collusion resistance}, it is harder to act in ways that benefit a small group at the expense of other participants}
  \item{\textbf{Power} can be \textbf{distributed} ``equally''}
\end{itemize}

\subsubsection{Reasons for logical decentralization}
Logical decentralization is not always possible or even wanted, especially in use cases we cover. An example are \textbf{Distributed Ledgers}, where the goal is to have one commonly agreed state of the system at any point in time.

\subsection{Risks of decentralization}
Decentralized systems come with risks/challenges to avoid harm to the system or its participants:
\begin{itemize}
  \item{\textbf{Impersonation} or \textbf{Misrepresentation}}
  \item{\textbf{Fraudulent actions}}
  \item{\textbf{Collusion}}
  \item{Denial-of-Service-attacks}
\end{itemize}

\subsection{Two Generals' Problem}
Cryptography can be used to ensure authenticity, integrity and confidentiality of a message sent over an (unreliable) channel. But it's not possible to ensure \textbf{availability} (that a message has been delivered), we can only reduce the probability of these events by using heuristics like \textit{sequence numbers} or \textit{retransmissions}.

\newpage
\bibliographystyle{apalike}
\bibliography{bibfile}


\end{document}