\documentclass[12pt,A4]{extarticle}	
\usepackage[table]{xcolor}

\newcommand{\lectureTitle}{Steuerrecht}
\newcommand{\semester}{Sommersemester 2024}

\usepackage[a4paper,left=0.9cm,right=1cm,top=1.37cm,bottom=2.5cm]{geometry}
\usepackage[utf8]{inputenc}
\usepackage{xifthen}
\usepackage{cmbright}
\usepackage{fontawesome}
\usepackage[T1]{fontenc}
\usepackage{lastpage,lipsum}
\usepackage{hyperref}
\usepackage{transparent}
\usepackage{color}
\usepackage{fancyhdr}

\renewcommand*\familydefault{\sfdefault}
\setlength{\parindent}{0mm}

\definecolor{headerBg}{RGB}{11, 67, 158}
\definecolor{headerGrayColor}{RGB}{210, 210, 210}

\newcommand{\printTitle}{\textcolor{white}{\lectureTitle}\normalsize}
\newcommand{\printSubtitle}{
  \ifdefined\lectureSubtitle
    \textcolor{white}{\small{\lectureSubtitle}}\\
  \fi
}

\fancyhf{}
\pagestyle{fancy}
\fancyhead[C]{
  \fcolorbox{headerBg}{headerBg}{
    \hspace{0.6cm}\begin{minipage}[c][50pt][c]{\paperwidth}
      \begin{minipage}[c]{.7\textwidth}
        \ifdefined\titleSize
          \titleSize \printTitle\\
        \else
          \huge\printTitle\\
        \fi
        \printSubtitle
        \textcolor{headerGrayColor}{\small{\semester}}
      \end{minipage}%
      \begin{minipage}[c]{.2\textwidth}
        \raggedleft
        \textcolor{white}{
          \small{\href{mailto:mail@nilslambertz.de}{\textcolor{white}{\faicon{envelope}} mail@nilslambertz.de}}\\
          \href{https://github.com/nilslambertz/kit-zusammenfassungen}{\textcolor{white}{\faicon{github}} \small{nilslambertz}}}
      \end{minipage}
    \end{minipage}}
}
\renewcommand{\headrulewidth}{0pt}
\setlength{\headheight}{40pt}

\newlength{\oddmarginwidth}
\setlength{\oddmarginwidth}{1in+\hoffset+\oddsidemargin}
\newlength{\evenmarginwidth}
\setlength{\evenmarginwidth}{\evensidemargin+1in}
\fancyhfoffset[LO,RE]{\oddmarginwidth}
\fancyhfoffset[LE,RO]{\evenmarginwidth}
\cfoot{\thepage\ $/$ \pageref*{LastPage}}

\definecolor{highlightColor}{RGB}{66, 135, 245}
\newcommand{\highlight}[1]{\textcolor{highlightColor}{\textbf{#1}}}
\definecolor{gesetzLink}{RGB}{194, 74, 14}
\newcommand{\estG}[2][]{\textbf{\textcolor{gesetzLink}{\href{https://www.gesetze-im-internet.de/estg/__#2.html}{§ #2 \ifthenelse{\equal{#1}{}}{}{#1 }EStG}}}}
\newcommand{\estGG}[2][]{\textbf{\textcolor{gesetzLink}{\href{https://www.gesetze-im-internet.de/estg/__#2.html}{§§ #1 EStG}}}}
\newcommand{\abgabenordnung}[2][]{\textbf{\textcolor{gesetzLink}{\href{https://www.gesetze-im-internet.de/ao_1977/__#2.html}{§ #2 \ifthenelse{\equal{#1}{}}{}{#1 }AO}}}}
\newcommand{\bgb}[2][]{\textbf{\textcolor{gesetzLink}{\href{https://www.gesetze-im-internet.de/bgb/__#2.html}{§ #2 \ifthenelse{\equal{#1}{}}{}{#1 }BGB}}}}
\newcommand{\hgb}[2][]{\textbf{\textcolor{gesetzLink}{\href{https://www.gesetze-im-internet.de/hgb/__#2.html}{§ #2 \ifthenelse{\equal{#1}{}}{}{#1 }HGB}}}}
\newcommand{\hgbb}[2][]{\textbf{\textcolor{gesetzLink}{\href{https://www.gesetze-im-internet.de/hgb/__#2.html}{§§ #1 HGB}}}}
\newcommand{\kstG}[2][]{\textbf{\textcolor{gesetzLink}{\href{https://www.gesetze-im-internet.de/kstg_1977/__#2.html}{§ #2 \ifthenelse{\equal{#1}{}}{}{#1 }KStG}}}}
\newcommand{\kstGG}[2][]{\textbf{\textcolor{gesetzLink}{\href{https://www.gesetze-im-internet.de/kstg_1977/__#2.html}{§§ #1 KStG}}}}

\def\contentsname{\empty}

\begin{document}

\disclaimer

\tableofcontents
\clearpage

\section{Grundlagen}
\subsection{Grundlegende Begriffe}
\subsubsection{Direkte und indirekte Steuern}
\begin{itemize}
  \item{\textbf{Direkte Steuern}: Belastet Person, Unternehmen, usw. unmittelbar}
  \item{\textbf{Indirekte Steuern}: Belastet Güter und Vorgänge und erreicht damit mittelbar einzelne Personen}
\end{itemize}

\subsubsection{Unternehmensteuerrecht}
Die Unternehmenssteuer ist keine eigene Steuerart, sondern durch den Dualismus von Einkommen- und Körperschaftsteuer geprägt. Die Einkommensteuer betrifft z.B. Einzelunternehmen, die Körperschaftsteuer betrifft Kapitalgesellschaften.

\subsubsection{Leistungsfähigkeitsprinzip}
Das Leistungsfähigkeitsprinzip ist (vereinfacht gesagt) die Fähigkeit, Steuern zu bezahlen und damit zum Steueraufkommen beizutragen. Es gibt \textbf{drei Leistungsfähigkeitsindikatoren}:
\begin{itemize}
  \item{\textbf{Einkommen} (Vermögenszugang, z.B. EStG, KStG, GewStG)}
  \item{\textbf{Vermögen} (Konsum- und Investitionsfonds, z.B. GrStG, GrErwStG, ErbStG)}
  \item{\textbf{Konsum} (Güterverbrauch, z.B. UStG)}
\end{itemize}
Bei der Einkommensermittlung gilt das \textbf{Nettoprinzip} zur Bemessung der Leistungsfähigkeit. Es wird also nur der Gewinn besteuert statt die kompletten Einnahmen (\highlight{objektives Nettoprinzip}).\par
Das \highlight{private/subjektive Nettoprinzip} verlangt zudem die Abzugsfähigkeit (für die Lebensführung) unvermeidbarer Ausgaben (daher z.B. Grundfreibetrag).

\subsubsection{Der Steuerbegriff}
Der Steuerbegriff ist in \abgabenordnung[Abs. 1]{3} definiert, wichtige Details:
\begin{itemize}
  \item{Steuern sind \textbf{Geldleistungen}}
  \item{keine Gegenleistung für eine \textbf{besondere Leistung} (\textit{also keine konkrete Gegenleistung})}
  \item{werden von einem öffentlich-rechtlichen \textbf{Gemeinwesen} (Gemeinden, Länder, Bund) zur \textbf{Erzielung von Einnahmen} erhoben}
  \item{sind \textbf{allen auferlegt}, bei denen der Tatbestand der Leistungspflicht zutrifft}
\end{itemize}

\section{Einkommensteuer}
\subsection{Tatbestandsmerkmale}
Gemäß \estG[Abs. 1]{1} sind
\begin{itemize}
  \item{\textbf{natürliche} Personen, die}
  \item{im Inland}
  \item{einen Wohnsitz oder ihren gewöhnlichen Aufenthalt haben}
\end{itemize}
\textbf{unbeschränkt einkommensteuerpflichtig}.

\subsection{Begriffe}

\subsubsection{Natürliche Personen}
\textbf{Natürliche Personen} sind Menschen (also keine Unternehmen etc.). Die Rechtsfähigkeit beginnt gemäß \bgb{1} mit der Vollendung der Geburt.

\subsubsection{Inland}
Das Inland i.S.d. \estG[Abs. 1]{1} ist das Staatsgebiet der Bundesrepublik Deutschland (inklusive der 12-Meilen-Zone). Zudem gehören deutsche Flugzeuge, Handelsschiffe unter deutscher Flagge und Kriegsschiffe zum Inland.

\subsubsection{Wohnsitz}
Nach \abgabenordnung{8} hat jemand einen Wohnsitz dort, wo er eine Wohnung unter Umständen innehat, die darauf schließen lassen, dass er die Wohnung beibehalten und benutzen wird. Die Absicht einen Wohnsitz zu begründen oder aufzugeben bzw. die An- und Abmeldung der Wohnung sind für die steuerliche Wirkung nicht entscheidend.

\subsubsection{Gewöhnlicher Aufenthalt}
Auch ohne Wohnsitz kann gemäß \abgabenordnung{9} die unbeschränkte Steuerpflicht bestehen, wenn jemand seinen \textbf{gewöhnlichen Aufenthalt} in Deutschland hat. Hier kommt es auf die subjektiven Absichten des Steuerpflichtigen an (183-Tage-Regelung).

\subsubsection{Welteinkommensprinzip}\label{sec:welteinkommensprinzip}
Das \highlight{Welteinkommensprinzip} besagt, dass für die Steuerpflicht das gesamte Welteinkommen herangezogen wird, unabhängig davon wo es erwirtschaftet wurde. Ohne Doppelbesteuerungsabkommen kann es zu einer Doppelbesteuerung kommen.

\subsubsection{Markteinkommensprinzip}
Das \textbf{Markteinkommensprinzip} umschreibt die \textbf{sieben Einkunftsarten}, die einkommensteuerbar sind. Erträge ohne Markt (Schenkung, Erbschaft) sind nicht einkommensteuerbar.

\subsubsection{Wirtschaftsgut}
Während nach dem Handelsrecht \textbf{Vermögensgegenstände} aktiviert werden müssen (\hgbb[246 ff.]{246}), spricht das Steuerrecht von \highlight{Wirtschaftsgütern}, die u.a. nach \estG{6} bewertet werden müssen.\par
Wirtschaftsgüter können Sachen (\bgb{90}), Tiere (\bgb{90a}), Rechte oder tatsächliche Zustände, konkrete Möglichkeiten oder Vorteile für den Betrieb sein, deren \textbf{Erlangung} der \textbf{Kaufmann} sich \textbf{etwas kosten lässt}, die einer besonderen \textbf{Bewertung zugänglich} sind, i.d.R eine Nutzung für mehrere Wirtschaftsjahre erbringen und zumindest mit dem Betrieb übertragen werden können.\par
Zu unterscheiden ist zunächst zwischen \textbf{Vermögensgegenständen} und \textbf{Schulden} (\hgb[Abs. 1]{240}, \hgb{253}). Die Wirtschaftsgüter selbst werden in unterschiedliche Arten unterteilt, da das EStG an diese Arten unterschiedliche Regelungen knüpft.

\subsection{Sachliche Steuerpflicht}
Die \textbf{sachliche Steuerpflicht} beschreibt, was besteuert wird und ist in \estG{2} definiert. Es wird zwischen sieben Einkunftsarten und zwei Gruppen (Gewinneinkünfte und Überschusseinkünfte) unterschieden.

\subsubsection{Gewinneinkünfte}
\textbf{Gewinneinkünfte} sind gemäß \estG[Abs. 2 Nr. 1]{2} der \textbf{Gewinn} aus
\begin{itemize}
  \item{Land- und Forstwirtschaft (\estG{13})}
  \item{Gewerbebetrieb (\estG{15})}
  \item{selbständiger Arbeit (\estG{18})}
\end{itemize}

\subsubsection{Überschusseinkünfte}
\textbf{Überschusseinkünfte} sind gemäß \estG[Abs. 2 Nr. 2]{2} \textbf{Überschuss der Einnahmen über die Werbungskosten} aus
\begin{itemize}
  \item{nichtselbstständiger Arbeit (\estG{19})}
  \item{Kapitalvermögen (\estG{20})}
  \item{Vermietung und Verpachtung (\estG{21})}
  \item{sonstigen Einkünften (\estG{22})}
\end{itemize}

\subsubsection{Ermittlung des Einkommens}\label{sec:ermittlungDesEinkommens}
\estG{2} beschreibt die Schritte zur Berechnung des Einkommens:\par
\begin{enumerate}
  \item{\textbf{Einkünfte} (\estG[Abs. 1]{2})}
  \item{\textbf{Summe der Einkünfte} (\estG[Abs. 2]{2})}
  \item{\textbf{Gesamtbetrag der Einkünfte} (\estG[Abs. 3]{2}): Summe der Einkünfte abzüglich ggf. eines Altersentlastungsbetrag (\estG{24a}), Land- und Forstwirte erhalten ebenefalls nach \estG[Abs. 3]{13} einen Abzugsbetrag}
  \item{\textbf{Einkommen} (\estG[Abs. 4]{2}): Gesamtbetrag der Einkünfte abzüglich Sonderausgaben und außergewöhnlichen Belastungen}
  \item{\textbf{zu versteuerndes Einkommen} (\estG[Abs. 5]{2}): Einkommen abzüglich (z.B. Kinder-)Freibeträgen}
\end{enumerate}

\subsection{Gewinnermittlung bei Gewinneinkünften}
\subsubsection{Gewinn}
Der \highlight{Gewinn} eines Unternehmens ist gem. \estG[Abs. 1, 5]{4} der \textbf{Unterschiedsbetrag} zwischen dem \textbf{Betriebsvermögen am Schluss des Wirtschaftsjahres} und dem \textbf{Betriebsvermögen am Schluss des vorhergehenden Wirtschaftsjahres, vermehrt} um den Wert der Entnahmen und \textbf{vermindert} um den Wert der Einlagen.

\subsubsection{Objektives Nettoprinzip}
Prinzipiell sind alle Aufwendungen als Betriebsausgaben oder Werbungskosten abziehbar. Die Einkommensteuer besteuert gemäß \estG[Abs. 2]{2} nur den Gewinn bzw. den Überschuss der Einnahmen über die Werbungskosten.

\subsubsection{Buchführungspflicht und Gewinnermittlung}
Die steuerliche \highlight{Buchführungspflicht} ergibt sich \textbf{derivativ} (abgeleitet aus anderen Gesetzen, z.B. \hgb{238}) aus \abgabenordnung{140} und \textbf{originär} aus \abgabenordnung{141}.\par
Die Gewinnermittlung erfolgt bei Buchführungspflichtigen und denen, die freiwillig Buch führen durch den Betriebsvermögensvergleich nach \estG{5} in Verbindung mit \estG[Abs. 1]{4}.\par
Bei Steuerpflichtigen, die keine Bilanz aufstellen müssen, erfolgt die Gewinnermittlung nach \estG[Abs. 3]{4}.

\subsubsection{Formel zur Gewinnermittlung}
Die Gewinnermittlung erfolgt gemäß \estG[Abs. 1 S. 1]{4} durch die Formel:
\begin{align*}
    & \hspace{0.5cm} \text{Betriebsvermögen am Schluss des Wirtschaftsjahres}                 \\
  - & \hspace{0.5cm} \text{Betriebsvermögen am Schluss des vorangegangenen Wirtschaftsjahres} \\
  = & \hspace{0.5cm} \textbf{Unterschiedsbetrag}                                              \\
  + & \hspace{0.5cm} \text{Entnahmen}                                                         \\
  - & \hspace{0.5cm} \text{Einlagen}                                                          \\
  = & \hspace{0.5cm} \textbf{Gewinn}
\end{align*}

\subsubsection{Betriebsvermögen}
Das \highlight{Betriebsvermögen} setzt sich zusammen aus allen aktiven Wirtschaftsgütern, die ein Betrieb zur Gewinnerzielung einsetzt, und allen Verbindlichkeiten, die betrieblich veranlasst sind.

\subsubsection{Steuerrechtliches Betriebsvermögen}
Der steuerliche Betriebsvermögensvergleich bezieht sich nur auf das Betriebsvermögen. Das steuerliche Betriebsvermögen gliedert sich in notwendiges und gewillkürtes Betriebsvermögen. Wirtschaftsgüter werden wie folgt unterteilt:
\begin{enumerate}
  \item{\highlight{Notwendiges Betriebsvermögen}: Ausschließliche und unmittelbare betriebliche Nutzung. Wirtschaftsgüter (außer (Teil-)Grundstücke), die zu \textbf{mehr als 50 \% eigenbetrieblich} genutzt werden, sind in vollem Umfang notwendiges Betriebsvermögen.}
  \item{\highlight{Gewillkürtes Betriebsvermögen}: Wirtschaftsgüter, die in einem gewissen objektivem Zusammenhang mit dem Betrieb stehen und ihn zu fördern bestimmt und geeignet sind. Bei einer \textbf{betrieblichen Nutzung zwischen 10 \% und 50 \%} ist der Ausweis des Wirtschaftsguts als gewillkürtes Betriebsvermögen in vollem Umfang möglich.}
  \item{\highlight{Notwendiges Privatvermögen}: Wirtschaftsgüter, die ausschließlich oder zu \textbf{mehr als 90 \% außerbetrieblichen Zwecken} dienen, sind notwendiges Privatvermögen.}
\end{enumerate}
Wirtschaftsgüter können einheitlich entweder Betriebsvermögen oder Privatvermögen sein, eine Aufspaltung ist nicht möglich (\textbf{Einheitlichkeitsgrundsatz}). Grundstücke bilden dabei eine Ausnahme.

\subsubsection{Betriebsvermögen: Besonderheit bei Gebäuden}
Wird ein Gebäude teils eigengewerblich, teils fremdgewerblich, teils zu eigenen und teil zu fremden Wohnzwecken genutzt, so ist \textbf{jeder Gebäudeteil ein eigenes Wirtschaftsgut}. Das Gebäude wird nach dem Verhältnis der Nutzfläche aufgeteilt, gehört ein Grundstück nur teilweise dem Betriebsinhaber kann es auch nur insoweit Betriebsvermögen sein, als es dem Betriebsinhaber gehört.

\subsubsection{Entnahmen}
\highlight{Entnahmen} sind gem. \estG[Abs. 1 S. 2]{4} alle Wirtschaftsgüter, die der Steuerpflichtige dem Betrieb für sich, für seinen Haushalt oder für andere betriebsfremde Zwecke im Laufe des Wirtschaftsjahres entnimmt.\par
Dazu zählen neben Barentnahmen auch Waren, Erzeugnisse, Nutzungen und Leistungen.

\subsubsection{Einlagen}
\highlight{Einlagen} sind gem. \estG[Abs. 1 S. 8]{4}  alle Wirtschaftsgüter, die der Steuerpflichtige dem Betrieb im Laufe des Wirtschaftsjahres zugeführt hat. Hierzu zählen insbesondere Bareinzahlungen und Sacheinlagen.

\subsubsection{Betriebsausgaben}
\highlight{Betriebsausgaben} sind gem. \estG[Abs. 4]{4} die \textbf{Aufwendungen, die durch den Betrieb veranlasst sind}.\par
Dabei kommt es nicht darauf an, ob die Aufwendungen für den Betrieb des Steuerpflichtigen notwendig, zweckmäßig oder üblich sind.

\subsubsection{Betriebseinnahmen}
Betriebseinnahmen sind im Gesetz nicht definiert, die Rechtsprechung definiert mithilfe von \estG{8} die \highlight{Betriebseinnahmen} als \textbf{Zugänge von Wirtschaftsgütern} (in Form von Geld oder Geldeswert), die durch den Betrieb veranlasst sind.

\subsubsection{Dreiteilung der Kosten}
Das Einkommensteuerrecht wird beherrscht von einer \textbf{Dreiteilung der Kosten}. Danach ist zu unterscheiden, ob es sich bei den Kosten des Steuerpflichtigen um \textbf{Betriebsausgaben}, \textbf{Werbungskosten} oder \textbf{Kosten der Lebensführung} handelt.\par
Betrieblich veranlasste Kosten stellen nach \estG[Abs. 4]{4} Betriebsausgaben dar. Ausschließlich privat veranlasste Kosten dürfen den Gewinn demnach nicht mildern, was in \estG[Nr. 1]{12} nochmal ausdrücklich klargestellt ist.

\subsubsection{Werbungskosten}
Werbungskosten sind gemäß \estG[Abs. 1]{9} ``Aufwendungen zur Erwerbung, Sicherung und Erhaltung der Einnahmen''. Zwischen Aufwendungen und Einnahmen muss demnach ein \textbf{kausaler Zusammenhang} bestehen.

\subsection{Bewertungsgrundsätze}
Zwei grundsätzliche Fragen sind zu beantworten:
\begin{enumerate}
  \item{Welche Wirtschaftsgüter müssen in der Bilanz (Betriebsvermögensvergleich) erfasst werden?}
  \item{Mit welchem Wert sind diese Wirtschaftsgüter anzusetzen?}
\end{enumerate}
Allgemeine Bewertungsgrundsätze sind in \hgb[Abs. 1]{252} festgeschrieben. Von ihnen ist gemäß \hgb[Abs. 2]{252} nur in begründeten Ausnahmefällen abzuweichen.

\subsubsection{Bilanzidentitätsprinzip}
\highlight{Bilanzidentitätsprinzip} (\hgb[Abs. 1 Nr. 1]{252}): Die Wertansätze in einer Bilanz zum Jahresbeginn müssen identisch mit denen der Schlussbilanz des vorherigen Geschäftsjahres sein.

\subsubsection{Fortführungsprinzip}
\highlight{Fortführungsprinzip} (\hgb[Abs. 1 Nr. 2]{252}): Es ist grundsätzlich von der Fortführung der Unternehmenstätigkeit auszugehen, sofern nicht \textbf{tatsächliche} (z.B. massive wirtschaftliche Schwierigkeiten) oder \textbf{rechtliche} Gegebenheiten (z.B. ein gestellter Insolvenzantrag) dagegen sprechen.\par
Ist eine Fortführung nicht zu unterstellen, sind Liquidationswerte anzusetzen. Ansonsten richtet sich die Bewertung nach den Vorschriften der \hgbb[253 - 256a]{253}.

\subsubsection{Stichtagsprinzip}
\highlight{Stichtagsprinzip} (\hgb[Abs. 1 Nr. 3]{252}): Die Bewertung aller Vermögensgegenstände und Schulden erfolgt zu einem bestimmten Stichtag.

\subsubsection{Einzelbewertungsprinzip}
\highlight{Einzelbewertungsprinzip} (\hgb[Abs. 1 Nr. 3]{252}, \estG[Abs. 1 S. 1]{6}): Jedes Wirtschaftsgut ist einzeln zu bewerten, Synergieeffekte zwischen den Vermögensgegenständen werden nicht berücksichtigt.

\subsubsection{Vorsichtsprinzip}
\highlight{Vorsichtsprinzip} (\hgb[Abs. 1 Nr. 4]{252}): Vermögensgegenstände und Schulden sind vorsichtig zu bewerten, insbesondere sind alle vorhersehbaren Risiken und Verluste, die bis zum Abschlussstichtag entstanden sind, zu berücksichtigen.\par
Liegen keine Erwartungswerte vor, ist für Vermögensgegenstände der niedrigste, für Schulden der höchste Wert anzusetzen.

\subsubsection{Realisationsprinzip}
\highlight{Realisationsprinzip} (\hgb[Abs. 1 Nr. 4]{252}) und0 \highlight{Periodisierungsprinzip} (\hgb[Abs. 1 Nr. 5]{252}): Gewinne werden nur berücksichtigt, wenn sie am Abschlussstichtag realisiert sind.\par
Aufwendungen und Erträge sind unabhängig von den Zeitpunkten der tatsächlichen Zahlung anzusetzen. Der Realisationszeitpunkt wird im Allgemeinen an den Zeitpunkt der Lieferung/Leistung geknüpft.

\subsubsection{Stetigkeitsprinzip}
\highlight{Stetigkeitsprinzip} (\hgb[Abs. 1 Nr. 6]{252}): In den Folgejahren sind die auf den vorhergehenden Jahresabschluss angewandten Bewertungsmethoden beizubehalten, sofern nicht begründete Ausnahmefälle i.S.d. \hgb[Abs. 1 Nr. 2]{252} vorliegen.

\subsubsection{Historische bzw. fortgeführte Anschaffungs- oder Herstellungskosten}
\highlight{Anschaffungskosten} bilden den Bewertungsmaßstab für alle von Dritten erworbenen Vermögensgegenstände und sind gleichzeitig gemäß \hgb[Abs. 1]{253} die Wertobergrenze.\par
\highlight{Herstellungskosten} sind Aufwendungen, die im Rahmen des ursprünglichen oder nachträglichen (Wiederherstellung, Veränderung, Erweiterung, etc.) Herstellungsvorgangs eines Vermögensgegenstands anfallen (z.B. durch den Verbrauch von Gütern oder die Inanspruchnahme von Dienstleistungen für die Herstellung).

\subsubsection{Börsen- oder Marktpreis}
\hgb[Abs. 4]{253} definiert eine Hierarchie der möglichen Bewertungsmaßstäbe zur Bestimmung potenzieller außerplanmäßiger Abschreibungen für das Umlaufvermögen.\par
Vorrangig ist zu prüfen, ob der Börsen- oder Marktpreis am Abschlussstichtag unter den Anschaffungs- oder Herstellungskosten liegt. Der \highlight{Börsenpreis} ist der an einer amtlich anerkannten Börse im In- oder Ausland amtlich oder im Freiverkehr festgestellter Kurs\par
\highlight{Marktpreise} liegen vor, wenn Güter einer bestimmten Gattung und von durchschnittlicher Art und Güte an anderen Handelsplätzen regelmäßig umgesetzt werden.\par
Lässt sich weder ein Börsen- noch ein Marktpreis feststellen, stellt der Wert, der den Vermögensgegenständen am Abschlussstichtag beizulegen ist, den relevanten Vergleichswert dar.

\subsubsection{Erfüllungsbetrag}
Der \highlight{Erfüllungsbetrag} stellt den grundsätzlichen Wertansatz für Verbindlichkeiten dar. Es ist der Betrag, der für eine normale Abwicklung des Geschäfts notwendig ist, um die Verpflichtungen zu erfüllen.

\subsection{Verlustverrechnung bei der Einkommensteuer}
Gemäß \estG[Abs. 4]{2} ist das \hyperref[sec:ermittlungDesEinkommens]{Einkommen} der \textbf{Gesamtbetrag der Einkünfte}, vermindert um die Sonderausgaben und die außergewöhnlichen Belastungen.

\subsubsection{Verlustabzug: Verlustrücktrag}\label{sec:verlustruecktrag}
Gemäß \estG[Abs. 1]{10d} sind \textbf{negative Einkünfte}, die bei der Ermittlung des Gesamtbetrags der Einkünfte nicht ausgeglichen werden bis zu einem Betrag von 1.000.000 Euro \dots abzuziehen (\highlight{Verlustrücktrag}).\par
Gemäß \estG[Abs. 1 S. 6]{10d} ist \textbf{auf Antrag} des Steuerpflichtigen von der Anwendung des Verlustrücktrags abzusehen.

\subsubsection{Verlustabzug: Verlustvortrag}
Gemäß \estG[Abs. 2]{10d} sind \textbf{nicht ausgeglichene negative Einkünfte}, die nicht nach \hyperref[sec:verlustruecktrag]{Absatz 1} abgezogen worden sind in den folgenden Veranlagungszeiträumen bis zu einem Gesamtbetrag der Einkünfte von 1.000.000 Euro unbeschränkt, \textbf{darüber hinaus} bis zu 60 Prozent \dots abzuziehen.

\subsection{Einkünfte aus Gewerbebetrieb}
Gemäß \estG{15} sind \highlight{Einkünfte aus Gewerbebetrieb}
\begin{enumerate}
  \item{\textbf{Einkünfte aus gewerblichen Unternehmen} \dots}
  \item{die \textbf{Gewinnanteile der Gesellschafter} einer Offenen Handelsgesellschaft, einer Kommanditgesellschaft und einer anderen Gesellschaft, bei der der Gesellschafter als Unternehmer (Mitunternehmer) des Betriebs anzusehen ist,\\
              und die \textbf{Vergütungen}, die der Gesellschafter von der Gesellschaft
              \begin{itemize}
                \item{für seine Tätigkeit im Dienst der Gesellschaft (\textbf{Gehalt})}
                \item{für die Hingabe von Darlehen (\textbf{Zinsen})}
                \item{für die Überlassung von Wirtschaftsgütern (\textbf{Mieten})}
              \end{itemize}
              bezogen hat.
        }
  \item{die \textbf{Gewinnanteile und Vergütungen} der persönlich haftenden Gesellschafter (analog)}
\end{enumerate}
\estG[Abs. 1 Satz 1 Nr. 2]{15} gilt analog (``entsprechend anzuwenden'') für \textbf{Einkünfte aus Land- und Forstwirtschaft} (\estG[Abs. 7]{13}) und \textbf{Einkünfte aus selbständiger Arbeit} (\estG[Abs. 4]{18})

\subsubsection{Transparenzprinzip}\label{sec:transparenzprinzip}
Im Einkommensteuerrecht gilt das \highlight{Transparenzprinzip}, d.h. es wird nicht die Gesellschaft besteuert, sondern die Gesellschafter/Mitunternehmer.

\subsubsection{Mitunternehmer}
\textbf{Mitunternehmer} i.S.d. \estG[Abs. 1 Nr. 2]{15} ist, wer zivilrechtlich Gesellschafter einer Personengesellschaft ist und eine gewisse unternehmerische Initiative entfalten (Entscheidungen treffen) kann sowie unternehmerisches Risiko (Teilnahme am Erfolg oder Misserfolg des Unternehmens, z.B. Beteiligung am Gewinn oder Verlust) trägt.

\subsubsection{Definition: Gewerbebetrieb}
Der \highlight{Gewerbebetrieb} ist in \estG[Abs. 2]{15} definiert:\par
Eine
\begin{itemize}
  \item{selbstständige}
  \item{nachhaltige Betätigung, die}
  \item{mit der Absicht, Gewinn zu erzielen, unternommen wird und}
  \item{sich als Beteiligung am allgemeinen wirtschaftlichen Verkehr darstellt}
\end{itemize}
ist Gewerbebetrieb, wenn die Betätigung weder
\begin{itemize}
  \item{als Ausübung von Land- und Forstwirtschaft noch}
  \item{als Ausübung eines freien Berufs noch als eine andere selbständige Arbeit anzusehen ist}
\end{itemize}

\subsubsection{Tatbestandsmerkmale: Gewerbebetrieb}
Erklärungen zu den vier Tatbestandsmerkmalen:
\begin{itemize}
  \item{\textbf{selbstständig}: Tätigkeit auf eigene Rechnung (Risiko) und eigene Verantwortung (Initiative)}
  \item{\textbf{nachhaltige Betätigung}: Keine einmalige Tätigkeit}
  \item{\textbf{Absicht, Gewinn zu erzielen}: Subjektives Merkmal, Absicht über längere Dauer positive Einkünfte zu erzielen}
  \item{\textbf{allgemeiner wirtschaftlicher Verkehr}: Nachhaltige Teilnahme am Leistungs- oder Güteraustausch}
\end{itemize}

\subsection{Einkünfte aus selbständiger Arbeit}
Gemäß \estG[Abs. 1]{15} sind \highlight{Einkünfte aus selbständiger Arbeit}
\begin{enumerate}
  \item{\textbf{Einkünfte aus freiberuflicher Tätigkeit}, dazu gehörigen die selbständig ausgeübte \textbf{wissenschaftliche}, \textbf{künstlerische}, \textbf{schriftstellerische}, \textbf{unterrichtende oder erzieherische} Tätigkeit und die selbständige Berufstätigkeit der Ärzte, Rechtsanwälte, Notare, Ingenieure, etc. (\highlight{Katalogberufe})}
  \item{\textbf{Einkünfte der Einnehmer einer staatlichen Lotterie}, wenn sie nicht Einkünfte aus Gewerbebetrieb sind}
  \item{\textbf{Einkünfte aus sonstiger selbständiger Arbeit}, z. B. Vergütungen für die Vollstreckung von Testamenten, für Vermögensverwaltung und für die Tätigkeit als Aufsichtsratsmitglied}
  \item{\textbf{Einkünfte, die ein Beteiligter an einer vermögensverwaltenden Gesellschaft oder Gemeinschaft} \dots als Vergütung \dots erzielt}
\end{enumerate}

\section{Körperschaftsteuer}
Die \highlight{Körperschaftsteuer} ist die Ertragsteuer für \textbf{juristische Personen} (insbesondere Kapitalgesellschaften, z.B. AG und GmbH) und andere Personenvereinigungen, die nicht Mitunternehmerschaften i.S.d. EStG sind, z.B. Vereine und Vermögensmassen.\par
``Die Körperschaftsteuer ist die Einkommensteuer der Körperschaften''.

\subsection{Körperschaftsteuer und Einkommensteuer}
Die Körperschaftsteuer ist ebenfalls eine \textbf{direkte Steuer} und ist eine Personensteuer, die nicht vom Einkommen abgezogen werden darf. Steuerliche Doppelbelastungen werden über verschiedene Maßnahmen berücksichtigt (\textbf{Teileinkünfteverfahren} oder \textbf{Abgeltungsteuer}) \par
Es gelten weitgehend die Grundsätze und Vorschriften des Einkommensteuerrechts, z.B. für die Gewinnermittlung.

\subsection{Tatbestandsmerkmale}\label{sec:kstTatbestandsmerkmale}
Gemäß \kstG[Abs. 1]{1} sind
\begin{itemize}
  \item{(die folgenden) Körperschaften, Personenvereinigungen und Vermögensmassen, die}
  \item{im Inland}
  \item{ihre Geschäftsleitung oder ihren Sitz haben}
\end{itemize}
unbeschränkt körperschaftsteuerpflichtig.\par
Zu Punkt 1 gehören unter anderem Kapitalgesellschaften (z.B. Aktiengesellschaften), Genossenschaften, Versicherungsfonds und Vereine. Die steuerpflichtigen juristischen Personen sind \textbf{abschließend aufgezählt!}

\subsubsection{Inland}
Das Inlandsbegriff des Körperschaftsteuerrechts ist in \kstG[Abs. 3]{1} definiert.

\subsubsection{Geschäftsleitung und Sitz}
Die \textbf{Geschäftsleitung} ist gemäß \abgabenordnung{10} der Mittelpunkt der geschäftlichen Oberleitung.\par
Der \textbf{Sitz} ist gemäß \abgabenordnung{11} der Ort, der durch Gesetz, Vertrag, Satzung, \dots bestimmt ist.

\subsubsection{Unbeschränkte Körperschaftsteuerpflicht}
Auch im Körperschaftsteuerrecht gilt das \hyperref[sec:welteinkommensprinzip]{Welteinkommensprinzip}, die unbeschränkte Körperschaftsteuerpflicht erstreckt sich auf sämtliche Einkünfte.

\subsection{Sachliche Steuerpflicht}
\subsubsection{Grundlagen der Besteuerung}
Die Grundlagen der Besteuerung sind in \kstG{7} definiert:
\begin{itemize}
  \item{Die Körperschaftsteuer bemisst sich nach dem zu versteuernden Einkommen (\kstG[Abs. 1]{7})}
  \item{Zu versteuerndes Einkommen ist das Einkommen im Sinne des \estG[Abs. 1]{8} vermindert um die Freibeträge der \kstGG[24 und 25]{24}}
  \item{Körperschaftsteuer wird in der Regel jeweils für ein Kalenderjahr ermittelt (\kstG[Abs. 3]{7})}
  \item{Bei Steuerpflichtigen die verpflichtet sind Bücher zu führen, ist der Gewinn nach dem Wirtschaftsjahr und nicht nach dem Kalenderjahr zu ermitteln (\kstG[Abs. 4]{7})}
\end{itemize}

\subsubsection{Ermittlung des Einkommens}
Die \textbf{Ermittlung des Einkommens} ist in \kstG{8} beschrieben. Gemäß \kstG[Abs. 1 Satz 1]{8} bestimmen die Vorgaben des Körperschaftsteuergesetzes und des Einkommensteuergesetzes die Art der Einkommensermittlung.\par
Bei \hyperref[sec:kstTatbestandsmerkmale]{unbeschränkt Steuerpflichtigen} im Sinne des \kstG[Abs. 1 Nr. 1 bis 3]{1} sind gemäß \kstG[Abs. 2]{8} \textbf{alle Einkünfte} als Einkünfte aus Gewerbebetrieb zu behandeln.


\subsection{Beschränkte Steuerpflicht}
Gemäß \kstG[Nr. 1]{2} sind Körperschaften, Personenvereinigungen und Vermögensmassen, die \textbf{weder ihre Geschäftsleitung noch ihren Sitz im Inland haben} beschränkt steuerpflichtig.\par
Die Steuerpflicht erstreckt sich dann nur auf die \textbf{inländischen Einkünfte}.

\subsection{Trennungsprinzip}
Während für Personengesellschaften das \hyperref[sec:transparenzprinzip]{Transparenzprinzip} gilt, gilt \textbf{bei Kapitalgesellschaften} das \highlight{Trennungsprinzip}. Dieses sieht eine vollständige Trennung der Besteuerung der Kapitalgesellschaft von der des Anteilseigners vor.\par
Laufende Gewinne der Kapitalgesellschaft werden ausschließlich auf Ebene der Gesellschaft besteuert (\kstG[Abs. 3 Satz 1]{8}). Die Ebene des Anteilseigners wird nur angesprochen, wenn Gewinnausschüttungen bewirkt werden (\estG[Abs. 1 Nr. 1]{20}).

\subsubsection{Zuflussprinzip im Einkommensteuerrecht}
Das Trennungsprinzip bewirkt, dass nur \textbf{beschlossene und zur Auszahlung gebrachte} Gewinnausschüttungen zu Einnahmen bei den Gesellschaftern führen (\highlight{Zuflussprinzip} genannt, \estG{11}).\par
Gewinne können so verzögert steuerlich vorteilhaft für die Gesellschafter ausgeschüttet werden.

\subsubsection{Rechtsgeschäft zwischen dem Gesellschafter und der Gesellschaft}
Soweit der Gesellschafter (Anteilseigner) eine natürliche Person oder Personengesellschaft ist, richtet sich die Besteuerung dort nach dem EStG. Ist der Gesellschafter wiederum eine Kapitalgesellschaft, so ist die Frage nach der Besteuerung nach dem Körperschaftsteuerrecht zu beurteilen.\par
Dadurch weicht das System der Besteuerung von Kapitalgesellschaften erheblich von dem der Besteuerung von Personengesellschaften ab. Während
\begin{itemize}
  \item{Vergütungen aus Anstellungsverträgen}
  \item{Zinsen aus Gesellschafterdarlehen und}
  \item{Mieten aus Grundstücksüberlassungen an die Gesellschaft}
\end{itemize}
bei Gesellschaftern von Personengesellschaften grundsätzlich zu den Einkünften aus Gewerbebetrieb zählen, erfolgt eine solche Umqualifizierung \textbf{bei Kapitalgesellschaften nicht}.\par
Vergütungen aus dem Anstellungsverhältnis bei einer GmbH stellen beim Gesellschafter \textbf{Einkünfte aus nichtselbständiger Arbeit} dar. Zinseinkünfte sind \textbf{Einkünfte aus Kapitalvermögen} und Miet- und Pachteinnahmen sind \textbf{Einnahmen aus Vermietung und Verpachtung}.\par
Die genannten Vergütungen wären bei Gesellschaftern einer Personengesellschaft \textbf{Einkünfte aus Gewerbebetrieb}, wodurch die Gewinnfeststellung einheitlich erfolgen würde. Das ist bei Kapitalgesellschaften nicht der Fall.\par

\end{document}