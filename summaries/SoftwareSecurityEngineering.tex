\documentclass[12pt,A4]{extarticle}	
\usepackage[table]{xcolor}
\usepackage{natbib}

\begin{filecontents}[overwrite]{\jobname.bib}
  @techreport{ISO22300,
  title        = {Security and resilience — Vocabulary},
  author       = {{International Organization for Standardization}},
  institution  = {{ISO}},
  number       = {ISO 22300:2021},
  year         = {2021},
  address      = {Geneva, Switzerland},
  url          = {https://www.iso.org/standard/77008.html}
  }
  @inproceedings{ComparingAndEvaluatingCVSSBaseMetrics,
  author = {Mussa, Awad A Younis and Malaiya, Yashwant},
  year = {2015},
  month = {08},
  pages = {},
  title = {Comparing and Evaluating CVSS Base Metrics and Microsoft Rating System},
  doi = {10.1109/QRS.2015.44}
  }
  @techreport{ISOIEC27000,
  title        = {Information technology — Security techniques — Information security management systems — Overview and vocabulary},
  author       = {{International Organization for Standardization and International Electrotechnical Commission}},
  institution  = {{ISO/IEC JTC 1/SC 27}},
  number       = {ISO/IEC 27000:2018},
  year         = {2018},
  address      = {Geneva, Switzerland},
  url          = {https://www.iso.org/standard/73906.html}
  }
  @inbook{TowardANewFrameworkForInformationSecurity,
  author = {Parker, Donn},
  year = {2015},
  month = {09},
  pages = {3.1-3.23},
  title = {Toward a New Framework for Information Security?},
  isbn = {9781118134108},
  doi = {10.1002/9781118851678.ch3}
  }
  @book{boehm1981software,
  title={Software Engineering Economics},
  author={Boehm, B.W.},
  isbn={9780138221225},
  lccn={81013889},
  series={Prentice-Hall advances in computing science and technology series},
  url={https://books.google.de/books?id=mpZQAAAAMAAJ},
  year={1981},
  publisher={Prentice-Hall}
  }
\end{filecontents}

\newcommand{\lectureTitle}{Software Security Engineering [WIP]}
\newcommand{\semester}{Sommersemester 2025}

\usepackage[a4paper,left=0.9cm,right=1cm,top=1.37cm,bottom=2.5cm]{geometry}
\usepackage[utf8]{inputenc}
\usepackage{xifthen}
\usepackage{cmbright}
\usepackage{fontawesome}
\usepackage[T1]{fontenc}
\usepackage{lastpage,lipsum}
\usepackage{hyperref}
\usepackage{transparent}
\usepackage{color}
\usepackage{fancyhdr}

\renewcommand*\familydefault{\sfdefault}
\setlength{\parindent}{0mm}

\usepackage{transparent}
\usepackage{color}
\usepackage{fancyhdr}

\definecolor{headerBg}{RGB}{11, 67, 158}
\definecolor{headerGrayColor}{RGB}{210, 210, 210}

\pagestyle{fancy}
\fancyhead[C]{
  \fcolorbox{headerBg}{headerBg}{
    \hspace{0.6cm}\begin{minipage}[c][50pt][c]{\paperwidth}
      \begin{minipage}[c]{.45\textwidth}
        \huge{\textcolor{white}{Vorlesung}}\normalsize\\
        \textcolor{headerGrayColor}{\small{Wintersemester 2022/23}}
      \end{minipage}%
      \begin{minipage}[c]{.45\textwidth}
        \raggedleft
        \textcolor{white}{
          \small{\href{https://nilslambertz.de/}{nilslambertz.de}}\\
          \href{https://github.com/nilslambertz/}{\textcolor{white}{\faicon{github}} \small{nilslambertz}}}
      \end{minipage}
    \end{minipage}}
}
\renewcommand{\headrulewidth}{0pt}
\setlength{\headheight}{40pt}

\newlength{\oddmarginwidth}
\setlength{\oddmarginwidth}{1in+\hoffset+\oddsidemargin}
\newlength{\evenmarginwidth}
\setlength{\evenmarginwidth}{\evensidemargin+1in}
\fancyhfoffset[LO,RE]{\oddmarginwidth}
\fancyhfoffset[LE,RO]{\evenmarginwidth}
\cfoot{\thepage\ $/$ \pageref*{LastPage}}

\definecolor{highlightColor}{RGB}{66, 135, 245}
\newcommand{\highlight}[1]{\textcolor{highlightColor}{\textbf{#1}}}

\def\contentsname{\empty}

\begin{document}
\defcitealias{ISO22300}{ISO 22300:2021}
\defcitealias{ISOIEC27000}{ISO/IEC 27000:2018}

\disclaimer

\tableofcontents
\clearpage

\section{Introduction}
\subsection{Motivation}
Influencing factors of secure software include
\begin{itemize}
  \item{\textbf{Security features}: Authentication, Authorization, Access control, cryptography, etc.}
  \item{\textbf{Technology}: Programming languages, development tools, etc.}
  \item{\textbf{Operational environments}: Firewalls, Intrusion detection systems, etc.}
\end{itemize}

\subsection{Terminology}
\begin{itemize}
  \item{\textbf{Asset}: anything that has value to an organization, including human, physical, information, intangible and environmental resources \citepalias{ISO22300}}
  \item{\textbf{Thread}: potential cause of an unwanted incident, which could result in harm to individuals, assets, a system or organization \citepalias{ISO22300}}
  \item{\textbf{Adversary}: any person or a thing that acts (or has the power to act) to cause, carry, transmit, or support a threat \cite{ComparingAndEvaluatingCVSSBaseMetrics}}
  \item{\textbf{Vulnerability}: weakness of an asset or control that can be exploited by one or more threats \citepalias{ISOIEC27000}}
  \item{\textbf{Exploit}: method that identifies and takes advantage of a vulnerability in an asset \cite{ComparingAndEvaluatingCVSSBaseMetrics}}
  \item{\textbf{Attack}: attempt to destroy, expose, alter, disable, steal or gain unauthorized access to or make unauthorized use of an asset \citepalias{ISOIEC27000}}
\end{itemize} 

\subsection{Protection Goals}
\begin{itemize}
  \item{\highlight{Confidentiality}: property that information is not made available or disclosed to unauthorized individuals, entities, or processes \citepalias{ISOIEC27000}}
  \item{\highlight{Integrity}: property of accuracy and completeness \citepalias{ISOIEC27000}}
  \item{\highlight{Availability}: property of being accessible and usable on demand by an authorized entity \citepalias{ISOIEC27000}}
  \item{\highlight{Authenticity}: property that an entity is what it claims to be \citepalias{ISOIEC27000}}
  \item{\highlight{Possession}: holding, controlling, and having the ability to use information \cite{TowardANewFrameworkForInformationSecurity}}
  \item{\highlight{Utility}: usefulness of information \cite{TowardANewFrameworkForInformationSecurity}}
\end{itemize}

\section{Security Requirements Engineering}
\subsection{What makes security requirements special?}
\begin{itemize}
  \item{typically \textbf{quality/non-functional} requirements}
  \item{often \textbf{negative} requirements: Describe what the system \textbf{should not} do, hard to validate}
  \item{often \textbf{immeasurable} as there is no clear satisfaction criteria}
  \item{often \textbf{incalculable}: Cost-benefit ratio hard to determine}
  \item{often \textbf{uncertain}, hard to identify without knowing the whole system design upfront}
  \item{hard to recognize vulnerabilities during productive/test operation, not ``implicitly'' tested by stakeholders/users when they use the system}
\end{itemize}

\subsection{Requirements Engineering - Core Activities}
\begin{itemize}
  \item{\textbf{Elicitation}: Obtain requirements from stakeholders, refine them}
  \item{\textbf{Documentation}: Describe requirements adequately using natural language or models}
  \item{\textbf{Validation and Negotation}: Guarantee that requirements meet quality criteria}
  \item{\textbf{Management}: Structure requirements, maintain consistency, ensure implementation}
\end{itemize}

\subsubsection{Verification and Validation (V\&V)}
\begin{itemize}
  \item{\textbf{Verification}: ``Are we building the product right?'' \cite{boehm1981software}
    \begin{itemize}
      \item{assurance, that a product, service or system meets the \textbf{needs} of the customers and stakeholders}
      \item{often involves acceptance and suitability}
    \end{itemize}
  }
  \item{\textbf{Validation}: ``Are we building the right product?'' \cite{boehm1981software}}
\end{itemize}

\subsection{Security Requirements - Definition}
\begin{itemize}
  \item{\textbf{Security needs}: Design objectives concerning the protection of its stakeholders from consequences of (intentional) threats that conflict with stakeholders' goals}
  \item{\textbf{Security requirements}: Express security needs in a form suitable to be used and make its results verifiable}
\end{itemize}

\subsection{Security Needs - Influencing Factors}
\subsubsection{Goals}
\highlight{Security goals} describe the desired protection of a system and its environment:
\begin{itemize}
  \item{Assets (e.g. documents) and their protection needs}
  \item{Dreaded consequences (e.g. loss of reputation)}
  \item{Organizational policies (e.g. only staff with security clearance can access classified information)}
  \item{Legal or business requirements (e.g. data protection laws)}
\end{itemize}

\subsubsection{Threats}
A \highlight{threat} statement describes adversaries - their goals, capabilities or expected behavior:
\begin{itemize}
  \item{Capabilities (e.g. adversary controls a botnet)}
  \item{Objectives (e.g. work for profit and aim to evade apprehension)}
  \item{Target selection behavior (e.g. opportunistic, targeted)}
  \item{Technical attack patterns (e.g. transparent rerouting of network traffic)}
\end{itemize} 

\subsubsection{Design}
Security needs can be described by the \highlight{design} of a system or software:
\begin{itemize}
  \item{Required security functions (e.g. logging)}
  \item{Design details with security implications (e.g. long random session IDs)}
  \item{Common volnerabilities to avoid (e.g. XSS)}
\end{itemize}

\newpage
\subsection{Abuse vs. Misuse}
\highlight{Abuse} described \textbf{malicious} and \textbf{deliberate} acts.\par
\highlight{Misuse} describes \textbf{spontaneous} acts and possibly \textbf{careless} use of a system.\par

\subsubsection{Misuse Cases}
``A misuse case is the inverse of a use case, a function that the system \textbf{should not allow}.''. It causes harm to some stakeholder if the sequence is allowed to complete.

\begin{itemize}
  \item{Misuse cases can be related to regular use cases}
  \item{\textbf{Threaten} relationship: Use case is \textbf{exploited} or \textbf{hindered} by a misuse case}
  \item{\textbf{Mitigate} relationship (of a ``Security use case''): Use case is \textbf{countermeasure} against a misuse case, reduces the misuse case's chance of success}
\end{itemize}

\subsection{SQUARE Process Model}
Security Quality Requirements Engineering (\highlight{SQUARE}) is a process model for categorizing and prioritizing security requirements. The focus is to build security concept into early stages of the development lifecycle.

\subsubsection{SQUARE Steps}
\begin{enumerate}
  \item{\textbf{Agree on Definitions}:
    \begin{itemize}
      \item{ensure that stakeholders agree of definitions of terms}
      \item{Participants: Stakeholders, requirements engineers}
    \end{itemize}
  }
  \item{\textbf{Identify Security Goals}:
    \begin{itemize}
      \item{identify protection goals of stakeholders and prioritize them}
      \item{Participants: Stakeholders, requirements engineers}
    \end{itemize}
  }
  \item{\textbf{Develop Artifacts}:
    \begin{itemize}
      \item{develop artifacts to support security requirements definition}
      \item{Participants: Requirements engineers}
    \end{itemize}
  }
  \item{\textbf{Perform Risk Assessment}:
    \begin{itemize}
      \item{assess risks using a method recommended by a risk expert}
      \item{Participants: Requirements engineers, risk experts, stakeholders}
    \end{itemize}
  }
  \item{\textbf{Select Elicitation Technique}:
    \begin{itemize}
      \item{overcome communication issues between stakeholders by selecting a proper elicitation technique}
      \item{Participants: Requirements engineers}
    \end{itemize}
  }
  \item{\textbf{Elicit Security Requirements}:
    \begin{itemize}
      \item{actual elicitation process using the techniques selected in the previous step}
      \item{Participants: Stakeholders}
    \end{itemize}
  }
  \item{\textbf{Categorize Requirements}:
    \begin{itemize}
      \item{categorize requirements as to level (e.g. system, software) and if they are quality requirements or constraints}
      \item{Participants: Requirements engineers, other specialists}
    \end{itemize}
  }
  \item{\textbf{Prioritize Requirements}:
    \begin{itemize}
      \item{possible cost-benefit analysis considering consequences}
      \item{Participants: Stakeholders}
    \end{itemize}
  }
  \item{\textbf{Inspect Requirements}:
    \begin{itemize}
      \item{Peer-review of requirements for problems, concerns, workload etc.}
      \item{Participants: Inspection team}
    \end{itemize}
  }
\end{enumerate}

\newpage
\bibliographystyle{apalike}
\bibliography{\jobname}

\end{document}