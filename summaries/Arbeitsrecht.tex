\documentclass[12pt,A4]{extarticle}	
\usepackage[table]{xcolor}

\newcommand{\lectureTitle}{Arbeitsrecht [unvollständig]}
\newcommand{\semester}{Sommersemester 2024}

\usepackage[a4paper,left=0.9cm,right=1cm,top=1.37cm,bottom=2.5cm]{geometry}
\usepackage[utf8]{inputenc}
\usepackage{xifthen}
\usepackage{cmbright}
\usepackage{fontawesome}
\usepackage[T1]{fontenc}
\usepackage{lastpage,lipsum}
\usepackage{hyperref}
\usepackage{transparent}
\usepackage{color}
\usepackage{fancyhdr}

\renewcommand*\familydefault{\sfdefault}
\setlength{\parindent}{0mm}

\definecolor{headerBg}{RGB}{11, 67, 158}
\definecolor{headerGrayColor}{RGB}{210, 210, 210}

\newcommand{\printTitle}{\textcolor{white}{\lectureTitle}\normalsize}
\newcommand{\printSubtitle}{
  \ifdefined\lectureSubtitle
    \textcolor{white}{\small{\lectureSubtitle}}\\
  \fi
}

\fancyhf{}
\pagestyle{fancy}
\fancyhead[C]{
  \fcolorbox{headerBg}{headerBg}{
    \hspace{0.6cm}\begin{minipage}[c][50pt][c]{\paperwidth}
      \begin{minipage}[c]{.7\textwidth}
        \ifdefined\titleSize
          \titleSize \printTitle\\
        \else
          \huge\printTitle\\
        \fi
        \printSubtitle
        \textcolor{headerGrayColor}{\small{\semester}}
      \end{minipage}%
      \begin{minipage}[c]{.2\textwidth}
        \raggedleft
        \textcolor{white}{
          \small{\href{mailto:mail@nilslambertz.de}{\textcolor{white}{\faicon{envelope}} mail@nilslambertz.de}}\\
          \href{https://github.com/nilslambertz/kit-zusammenfassungen}{\textcolor{white}{\faicon{github}} \small{nilslambertz}}}
      \end{minipage}
    \end{minipage}}
}
\renewcommand{\headrulewidth}{0pt}
\setlength{\headheight}{40pt}

\newlength{\oddmarginwidth}
\setlength{\oddmarginwidth}{1in+\hoffset+\oddsidemargin}
\newlength{\evenmarginwidth}
\setlength{\evenmarginwidth}{\evensidemargin+1in}
\fancyhfoffset[LO,RE]{\oddmarginwidth}
\fancyhfoffset[LE,RO]{\evenmarginwidth}
\cfoot{\thepage\ $/$ \pageref*{LastPage}}

\definecolor{highlightColor}{RGB}{66, 135, 245}
\newcommand{\highlight}[1]{\textcolor{highlightColor}{\textbf{#1}}}
\definecolor{gesetzLink}{RGB}{194, 74, 14}
\newcommand{\bgb}[2][]{\textbf{\textcolor{gesetzLink}{\href{https://www.gesetze-im-internet.de/bgb/__#2.html}{§ #2 \ifthenelse{\equal{#1}{}}{}{#1 }BGB}}}}
\newcommand{\agG}[2][]{\textbf{\textcolor{gesetzLink}{\href{https://www.gesetze-im-internet.de/agg/__#2.html}{§ #2 \ifthenelse{\equal{#1}{}}{}{#1 }AGG}}}}
\newcommand{\agGG}[2][]{\textbf{\textcolor{gesetzLink}{\href{https://www.gesetze-im-internet.de/agg/__#2.html}{§§ #1 AGG}}}}
\newcommand{\nachwG}[2][]{\textbf{\textcolor{gesetzLink}{\href{https://www.gesetze-im-internet.de/nachwg/__#2.html}{§ #2 \ifthenelse{\equal{#1}{}}{}{#1 }NachwG}}}}
\newcommand{\tzbfG}[2][]{\textbf{\textcolor{gesetzLink}{\href{https://www.gesetze-im-internet.de/tzbfg/__#2.html}{§ #2 \ifthenelse{\equal{#1}{}}{}{#1 }TzBfG}}}}
\newcommand{\kschG}[2][]{\textbf{\textcolor{gesetzLink}{\href{https://www.gesetze-im-internet.de/kschg/__#2.html}{§ #2 \ifthenelse{\equal{#1}{}}{}{#1 }KSchG}}}}
\newcommand{\muschG}[2][]{\textbf{\textcolor{gesetzLink}{\href{https://www.gesetze-im-internet.de/muschg_2018/__#2.html}{§ #2 \ifthenelse{\equal{#1}{}}{}{#1 }MuSchG}}}}
\newcommand{\beeG}[2][]{\textbf{\textcolor{gesetzLink}{\href{https://www.gesetze-im-internet.de/beeg/__#2.html}{§ #2 \ifthenelse{\equal{#1}{}}{}{#1 }BEEG}}}}
\newcommand{\bdsG}[2][]{\textbf{\textcolor{gesetzLink}{\href{https://www.gesetze-im-internet.de/bdsg_2018/__#2.html}{§ #2 \ifthenelse{\equal{#1}{}}{}{#1 }BDSG}}}}
\newcommand{\sgbIX}[2][]{\textbf{\textcolor{gesetzLink}{\href{https://www.gesetze-im-internet.de/sgb_9_2018/__#2.html}{§ #2 \ifthenelse{\equal{#1}{}}{}{#1 }SGB IX}}}}
\newcommand{\betrvG}[2][]{\textbf{\textcolor{gesetzLink}{\href{https://www.gesetze-im-internet.de/betrvg/__#2.html}{§ #2 \ifthenelse{\equal{#1}{}}{}{#1 }BetrVG}}}}
\newcommand{\burlG}[2][]{\textbf{\textcolor{gesetzLink}{\href{https://www.gesetze-im-internet.de/burlg/__#2.html}{§ #2 \ifthenelse{\equal{#1}{}}{}{#1 }BUrlG}}}}
\newcommand{\GG}[2][]{\textbf{\textcolor{gesetzLink}{\href{https://www.gesetze-im-internet.de/gg/art_#2.html}{Art #2 \ifthenelse{\equal{#1}{}}{}{#1 }GG}}}}

\def\contentsname{\empty}

\begin{document}

\disclaimer

\vspace{1cm}
\textbf{Vorlesung 4 und Vorlesung 10 (11.07.2024) fehlen in der Zusammenfassung.}
\vspace{-1cm}

\tableofcontents
\clearpage

\section{Grundlagen}
\subsection{Arbeitsvertrag vs. Dienstvertrag}
Bei \textbf{Arbeitsverträgen} (oder Dienstverträgen) steht der \textbf{Dienst} im Vordergrund. Der Arbeiternehmer muss den Dienst leisten, wird jedoch bezahlt, auch wenn der gewünschte Erfolg nicht eintritt.\par
Bei \textbf{Werkverträgen} steht der \textbf{Erfolg} im Vordergrund. Wenn das ``Werk'' nicht erbracht wird, muss der Auftragnehmer nicht bezahlt werden (z.B. Handwerker).

\subsection{Pflichten vor Beginn des Arbeitsvertrags}
Auch schon während der Bewerbungsphase existieren vorvertragliche Pflichten, auch wenn noch keine Verträge geschlossen wurden. \bgb[Abs. 2]{311} definiert gegenseitige Rücksichts- und Geheimhaltungspflichten.

\section{Allgemeines Gleichbehandlungsgesetz (Antidiskriminierungsgesetz)}
\subsection{Grundlagen}
\subsubsection{Ziel}\label{sec:aggZiel}
Das Ziel des Allgemeines Gleichbehandlungsgesetzes (auch \highlight{Antidiskriminierungsgesetz} genannt) ist in \agG{1} definiert. Es soll Benachteiligungen aufgrund bestimmter (aufgezählter) Merkmale verhindern.

\subsubsection{Benachteiligungsverbot}
Das \textbf{Benachteiligungsverbot} ist in \agG{7} definiert. Demnach dürften Beschäftigte nicht wegen der \hyperref[sec:aggZiel]{genannten Merkmale} benachteiligt werden.\par
Dies gilt auch, wenn das Merkmal gar nicht auf die Person zutrifft, sondern nur angenommen wird.

\subsubsection{Ausschreibungen}
Gemäß \agG{11} dürfen Arbeitsplätze nicht unter Verstoß gegen \agG[Abs. 1]{7} ausgeschrieben werden.

\subsection{Arten der Benachteiligung}
Die Arten der Benachteiligung sind in \agG{3} definiert.

\subsubsection{Unmittelbare Benachteiligung}
Eine \highlight{unmittelbare Benachteiligung} liegt gemäß \agG[Abs. 1]{3} vor, wenn eine Person aufgrund der \hyperref[sec:aggZiel]{genannten Merkmale} eine weniger günstige Behandlung erfährt als andere Personen.\par
Dies gilt ebenfalls für Frauen, die aufgrund einer Schwangerschaft oder Mutterschaft benachteiligt werden.

\subsubsection{Mittelbare Benachteiligung}
Eine \highlight{mittelbare Benachteiligung} liegt gemäß \agG[Abs. 2]{3} vor, wenn dem Anschein nach neutrale Vorschriften oder Verfahren Personen wegen der \hyperref[sec:aggZiel]{genannten Merkmale} benachteiligen, wenn sie nicht sachlich gerechtfertigt, angemessen und erforderlich sind.\par
Hierbei erfolgt keine direkte Diskriminierung wegen der Merkmale, sondern über Faktoren, die überproportional von dieser Gruppe erfüllt werden.\par
Ein Beispiel ist die Verweigerung von Teilzeitstellen (benachteiligt Frauen überproportional).

\subsubsection{Belästigung}
Eine \highlight{Belästigung} ist gemäß \agG[Abs. 3]{3} eine \textbf{unerwünschte Verhaltensweise}, die mit den \hyperref[sec:aggZiel]{genannten Merkmale} in Verbindung steht, die Würde der Person verletzt und das Arbeitsumfeld negativ beeinflussen.\par

\subsubsection{Sexuelle Belästigung}
Eine \highlight{sexuelle Belästigung} (\agG[Abs. 4]{3}) ist analog für sexuelle Verhaltensweisen definiert.\par
Darunter fallen sexuelle Handlungen, Aufforderungen, körperliche Berührungen, Bemerkungen sowie unerwünschtes Zeigen von pornographischen Darstellungen.

\subsection{Zulässige unterschiedliche Behandlung}
In \agGG[8-10]{8} sind Ausnahmen definiert, in denen unterschiedliche Behandlungen zulässig sind.

\subsubsection{Berufliche Anforderungen}
Wenn die Art der Tätigkeit oder die Bedingungen eine wesentliche und entscheidende berufliche Anforderung darstellt, ist gemäß \agG{8} eine unterschiedliche Behandlung zulässig.

\subsubsection{Religion oder Weltanschauung}
Für Religionsgemeinschaften und ihnen zugeordneten Einrichtungen ist gemäß \agG{9} eine Benachteiligung zulässig, wenn die Religion oder Weltanschauung eine gerechtfertigte berufliche Anforderung darstellt.\par
Eine Religionsgemeinschaft kann z.B. fordern, dass ein Pastor ihrer Religion angehört.

\subsubsection{Alter}
Gemäß \agG{10} ist eine unterschiedliche Behandlung wegen des Alters zulässig, wenn sie objektiv angemessen und durch ein legitimes Ziel gerechtfertigt ist.\par
Das können z.B. Mindest- oder Höchstalter für bestimmte Tätigkeiten sein.

\subsubsection{Unterschiedliche Behandlung wegen mehrerer Gründe}
Wenn die unterschiedliche Behandlung wegen mehrerer Gründe erfolgt, kann diese gemäß \agG{4} nur durch \agGG[8 bis 10 und 20]{8} nur gerechtfertigt sein, wenn die Rechtfertigung sich auf \textbf{alle diese Gründe erstreckt}.

\subsection{Folgen einer Benachteiligung}
\subsubsection{Pflichten des Arbeitgebers}
Gemäß \agG{12} verpflichtet, (auch vorbeugende) Maßnahmen zum Schutz vor Benachteiligungen zu treffen. Während der Aus- und Fortbildung muss er gemäß \agG[Abs. 2]{12} auf die Unzulässigkeit solcher Benachteiligungen hinweisen.\par
Bei Verstößen gegen das Benachteiligungsverbot muss der Arbeitgeber gemäß \agG[Abs. 3]{12} geeignete Maßnahmen zur Unterbindung ergreifen (z.B. Abmahnung, Versetzung oder Kündigung).\par
Auch wenn die Benachteiligung bei der Ausübung der Tätigkeit \textbf{durch Dritte} (z.B. Kunden) erfolgt, muss der Arbeitgeber gemäß \agG[Abs. 4]{12} geeignete Maßnahmen ergreifen.

\subsubsection{Leistungsverweigerungsrecht}
Falls der Arbeitgeber keine oder offensichtlich ungeeignete Maßnahmen zur Unterbindung einer (sexuellen) Belästigung am Arbeitsplatz, sind die Betroffenen gemäß \agG{14} berechtigt, ihre Tätigkeit einzustellen.\par
Sie verlieren dabei nicht den Anspruch auf das Arbeitsentgelt.

\subsubsection{Entschädigung und Schadensersatz}
Wenn der Arbeitgeber das Verstoß gegen das Benachteiligungsverbot zu vertreten hat, ist er gemäß \agG[Abs. 1]{15} zu Schadensersatz verpflichtet (selten, da meist kein finanzieller Schaden vorliegt).\par
Gemäß \agG[Abs. 2]{15} entsteht Anspruch auf Schadensersatz auch für Schäden, die nicht Vermögensschäden sind.\par
\textbf{Frist}: Gemäß \agG[Abs. 4]{15} muss ein Anspruch innerhalb von \textbf{zwei Monaten} \textbf{schriftlich} geltend gemacht werden. Die Frist beginnt bei einer Bewerbung mit dem Zugang der Ablehnung und sonst zu dem Zeitpunkt, in dem der Arbeitnehmer von der Benachteiligung Kenntnis erlangt.

\subsubsection{Beweislast}
Gemäß \agG{22} wird die \textbf{Beweislast umgekehrt}, wenn der Arbeitnehmer Indizien für eine Benachteiligung ``beweist''.

\section{Arbeitsvertrag}
\subsection{Form}
Grundsätzlich ist ein Arbeitsvertrag \highlight{formfrei}. Aufgrund des hohen Missbrauchspotentials existiert jedoch ein Nachweisgesetz, das den Arbeitgeber zur Verschriftlichung wichtiger Vertragsbedingungen verpflichtet.

\subsubsection{Nachweisgesetz}
Gemäß \nachwG{2} muss der Arbeitgeber dem Arbeitnehmer die \textbf{wesentlichen Vertragsbedingungen} unterschreiben und aushändigen, dazu gehören:
\begin{itemize}
  \item{Namen und Anschrift der Vertragsparteien}
  \item{Beginn (und ggf. Ende) des Arbeitsverhältnisses}
  \item{Beschreibung der vom Arbeitnehmer zu leistenden Tätigkeit}
  \item{Dauer der Probezeit}
  \item{Zusammensetzung und Höhe des Entgelts (inklusive Regelungen für Überstunden, Prämien, etc.)}
  \item{Dauer des Urlaubs}
\end{itemize}
Der Nachweis muss \textbf{schriftlich} spätestens \textbf{einen Monat} nach vereinbartem Beginn des Arbeitsverhältnisses erfolgen.

\subsection{Was regelt der Arbeitsvertrag?}
Im Arbeitsvertrag sind die Vertragsbedingungen des Arbeitsverhältnisses geregelt, dazu gehören:
\begin{itemize}
  \item{Entgelt (und Extras wie Boni, Firmenwagen etc.)}
  \item{Urlaub}
  \item{Probezeit}
  \item{Arbeitszeit (und Überstunden)}
  \item{Arbeitesplatzbeschreibung (Ort und Tätigkeit)}
  \item{Arbeitgeber}
  \item{Regelungen zum Home-Office}
  \item{Betriebliche Extras}
  \item{Datenschutzpflichten, Geheimhaltungsklauseln, Urheberrechte}
  \item{Regeln im Krankheitsfall/Verhinderung (Mutterschutz, Elternzeit)}
\end{itemize}

\subsubsection{Synallagma im Arbeitsvertrag}\label{sec:arbeitsvertragSynallagma}
Die \textbf{Tätigkeitsbeschreibung} und das \textbf{Entgelt} prägen den Arbeitsvertrag und stellen ein \highlight{Synallagma} (Vertrag mit gegenseitigen Verpflichtungen) dar.\par
Der Arbeitnehmer ist zur Arbeitsleistung verpflichtet, der Arbeitgeber zur Entgeltzahlung. Leistet eine Partei nicht, ist die andere Partei ebenfalls nicht zur Leistung verpflichtet.

\subsubsection{Boni}
\textbf{Boni} gehören ebenfalls zum Entgelt. Diese können entweder \textbf{individuell} (Bonus für abgeschlossene Verträge) oder z.B. \textbf{abteilungsweise} (bei Erhöhung des Umsatzes) vereinbart werden.

\subsubsection{Urlaubs- und Weihnachtsgeld}
\textbf{Urlaubs- und Weihnachtsgeld} sind nicht zwingend Teil des Entgelts (und somit des \hyperref[sec:arbeitsvertragSynallagma]{Synallagmas}).\par
Im Streitfall kommt es auf die Auslegung des Arbeitsvertrags an, ob z.B. das Urlaubs- und Weihnachtsgeld noch anteilig für das laufende Jahr ausbezahlt werden muss. Falls es im Arbeitsvertrag im Kapitel ``Vergütung'' steht, deutet die \textit{systematische Auslegung} auf eine Einbeziehung ins Synallagma hin.

\section{Teilzeit und befristete Arbeitsverträge}
Teilzeit und befristete Arbeitsverträge sind im \textbf{Gesetz über Teilzeitarbeit und befristete Arbeitsverträge} (TzBfG) geregelt.

\subsection{Befristete Arbeitsverträge}
\subsubsection{Zulässigkeit}\label{sec:befristeteArbeitsvertraegeZulaessigkeit}
Die Befristung ist gemäß \tzbfG[Abs. 1]{14} zulässig, wenn sie durch einen sachlichen Grund gerechtfertigt ist, darunter fallen:
\begin{enumerate}
  \item{nur vorübergehender betrieblicher Bedarf an der Arbeitsleistung}
  \item{im Anschluss an Ausbildung oder Studium}
  \item{Vertretung}
  \item{Eigenart der Arbeit (z.B. Skilehrer, Bademeister, usw.)}
  \item{Befristung zur Erprobung (\textbf{keine Probezeit!})}
  \item{Wunsch des Arbeitnehmers}
  \item{bei Vergütung aus befristeten Haushaltsmitteln (z.B. durch den Staat)}
  \item{gerichtlicher Vergleich}
\end{enumerate}

\subsubsection{Dauer}
Gemäß \tzbfG[Abs. 2]{14} ist die Befristung \textbf{ohne Vorliegen} eines \hyperref[sec:befristeteArbeitsvertraegeZulaessigkeit]{sachlichen Grundes} ist möglich, wenn die Befristung \textbf{maximal dreimal verlängert} wird und die Gesamtdauer \textbf{zwei Jahre} nicht überschreitet. Wenn zuvor ein Arbeitsverhältnis bestanden hat, ist die Befristung nicht zulässig (\tzbfG[Abs. 2 Satz 2]{14}).\par
Gemäß \tzbfG[Abs. 2a]{14} ist bei der Neugründung eines Unternehmens die Befristung innerhalb der \textbf{ersten vier Jahre} zulässig.\par
Gemäß \tzbfG[Abs. 3]{14} ist die Befristung \textbf{ohne Vorliegen} eines \hyperref[sec:befristeteArbeitsvertraegeZulaessigkeit]{sachlichen Grundes} auch zulässig, wenn der Arbeitnehmer \textbf{älter als 52} ist und unmittelbar vor Beginn des Arbeitsverhältnisses \textbf{mindestens vier Monat arbeitslos} war.

\subsubsection{Form}
Gemäß \tzbfG[Abs. 4]{14} muss die Befristung \textbf{schriftlich} erfolgen, um wirksam zu sein.

\subsubsection{Folgen unwirksamer Befristung}
Ist die Befristung unwirksam, dann gilt gemäß \tzbfG{16} der Vertrag als auf unbestimmte Zeit geschlossen. Die ordentliche Kündigung ist dann i. d. R. frühstens zum vereinbarten Ende möglich.

\subsubsection{Fristen für Klage bei unwirksamer Befristung}
Der Arbeitnehmer muss gemäß \tzbfG{17} innerhalb von \textbf{drei Wochen nach vereinbartem Ende} des befristeten Arbeitsvertrags Klage auf Feststellung erheben, dass das Arbeitsverhältnis auf Grund der Befristung nicht beendet ist.

\subsubsection{Ende des befristeten Arbeitsvertrages}
Ein kalendermäßig befristeter Arbeitsvertrag endet gemäß \tzbfG[Abs. 1]{15} mit Ablauf der vereinbarten Zeit.\par
Ein zweckbefristeter Arbeitsvertrag endet gemäß \tzbfG[Abs. 2]{15} mit Erreichen des Zwecks, der Arbeitnehmer muss jedoch \textbf{mindestens zwei Wochen} vorher \textbf{schriftlich} informiert werden.

\subsubsection{Information über unbefristete Arbeitsplätze}
Der Arbeitgeber muss gemäß \tzbfG[Abs. 1]{18} unbefristet beschäftigte Arbeitnehmer über unbefristete Arbeitsplätze zu informieren.

\subsection{Teilzeit}
\subsubsection{Förderung von Teilzeitarbeit}
Gemäß \tzbfG{6} muss der Arbeitgeber i. d. R. Teilzeitarbeit ermöglichen.

\subsubsection{Ablauf Wechsel von Vollzeit zu Teilzeit}\label{sec:wechselVollzeitTeilzeit}
Der Arbeitgeber muss gemäß \tzbfG[Abs. 3]{7} \textbf{auf Wunsch eines Arbeitnehmers} (der länger als sechs Monate beschäftigt ist) \textbf{einen Monat nach Zugang} der (in Textform) gestellten Anfrage über die Veränderung der Arbeitszeit eine \textbf{begründete Antwort} in Textform mitteilen.\par
Gemäß \tzbfG[Abs. 4]{8} muss der Arbeitgeber der Anfrage zustimmen, soweit keine betrieblichen Gründe entgegenstehen (z.B. Sicherheitsbedenken, Organisationsprobleme).\par
Der Arbeitgeber muss dem Arbeitnehmer die Entscheidung \textbf{spätestens einen Monat} vor gewünschtem Beginn der Verringerung mitteilen, \textbf{ansonsten verringert sich die Arbeitszeit wie vom Arbeitnehmer gefordert} (\tzbfG[Abs. 5 Satz 2]{8}).

\subsubsection{Erneute Anfrage zur Verringerung der Arbeitszeit}
Gemäß \tzbfG[Abs. 6]{8} kann der Arbeitnehmer frühstens zwei Jahre nachdem der Arbeitgeber einer Verringerung zugestimmt oder sie begründet abgelehnt hat, erneut eine Verringerung der Arbeitszeit verlangen.

\subsubsection{Ablauf Wechsel von Teilzeit zu Vollzeit}
Gemäß \tzbfG{9} muss der Arbeitgeber Teilzeitbeschäftigte \textbf{bevorzugt berücksichtigen}, wenn sie den Wunsch nach Verlängerung der Arbeitszeit angezeigt haben (mit Ausnahmen).

\subsubsection{Zeitlich begrenzte Verringerung der Arbeitszeit}
Gemäß \tzbfG[Abs. 1]{9a} kann der Arbeitnehmer analog zum \hyperref[sec:wechselVollzeitTeilzeit]{Wechsel von Vollzeit zu Teilzeit} eine zeitlich begrenzte Verringerung der Arbeitszeit verlangen.\par
Der Zeitraum muss dabei \textbf{mindestens ein Jahr} und \textbf{höchstens fünf Jahre} betragen. Der Anspruch besteht nur bei Unternehmen mit \textbf{mehr als 45 Arbeitnehmern}.\par
\tzbfG[Abs. 2]{9a} definiert Ausnahmen, z.B. wenn bereits mehrere Arbeitnehmer von dieser Regel Gebrauch gemacht haben.

\subsubsection{Kündigungsverbot bei Weigerung der Arbeitszeitanpassung}
Gemäß \tzbfG{11} ist eine \textbf{Kündigung unwirksam}, wenn sie wegen der Weigerung des Arbeitnehmers von Vollzeit in Teilzeit (oder umgekehrt) zu wechseln erfolgt ist.

\section{Beendigung des Arbeitsverhältnisses}
Ein Arbeitsverhältnis kann auf verschiedene Arten beendet werden, darunter Fristablauf (bei befristeten Arbeitsverträgen), Tod des Arbeitnehmers, Gerichtsurteil (z.B. nach Kündigungsschutzklage), Vertrag, gerichtlicher Vergleich oder Kündigung durch Arbeitgeber oder Arbeitnehmer.

\subsection{Welche Schutzvorschriften gibt es?}
Bei Kündigungen kommen mehrere Schutzvorschriften in Frage:
\begin{itemize}
  \item{Allgemeiner Kündigungsschutz (BGB)}
  \item{Kündigungsschutzgesetz (KSchG)}
  \item{sonstige Regeln zum Kündigungsschutz (z.B. für Schwangere)}
\end{itemize}

\subsection{Allgemeiner Kündigungsschutz}
Der allgemeine Kündigungsschutz ist im BGB festgehalten.

\subsubsection{Form}
Gemäß \bgb{623} muss die Kündigung \textbf{schriftlich} erfolgen.

\subsubsection{Fristen}
Gemäß \bgb[Abs. 1]{622} kann mit einer Frist von vier Wochen zum Fünfzehnten oder zum Ende eines Monats gekündigt werden.\par
Bei Kündigung durch den \textbf{Arbeitgeber} verlängert sich die Kündigungsfrist gemäß \bgb[Abs. 2]{622} je nach Dauer der Betriebszugehörigkeit.\par
In der Probezeit beträgt die Kündigungsfrist gemäß \bgb[Abs. 3]{622} zwei Wochen.\par
In Tarifverträgen (\bgb[Abs. 4]{622}) und in bestimmten Fällen (\bgb[Abs. 5]{622}) kann von den genannten Fristen abgewichen werden.

\subsubsection{Fristlose Kündigung}
Bei vorliegen eines \textbf{wichtigen Grundes}, unter dem die Fortsetzung des Arbeitsverhältnisses unzumutbar ist, kann gemäß \bgb[Abs. 1]{626} fristlos gekündigt werden.\par
Gemäß \bgb[Abs. 2]{626} kann die Kündigung nur \textbf{innerhalb von zwei Wochen nach Kenntnis} über die für die Kündigung maßgebende Tatsachen erfolgen.

\subsection{Kündigungsschutzgesetz}
\subsubsection{Anwendungsbereich}
Gemäß \kschG[Abs. 1]{1} gilt das Kündigungsschutzgesetz nur für Arbeitnehmer, dessen Arbeitsverhältnis in dem Betrieb \textbf{länger als sechs Monate} bestanden hat.\par
\kschG{23} definiert weitere Ausnahmen, so sind z.B. Betriebe mit \textbf{weniger als zehn Arbeitnehmern} (Auszubildende werden nicht mitgezählt) von den meisten Regelungen ausgenommen.

\subsubsection{Ausnahmen für leitende Angestellte}\label{sec:kuendigungsschutzLeitendeAngestellte}
Gemäß \kschG{14} gelten die Schutzvorschriften nicht für \textbf{leitende Angestellte}, soweit diese selbständig \textbf{einstellen \underline{und} entlassen} können.

\subsubsection{Zulässige Kündigungsgründe}
Gemäß \kschG[Abs. 2]{1} ist eine Kündigung \highlight{sozial ungerechtfertigt}, wenn der Grund nicht in einem der drei folgenden liegt:
\begin{itemize}
  \item{in dem Verhalten des Arbeitnehmers (\highlight{verhaltensbedingt})}
  \item{in der Person des Arbeitnehmers (\highlight{personenbedingt})}
  \item{durch dringende betriebliche Erfordernisse (\highlight{betriebsbedingt})}
\end{itemize}

\subsubsection{Verhaltensbedingte Kündigung}
Bei einer \highlight{verhaltensbedingte Kündigung} liegt der Grund in dem Verhalten des Arbeitnehmers, das es dem Arbeitgeber unzumutbar macht, das Arbeitsverhältnis fortzusetzen.\par
Zu den Gründen gehören beispielsweise:
\begin{itemize}
  \item{Straftaten gegen Arbeitnehmer oder Kollegen}
  \item{Schwerwiegende Verstöße gegen den Arbeitsvertrag (z.B. wiederholte Unpünktlichkeit nach Abmahnung)}
\end{itemize}

\subsubsection{Personenbedingte Kündigung}
Bei einer \highlight{personenbedingten Kündigung} liegt der Grund in der Person des Arbeitnehmers.\par
Zu den Gründen gehören beispielsweise:
\begin{itemize}
  \item{Langzeit-Krankheit}
  \item{Personenbedingte Schlechtleistung (Arbeitnehmer arbeitet nur noch langsam/schlecht)}
  \item{Verdacht von grobem Fehlverhalten (z.B. Diebstahl, Unregelmäßigkeiten in Büchern)}
\end{itemize}

\subsubsection{Betriebsbedingte Kündigung}\label{sec:betriebsbedingteKuendigung}
Eine \highlight{betriebsbedingte Kündigung} kann entweder wegen \textbf{äußerer betrieblicher Gründe} oder \textbf{innerbetrieblicher Gründe} erfolgen.\par
Beispiele für äußere betrieblicher Gründe:
\begin{itemize}
  \item{Schlechte Wirtschaftslage}
  \item{Teure Rohstoffe}
  \item{Rückgang der Aufträge}
\end{itemize}
Beispiele für innerbetrieblicher Gründe:
\begin{itemize}
  \item{Umstrukturierung des Betriebs}
  \item{Digitalisierung von Prozessen}
  \item{Outsourcing von Tätigkeiten}
\end{itemize}

\subsubsection{Sozialauswahl bei betriebsbedingter Kündigung}
Bei einer \hyperref[sec:betriebsbedingteKuendigung]{betriebsbedingten Kündigung} kann der Arbeitgeber nicht beliebige Arbeitnehmer kündigen, sondern muss gemäß \kschG[Abs. 3]{1} eine \highlight{Sozialauswahl} treffen.\par
Dabei wird den \textbf{vier Kriterien} Alter, Betriebszugehörigkeit, Unterhaltspflichten und Behinderung eine Gewichtung (Punkte) zugewiesen und alle in Frage kommenden Arbeitnehmer nach dieser Gewichtung bewertet. Eine Entlassung ist nur bei den Arbeitnehmern mit den wenigsten Punkten zulässig.\par
In einem Tarifvertrag oder durch den Betriebsrat kann gemäß \kschG[Abs. 4]{1} diese Gewichtung festgelegt werden, die gerichtlich nur auf grobe Fehlerhaftigkeit überprüft werden kann und ansonsten gilt.\par
Falls der Arbeitgeber die Gewichtung alleine vornimmt, kann diese vor Gericht überprüft und für unwirksam erklärt werden.

\subsubsection{Ausnahmen von der Sozialauswahl}
Gemäß \kschG[Abs. 3 Satz 2]{1} sind bestimmte Arbeitnehmer von der Sozialauswahl ausgenommen, z.B. aufgrund ihrer Erfahrung, Weiterbildung oder der Datenschutzbeauftragte.

\subsubsection{Folgen einer Kündigung}
Der Arbeitnehmer muss gemäß \kschG{4} \textbf{drei Wochen} nach Zugang der \textbf{schriftlichen} Kündigung Klage auf Feststellung erheben, dass das Arbeitsverhältnis durch die Kündigung nicht aufgelöst ist.\par
Ansonsten gilt die Kündigung gemäß \kschG{7} (mit Ausnahmen in \kschG{5} und \kschG{6}) als \textbf{von Anfang an wirksam}.

\subsection{Sonstige Kündigungsschutzregeln}
\subsubsection{Kündigungsverbot von Schwangeren und Müttern}
Gemäß \muschG{17} ist eine Kündigung von Schwangeren und Müttern für eine gewisse Zeit nach der Geburt des Kindes unzulässig.

\subsubsection{Kündigungsschutz während der Elternzeit}
Gemäß \beeG{18} können Eltern in Elternzeit (und kurz davor) nicht gekündigt werden.

\subsubsection{Kündigungsverbot für Datenschutzbeauftragten}
Gemäß \bdsG[Abs. 2]{38} können Datenschutzbeauftragte in nichtöffentlicher Stellen (Unternehmen) nicht gekündigt werden, wenn die Benennung eines Datenschutzbeauftragten verpflichtend ist.

\subsubsection{Unzulässigkeit der Kündigung des Betriebsrats und Vertretungen}
Gemäß \kschG{15} ist die Kündigung eines Mitglieds eines Betriebsrats, einer Jugend- und Auszubildendenvertretung usw. unzulässig (außer bei außerordentlichen Kündigungen).

\subsubsection{Erfordernis der Zustimmung bei Kündigung von Schwerbehinderten}
Gemäß \sgbIX{168} ist bei der Kündigung eines schwerbehinderten Arbeitnehmers die Zustimmung des Integrationsamts erforderlich.

\subsubsection{Anzeigepflicht bei Massenentlassungen}
Gemäß \kschG[Abs. 1]{17} muss der Arbeitgeber der Agentur für Arbeit mitteilen, wenn er eine größere Anzahl von Arbeitnehmern entlassen will (Staffelungen im Gesetz).

\subsubsection{Mitbestimmung des Betriebsrats bei Kündigungen}
Gemäß \betrvG{102} hat der Betriebsrat ein Mitbestimmungsrecht bei Kündigungen.\par
Eine ohne Anhörung des Betriebsrats ausgesprochene Kündigung ist gemäß \betrvG[Abs. 1]{102} unwirksam.\par
Widerspricht der Betriebsrat der Kündigung, kann der Arbeitgeber trotzdem kündigen, muss aber gemäß \betrvG[Abs. 4]{102} die Stellungnahme des Betriebsrats beilegen.

\section{Urlaub}
\subsection{Dauer des Urlaubs}
Gemäß \burlG[Abs. 1]{3} beträgt der gesetzliche Mindesturlaub \textbf{24 Werktage} pro Jahr, jedoch bei einer Sechs-Tage-Woche.\par
Bei einer Fünf-Tage-Woche beträgt der gesetzliche Mindesturlaub \textbf{20 Werktage} pro Jahr.

\subsection{Wartezeit und Entstehen des Anspruchs}
Gemäß \burlG{4} besteht der volle Anspruch nach sechsmonatigem Bestehen des Arbeitsverhältnisses.\par
Während dieser Wartezeit und bei Ausscheiden vor Ablauf des Kalenderjahres besteht gemäß \burlG[Abs. 1]{5} ein Anspruch auf \textbf{ein Zwölftel des Jahresurlaubs pro Monat}.

\subsection{Ausschluss von Doppelansprüchen}
Beim Wechsel des Arbeitgebers besteht gemäß \burlG[Abs. 1]{6} nur Anspruch auf den noch nicht verbrauchten Urlaub des laufenden Jahres.

\subsection{Zeitpunkt, Übertragbarkeit und Abgeltung des Urlaubs}
Der Urlaub muss gemäß \burlG[Abs. 3]{7} im laufenden Kalenderjahr genommen werden. Eine Übertragung ins nächste Jahr ist nur bei dringenden Gründen möglich.\par

\section{Betriebsräte}
Die Aufgabe und Rollen von \highlight{Betriebsräten} sind im Betriebsverfassungsgesetz (BetrVG) geregelt.

\subsection{Definitionen}
\subsubsection{Definition: Betrieb}
Der Begriff \highlight{Betrieb} ist im Betriebsverfassungsgesetz nicht definiert. In der Rechtsprechung wird ein Betrieb als \textbf{organisatorische Einheit} angesehen, innerhalb derer der Arbeitgeber mit den Arbeitnehmers bestimmte Zwecke verfolgt.\par
Ein Betrieb kann ein komplettes Unternehmen, einen Standort oder eine Abteilung umfassen. Betriebe können aber auch standort- und sogar unternehmensübergreifend existieren (\betrvG[Abs. 2]{1}).

\subsubsection{Definition: Arbeitnehmer}
Gemäß \betrvG[Abs. 1]{5} sind Arbeitnehmer im Sinne des Gesetzes Arbeiter und Angestellte (\textbf{auch Auszubildende!}), egal ob sie im Betrieb, im Außendienst oder im Home-Office arbeiten.\par
Beamte, Soldaten und Arbeitnehmer des öffentlichen Dienstes, die in privatrechtlich organisierten Unternehmen tätig sind, zählen ebenfalls dazu (z.B. Deutsche Post oder Bahn).\par

\subsubsection{Ausnahmen für leitende Angestellte}
Gemäß \betrvG[Abs. 3]{5} gelten die Vorschriften nicht für \textbf{leitende Angestellte}. Im Gegensatz zum \hyperref[sec:kuendigungsschutzLeitendeAngestellte]{Kündigungsschutzgesetz} sind leitende Angestellte im Sinne dieses Gesetzes solche, die selbständig \textbf{einstellen \underline{oder} entlassen} können.

\subsection{Wahl eines Betriebsrats}
\subsubsection{Recht zur Errichtung von Betriebsräten}
Gemäß \betrvG[Abs. 1]{1} ist die Errichtung eines Betriebsrats in Betrieben mit \textbf{mindestens fünf ständigen wahlberechtigten Arbeitnehmern} möglich.\par
Einen Zwang zur Wahl eines Betriebsrats existiert jedoch nicht, da die Vereinigungsfreiheit (\GG{9}) auch negativ gilt, es gibt also auch das Recht, sich nicht zu vereinigen.

\subsubsection{Wahlschutz}
Der Arbeitgeber darf gemäß \betrvG[Abs. 1]{20} die Wahl des Betriebsrats \textbf{nicht behindern} und gemäß \betrvG[Abs. 2]{20} auch \textbf{nicht durch Nachteile oder Vorteile beeinflussen}.

\subsubsection{Wahlberechtigung}
Gemäß \betrvG{7} sind alle \textbf{Arbeitnehmer} (also \textbf{keine leitenden Angestellten!}) ab 16 Jahren wahlberechtigt. Auch Leiharbeiter sind wahlberechtigt, wenn sie länger als drei Monate im Betrieb tätig sind.

\subsubsection{Wählbarkeit}
Wählbar sind gemäß \betrvG[Abs. 1]{8} alle Wahlberechtigten, die über 18 Jahre alt sind und mindestens sechs Monate im Betrieb tätig sind.\par
Besteht der Betrieb seit weniger als sechs Monaten, so gibt es gemäß \betrvG[Abs. 2]{8} keine Voraussetzungen bei der Dauer der Betriebszugehörigkeit.

\subsubsection{Zahl der Betriebsratsmitglieder}
Die Zahl der Betriebsratsmitglieder ist gemäß \betrvG{9} nach der Zahl der Arbeitnehmer im Betrieb gestaffelt.

\subsubsection{Wahlperiode}
Gemäß \betrvG[Abs. 1]{13} finden regelmäßige Wahlen \textbf{alle vier Jahre} zwischen dem 1. März und 31. Mai statt.\par
In Ausnahmesituationen, z.B. großen Veränderungen in der Zahl der Arbeitnehmern, bei Rücktritt der Mitglieder und Anfechtung der Wahl kann der Betriebsrat gemäß \betrvG[Abs. 2]{13} auch außerhalb dieser Zeit gewählt werden.

\subsubsection{Freistellungen von Betriebsräten}
Gemäß \betrvG{38} hat eine gewisse Anzahl von Betriebsratsmitgliedern (gestaffelt nach Zahl der Arbeitnehmer) Anspruch auf Freistellung von der Arbeit, um ihre Aufgaben als Betriebsrat wahrzunehmen.

\subsubsection{Vereinfachtes Wahlverfahren für Kleinbetriebe}
Gemäß \betrvG{14a} wird in kleineren Betrieben ein einfacheres Wahlverfahren angewendet.

\subsubsection{Zusammensetzung nach Geschlechter}
Gemäß \betrvG[Abs. 2]{15} muss das Geschlecht, das in der Belegschaft in der Minderheit ist, mindestens entsprechend dem Verhältnis im Betriebsrat vertreten sein (wenn dieser aus mindestens drei Mitgliedern besteht).\par
Hierbei zählt jeder Arbeitnehmer gleich viel, egal ob er in Vollzeit oder Teilzeit angestellt ist.

\subsection{Gesamtbetriebsräte}
\subsubsection{Voraussetzungen der Errichtung}
Wenn in einem Unternehmen mehrere Betriebsräte bestehen, wird gemäß \betrvG[Abs. 1]{47} ein Gesamtbetriebsrat errichtet.

\subsubsection{Zuständigkeit}
Der Gesamtbetriebsrat ist gemäß \betrvG[Abs. 1]{50} auch für Betriebe ohne eigenen Betriebsrat zuständig.

\subsection{Gewerkschaften}
Eine Gewerkschaft ist in einem Betrieb vertreten, sobald ein Arbeitnehmer Mitglied dieser Gewerkschaft ist.\par
Der Gewerkschaft muss gemäß \betrvG[Abs. 2]{2} Zugang zum Betrieb gewährt werden, falls nicht Betriebsgeheimnisse oder andere Vorschriften dem entgegenstehen.

\end{document}