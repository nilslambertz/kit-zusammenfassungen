\documentclass[12pt,A4]{extarticle}	
\usepackage[table]{xcolor}

\newcommand{\lectureTitle}{Arbeitsrecht [WIP]}
\newcommand{\semester}{Sommersemester 2024}

\usepackage[a4paper,left=0.9cm,right=1cm,top=1.37cm,bottom=2.5cm]{geometry}
\usepackage[utf8]{inputenc}
\usepackage{xifthen}
\usepackage{cmbright}
\usepackage{fontawesome}
\usepackage[T1]{fontenc}
\usepackage{lastpage,lipsum}
\usepackage{hyperref}
\usepackage{transparent}
\usepackage{color}
\usepackage{fancyhdr}

\renewcommand*\familydefault{\sfdefault}
\setlength{\parindent}{0mm}

\usepackage{transparent}
\usepackage{color}
\usepackage{fancyhdr}

\definecolor{headerBg}{RGB}{11, 67, 158}
\definecolor{headerGrayColor}{RGB}{210, 210, 210}

\pagestyle{fancy}
\fancyhead[C]{
  \fcolorbox{headerBg}{headerBg}{
    \hspace{0.6cm}\begin{minipage}[c][50pt][c]{\paperwidth}
      \begin{minipage}[c]{.45\textwidth}
        \huge{\textcolor{white}{Vorlesung}}\normalsize\\
        \textcolor{headerGrayColor}{\small{Wintersemester 2022/23}}
      \end{minipage}%
      \begin{minipage}[c]{.45\textwidth}
        \raggedleft
        \textcolor{white}{
          \small{\href{https://nilslambertz.de/}{nilslambertz.de}}\\
          \href{https://github.com/nilslambertz/}{\textcolor{white}{\faicon{github}} \small{nilslambertz}}}
      \end{minipage}
    \end{minipage}}
}
\renewcommand{\headrulewidth}{0pt}
\setlength{\headheight}{40pt}

\newlength{\oddmarginwidth}
\setlength{\oddmarginwidth}{1in+\hoffset+\oddsidemargin}
\newlength{\evenmarginwidth}
\setlength{\evenmarginwidth}{\evensidemargin+1in}
\fancyhfoffset[LO,RE]{\oddmarginwidth}
\fancyhfoffset[LE,RO]{\evenmarginwidth}
\cfoot{\thepage\ $/$ \pageref*{LastPage}}

\definecolor{highlightColor}{RGB}{66, 135, 245}
\newcommand{\highlight}[1]{\textcolor{highlightColor}{\textbf{#1}}}
\definecolor{gesetzLink}{RGB}{194, 74, 14}
\newcommand{\bgb}[2][]{\textbf{\textcolor{gesetzLink}{\href{https://www.gesetze-im-internet.de/bgb/__#2.html}{§ #2 \ifthenelse{\equal{#1}{}}{}{#1 }BGB}}}}
\newcommand{\agG}[2][]{\textbf{\textcolor{gesetzLink}{\href{https://www.gesetze-im-internet.de/agg/__#2.html}{§ #2 \ifthenelse{\equal{#1}{}}{}{#1 }AGG}}}}
\newcommand{\agGG}[2][]{\textbf{\textcolor{gesetzLink}{\href{https://www.gesetze-im-internet.de/agg/__#2.html}{§§ #1 AGG}}}}

\def\contentsname{\empty}

\begin{document}

\disclaimer

\tableofcontents
\clearpage

\section{Grundlagen}
\subsection{Arbeitsvertrag vs. Dienstvertrag}
Bei \textbf{Arbeitsverträgen} (oder Dienstverträgen) steht der \textbf{Dienst} im Vordergrund. Der Arbeiternehmer muss den Dienst leisten, wird jedoch bezahlt, auch wenn der gewünschte Erfolg nicht eintritt.\par
Bei \textbf{Werkverträgen} steht der \textbf{Erfolg} im Vordergrund. Wenn das ``Werk'' nicht erbracht wird, muss der Auftragnehmer nicht bezahlt werden (z.B. Handwerker).

\subsection{Pflichten vor Beginn des Arbeitsvertrags}
Auch schon während der Bewerbungsphase existieren vorvertragliche Pflichten, auch wenn noch keine Verträge geschlossen wurden. \bgb[Abs. 2]{311} definiert gegenseitige Rücksichts- und Geheimhaltungspflichten.

\section{Allgemeines Gleichbehandlungsgesetz (Antidiskriminierungsgesetz)}
\subsection{Grundlagen}
\subsubsection{Ziel}\label{sec:aggZiel}
Das Ziel des Allgemeines Gleichbehandlungsgesetzes (auch \highlight{Antidiskriminierungsgesetz} genannt) ist in \agG{1} definiert. Es soll Benachteiligungen aufgrund bestimmter (aufgezählter) Merkmale verhindern.

\subsubsection{Benachteiligungsverbot}
Das \textbf{Benachteiligungsverbot} ist in \agG{7} definiert. Demnach dürften Beschäftigte nicht wegen der \hyperref[sec:aggZiel]{genannten Merkmale} benachteiligt werden.\par
Dies gilt auch, wenn das Merkmal gar nicht auf die Person zutrifft, sondern nur angenommen wird.

\subsubsection{Ausschreibungen}
Gemäß \agG{11} dürfen Arbeitsplätze nicht unter Verstoß gegen \agG[Abs. 1]{7} ausgeschrieben werden.

\subsection{Arten der Benachteiligung}
Die Arten der Benachteiligung sind in \agG{3} definiert.

\subsubsection{Unmittelbare Benachteiligung}
Eine \highlight{unmittelbare Benachteiligung} liegt gemäß \agG[Abs. 1]{3} vor, wenn eine Person aufgrund der \hyperref[sec:aggZiel]{genannten Merkmale} eine weniger günstige Behandlung erfährt als andere Personen.\par
Dies gilt ebenfalls für Frauen, die aufgrund einer Schwangerschaft oder Mutterschaft benachteiligt werden.

\subsubsection{Mittelbare Benachteiligung}
Eine \highlight{mittelbare Benachteiligung} liegt gemäß \agG[Abs. 2]{3} vor, wenn dem Anschein nach neutrale Vorschriften oder Verfahren Personen wegen der \hyperref[sec:aggZiel]{genannten Merkmale} benachteiligen, wenn sie nicht sachlich gerechtfertigt, angemessen und erforderlich sind.\par
Hierbei erfolgt keine direkte Diskriminierung wegen der Merkmale, sondern über Faktoren, die überproportional von dieser Gruppe erfüllt werden.\par
Ein Beispiel ist die Verweigerung von Teilzeitstellen (benachteiligt Frauen überproportional).

\subsubsection{Belästigung}
Eine \highlight{Belästigung} ist gemäß \agG[Abs. 3]{3} eine \textbf{unerwünschte Verhaltensweise}, die mit den \hyperref[sec:aggZiel]{genannten Merkmale} in Verbindung steht, die Würde der Person verletzt und das Arbeitsumfeld negativ beeinflussen.\par

\subsubsection{Sexuelle Belästigung}
Eine \highlight{sexuelle Belästigung} (\agG[Abs. 4]{3}) ist analog für sexuelle Verhaltensweisen definiert.\par
Darunter fallen sexuelle Handlungen, Aufforderungen, körperliche Berührungen, Bemerkungen sowie unerwünschtes Zeigen von pornographischen Darstellungen.

\subsection{Zulässige unterschiedliche Behandlung}
In \agGG[8-10]{8} sind Ausnahmen definiert, in denen unterschiedliche Behandlungen zulässig sind.

\subsubsection{Berufliche Anforderungen}
Wenn die Art der Tätigkeit oder die Bedingungen eine wesentliche und entscheidende berufliche Anforderung darstellt, ist gemäß \agG{8} eine unterschiedliche Behandlung zulässig.

\subsubsection{Religion oder Weltanschauung}
Für Religionsgemeinschaften und ihnen zugeordneten Einrichtungen ist gemäß \agG{9} eine Benachteiligung zulässig, wenn die Religion oder Weltanschauung eine gerechtfertigte berufliche Anforderung darstellt.\par
Eine Religionsgemeinschaft kann z.B. fordern, dass ein Pastor ihrer Religion angehört.

\subsubsection{Alter}
Gemäß \agG{10} ist eine unterschiedliche Behandlung wegen des Alters zulässig, wenn sie objektiv angemessen und durch ein legitimes Ziel gerechtfertigt ist.\par
Das können z.B. Mindest- oder Höchstalter für bestimmte Tätigkeiten sein.

\subsubsection{Unterschiedliche Behandlung wegen mehrerer Gründe}
Wenn die unterschiedliche Behandlung wegen mehrerer Gründe erfolgt, kann diese gemäß \agG{4} nur durch \agGG[8 bis 10 und 20]{8} nur gerechtfertigt sein, wenn die Rechtfertigung sich auf \textbf{alle diese Gründe erstreckt}.

\subsection{Folgen einer Benachteiligung}
\subsubsection{Pflichten des Arbeitgebers}
Gemäß \agG{12} verpflichtet, (auch vorbeugende) Maßnahmen zum Schutz vor Benachteiligungen zu treffen. Während der Aus- und Fortbildung muss er gemäß \agG[Abs. 2]{12} auf die Unzulässigkeit solcher Benachteiligungen hinweisen.\par
Bei Verstößen gegen das Benachteiligungsverbot muss der Arbeitgeber gemäß \agG[Abs. 3]{12} geeignete Maßnahmen zur Unterbindung ergreifen (z.B. Abmahnung, Versetzung oder Kündigung).\par
Auch wenn die Benachteiligung bei der Ausübung der Tätigkeit \textbf{durch Dritte} (z.B. Kunden) erfolgt, muss der Arbeitgeber gemäß \agG[Abs. 4]{12} geeignete Maßnahmen ergreifen.

\subsubsection{Leistungsverweigerungsrecht}
Falls der Arbeitgeber keine oder offensichtlich ungeeignete Maßnahmen zur Unterbindung einer (sexuellen) Belästigung am Arbeitsplatz, sind die Betroffenen gemäß \agG{14} berechtigt, ihre Tätigkeit einzustellen.\par
Sie verlieren dabei nicht den Anspruch auf das Arbeitsentgelt.

\subsubsection{Entschädigung und Schadensersatz}
Wenn der Arbeitgeber das Verstoß gegen das Benachteiligungsverbot zu vertreten hat, ist er gemäß \agG[Abs. 1]{15} zu Schadensersatz verpflichtet (selten, da meist kein finanzieller Schaden vorliegt).\par
Gemäß \agG[Abs. 2]{15} entsteht Anspruch auf Schadensersatz auch für Schäden, die nicht Vermögensschäden sind.\par
\textbf{Frist}: Gemäß \agG[Abs. 4]{15} muss ein Anspruch innerhalb von \textbf{zwei Monaten} \textbf{schriftlich} geltend gemacht werden. Die Frist beginnt bei einer Bewerbung mit dem Zugang der Ablehnung und sonst zu dem Zeitpunkt, in dem der Arbeitnehmer von der Benachteiligung Kenntnis erlangt.

\subsubsection{Beweislast}
Gemäß \agG{22} wird die \textbf{Beweislast umgekehrt}, wenn der Arbeitnehmer Indizien für eine Benachteiligung ``beweist''.

\end{document}