\documentclass[12pt,A4]{extarticle}	

\usepackage{amsfonts}
\usepackage{amsmath}
\usepackage{amssymb}
\usepackage{graphicx,wrapfig,lipsum}
\usepackage[german]{babel}
\usepackage{tikz}
\usetikzlibrary{calc, decorations.text}
\usetikzlibrary{decorations.pathreplacing,calligraphy}

% For binary trees
\usepackage{forest}
\usepackage{adjustbox}

\newcommand{\lectureTitle}{Authentisierung und Verschlüsselung [WIP]}
\newcommand{\semester}{Sommersemester 2023}

\newcommand{\titleSize}{\LARGE}

\usepackage[a4paper,left=0.9cm,right=1cm,top=1.37cm,bottom=2.5cm]{geometry}
\usepackage[utf8]{inputenc}
\usepackage{xifthen}
\usepackage{cmbright}
\usepackage{fontawesome}
\usepackage[T1]{fontenc}
\usepackage{lastpage,lipsum}
\usepackage{hyperref}
\usepackage{transparent}
\usepackage{color}
\usepackage{fancyhdr}

\renewcommand*\familydefault{\sfdefault}
\setlength{\parindent}{0mm}

\definecolor{headerBg}{RGB}{11, 67, 158}
\definecolor{headerGrayColor}{RGB}{210, 210, 210}

\newcommand{\printTitle}{\textcolor{white}{\lectureTitle}\normalsize}
\newcommand{\printSubtitle}{
  \ifdefined\lectureSubtitle
    \textcolor{white}{\small{\lectureSubtitle}}\\
  \fi
}

\fancyhf{}
\pagestyle{fancy}
\fancyhead[C]{
  \fcolorbox{headerBg}{headerBg}{
    \hspace{0.6cm}\begin{minipage}[c][50pt][c]{\paperwidth}
      \begin{minipage}[c]{.7\textwidth}
        \ifdefined\titleSize
          \titleSize \printTitle\\
        \else
          \huge\printTitle\\
        \fi
        \printSubtitle
        \textcolor{headerGrayColor}{\small{\semester}}
      \end{minipage}%
      \begin{minipage}[c]{.2\textwidth}
        \raggedleft
        \textcolor{white}{
          \small{\href{mailto:mail@nilslambertz.de}{\textcolor{white}{\faicon{envelope}} mail@nilslambertz.de}}\\
          \href{https://github.com/nilslambertz/kit-zusammenfassungen}{\textcolor{white}{\faicon{github}} \small{nilslambertz}}}
      \end{minipage}
    \end{minipage}}
}
\renewcommand{\headrulewidth}{0pt}
\setlength{\headheight}{40pt}

\newlength{\oddmarginwidth}
\setlength{\oddmarginwidth}{1in+\hoffset+\oddsidemargin}
\newlength{\evenmarginwidth}
\setlength{\evenmarginwidth}{\evensidemargin+1in}
\fancyhfoffset[LO,RE]{\oddmarginwidth}
\fancyhfoffset[LE,RO]{\evenmarginwidth}
\cfoot{\thepage\ $/$ \pageref*{LastPage}}

\definecolor{highlightColor}{RGB}{66, 135, 245}
\newcommand{\highlight}[1]{\textcolor{highlightColor}{\textbf{#1}}}

\definecolor{noticeColor}{RGB}{235, 110, 38}
\newcommand{\notice}[1]{\textcolor{noticeColor}{#1}}

\def\contentsname{\empty}
\addto\captionsgerman{
  \renewcommand{\contentsname}{\empty}
}

\begin{document}
\disclaimer

\tableofcontents
\clearpage

\section{Einführung}
\subsection{Ziel von Kryptographischen Verfahren}
Kryptographische Verfahren sollen \highlight{Authentizität} (Dokument wurde von einer bestimmten Person signiert) und \highlight{Integrität} (Dokument wurde nicht verändert) sicherstellen.

\subsection{Informelle Definition von Signaturen}
\begin{itemize}
  \item{\textbf{asymmetrische} Verfahren}
  \item{Schlüsselpaar $(pk, sk)$}
  \item{Nachricht $m$ wird mit $sk$ signiert und erzeugt Signatur $\sigma$}
  \item{Mit $pk$ kann überprüft werden, ob eine Signatur $\sigma$ gültig für eine Nachricht $m$ ist}
\end{itemize}

\subsection{Digitale Signaturen}
\subsubsection{Definition}
Ein digitales Signaturverfahren für einen Nachrichtenraum $\mathcal{M}$ ist ein Tupel $\Sigma = (Gen, Sign, Vfy)$ von probabilistischen Polyzeit (PPT) Algorithmen:
\begin{itemize}
  \item{$Gen(1^k) \rightarrow (pk, sk)$}
  \item{$Sign(sk, m) \rightarrow \sigma$, $m \in \mathcal{M}$}
  \item{$Vfy(pk, m, \sigma) \in \{0, 1\}$}
\end{itemize}

\subsubsection{Correctness}
\highlight{Correctness} (``Das Verfahren funktioniert''): $\forall (pk, sk) \leftarrow Gen(1^k) \forall m \in \mathcal{M}: Vfy(pk, m, Sign(sk, m)) = 1$

\subsection{Sicherheitsdefinitionen}
Sicherheit besteht aus einem \highlight{Angreifermodell} (was kann der Angreifer tun, welche Angriffsmöglichkeiten stehen zur Verfügung) und einem \highlight{Angreiferziel} (was muss der Angreifer tun, um das Verfahren zu brechen).

\subsubsection{Angreifermodelle}
\begin{enumerate}
  \item{no-message attack (NMA)
              \begin{itemize}
                \item{Angreifer erhält nur $pk$}
              \end{itemize}
        }
  \item{\highlight{non-adaptive chosen-message attack (naCMA)}
              \begin{itemize}
                \item{Angreifer wählt $m_1, \dots, m_q$}
                \item{Angreifer erhält \textbf{danach} $pk$ und Signaturen $\sigma_1, \dots, \sigma_q$}
              \end{itemize}
        }
  \item{\highlight{(adaptive) chosen-message attack (CMA)}
              \begin{itemize}
                \item{Angreifer erhält $pk$}
                \item{Angreifer wählt dann (adaptiv) $m_1, \dots, m_q$ und erhält Signaturen $\sigma_1, \dots, \sigma_q$}
                \item{Adaptiv: Angreifer darf Wahl von $m_i$ abhängig von vorherigen $\sigma_j$ ($j < i$) und $pk$ machen}
              \end{itemize}
        }
\end{enumerate}

\subsubsection{Angreiferziele}
\begin{enumerate}
  \item{Universal Unforgeability (UUF)
              \begin{itemize}
                \item{Nachricht $m$ wird zufällig gewählt}
                \item{Angreifer muss $m$ signieren}
              \end{itemize}
        }
  \item{\highlight{Existential Unforgeablility (EUF)}
              \begin{itemize}
                \item{Angreifer kann Nachricht $m$ beliebig wählen und diese signieren}
              \end{itemize}
        }
\end{enumerate}

In den \textbf{Sicherheitsdefinitionen} werden \textbf{Angreiferziel} und \textbf{Angreifermodell} kombiniert, z.B.
\begin{itemize}
  \item{EUF-CMA}
  \item{EUF-naCMA}
\end{itemize}

\subsection{EUF-CMA-Sicherheitsexperiment}
Bei Sicherheitsexperimenten spielt ein Angreifer $\mathcal{A}$ gegen einen Challenger $\mathcal{C}$. $\mathcal{A}$ gewinnt, falls er die Sicherheit des Verfahrens bricht.\par
$\mathcal{A}$ muss dabei mit einer nicht vernachlässigbaren Wahrscheinlichkeit eine gültige Signatur erzeugen können, ohne den Schlüssel $sk$ zu kennen.

\subsubsection{Visualisierung: EUF-CMA-Sicherheitsexperiment}
\begin{tikzpicture}
  \node (A) at (0,6) {$\mathcal{C}_\text{EUF-CMA}$};
  \node (B) at (5,6) {$\mathcal{A}$};
  \node[label={[align=center]below:$Vfy(pk, m^*, \sigma^*) = 1$?\\ $\land$ \\ $m^* \notin \{m_1, \dots, m_q\}$?}] (C) at (0,0) {};
  \node (D) at (5,0) {};

  \draw[dashed] (A) -- (C);
  \draw[dashed] (B) -- (D);

  \node[label={left:$(pk, sk) \leftarrow Gen(1^k)$}] at (0,5) {};
  \node[label={left:$\sigma_i \leftarrow Sign(sk, m_i)$}]  at (0,3) {};

  \draw[decoration={calligraphic brace,amplitude=10pt}, decorate, line width=1.25pt] (5.2,4) -- (5.2,2)
  node[midway, right=0pt, font=\footnotesize] {\begin{minipage}{5cm}\begin{itemize}
        \item{Anfragen nacheinander}
        \item{$q = q(k)$ Anfragen}
        \item{$q$ Polynom}
      \end{itemize}\end{minipage}};

  \draw[->,shorten >=5pt, shorten <=5pt] (0,5) -- (5,4.5) node[midway, above, sloped] {$pk$};
  \draw[->,shorten >=5pt, shorten <=5pt] (5,4) -- (0,3.5) node[midway, above, sloped] {$m_i$};

  \draw[->,shorten >=5pt, shorten <=5pt] (0,2.5) -- (5,2) node[midway, above, sloped] {$\sigma_i$};
  \draw[->,shorten >=5pt, shorten <=5pt] (5,1.5) -- (0,1) node[midway, above, sloped] {$m^*, \sigma^*$};
\end{tikzpicture}

$\mathcal{A}$ gewinnt, falls $Vfy(pk, m^*, \sigma^*) = 1$ \textbf{und} $m^* \notin \{m_1, \dots, m_q\}$

\subsubsection{Definition: Vernachlässigbarkeit}
Eine Funktion $negl: \mathbb{N} \rightarrow [0, 1]$ ist \textit{vernachlässigbar}, wenn
\begin{flalign*}
  \forall c \in \mathbb{N} \exists k_0 \in \mathbb{N} \forall k \geq k_0: negl(k) < \frac{1}{k^c}
\end{flalign*}

\subsubsection{Definition: EUF-CMA}\label{sec:DefinitionEUFCMA}
Ein digitales Signaturverfahren $\Sigma = (Gen, Sign, Vfy)$ ist \textit{EUF-CMA-sicher}, wenn für alle PPT $\mathcal{A}$ gilt, dass
\begin{flalign*}
   & \Pr[\mathcal{A} \text{ gewinnt EUF-CMA-Experiment}]                                                                                    \\
   & = \Pr[\mathcal{A}^{\mathcal{C}_\text{EUF-CMA}}(pk) = (m^*, \sigma^*): Vfy(pk, m^*, \sigma^*) = 1 \land m^* \notin \{m_1, \dots, m_q\}] \\
   & \leq negl(k)
\end{flalign*}
für eine im Sicherheitsparameter $k$ vernachlässigbare Funktion $negl$.

\subsection{EUF-naCMA-Sicherheitsexperiment}
\subsubsection{Visualisierung: EUF-naCMA-Sicherheitsexperiment}
\begin{tikzpicture}
  \node (A) at (0,5) {$\mathcal{C}_\text{EUF-naCMA}$};
  \node (B) at (5,5) {$\mathcal{A}$};
  \node[label={[align=center]below:$Vfy(pk, m^*, \sigma^*) = 1$?\\ $\land$ \\ $m^* \notin \{m_1, \dots, m_q\}$?}] (C) at (0,0) {};
  \node (D) at (5,0) {};

  \draw[dashed] (A) -- (C);
  \draw[dashed] (B) -- (D);

  \node[label={right:\begin{minipage}{5cm}\footnotesize\begin{itemize}
            \item{$q = q(k)$ Nachrichten}
            \item{$q$ Polynom}
          \end{itemize}\end{minipage}}] at (5,4) {};
  \node[label={[align=left]left:$(pk, sk) \leftarrow Gen(1^k)$\\$\forall i: \sigma_i \leftarrow Sign(sk, m_i)$}] at (0,3) {};

  \draw[->,shorten >=5pt, shorten <=5pt] (5,4) -- (0,3.5) node[midway, above, sloped] {$m_1, \dots, m_q$};

  \draw[->,shorten >=5pt, shorten <=5pt] (0,2.5) -- (5,2) node[midway, above, sloped] {$pk, \sigma_1, \dots, \sigma_q$};
  \draw[->,shorten >=5pt, shorten <=5pt] (5,1.5) -- (0,1) node[midway, above, sloped] {$m^*, \sigma^*$};
\end{tikzpicture}

$\mathcal{A}$ gewinnt, falls $Vfy(pk, m^*, \sigma^*) = 1$ \textbf{und} $m^* \notin \{m_1, \dots, m_q\}$


\subsubsection{Definition: EUF-naCMA}
Ein digitales Signaturverfahren $\Sigma = (Gen, Sign, Vfy)$ ist \textit{EUF-naCMA-sicher}, wenn für alle PPT $\mathcal{A}$ gilt, dass
\begin{flalign*}
   & \Pr[\mathcal{A} \text{ gewinnt EUF-naCMA-Experiment}]                                                                                \\
   & = \Pr[\mathcal{A}^{\mathcal{C}_\text{EUF-naCMA}} = (m^*, \sigma^*): Vfy(pk, m^*, \sigma^*) = 1 \land m^* \notin \{m_1, \dots, m_q\}] \\
   & \leq negl(k)
\end{flalign*}
für eine im Sicherheitsparameter $k$ vernachlässigbare Funktion $negl$.

\subsection{Einmalsignaturen}
\begin{itemize}
  \item{Ziel: Signaturen, die viele Nachrichten signieren können}
  \item{Vorstufe: Signaturen, die nur \textbf{eine} Nachricht \textbf{sicher} signieren können (\highlight{Einmalsignaturen})}
  \item{für jeden \textit{public key} sollte nur eine einzige Signatur ausgestellt werden, sonst evtl. unsicher}
\end{itemize}

\subsubsection{Sicherheitsbegriffe für Einmalsignaturen}
Analog zum vorherigen Kapitel definieren wir \textbf{EUF-1-CMA} und \textbf{EUF-1-naCMA} für Einmalsignaturen.

\subsubsection{Beziehungen zwischen Sicherheitsdefinitionen}
\begin{tikzpicture}
  \node at (0,0) {EUF-1-naCMA};
  \node at (2,0) {$\Leftarrow$};
  \node at (4,0) {EUF-1-CMA};

  \node at (0,1) {$\Downarrow$};
  \node at (4,1) {$\Downarrow$};

  \node at (0,2) {EUF-naCMA};
  \node at (2,2) {$\Leftarrow$};
  \node at (4,2) {EUF-CMA};
\end{tikzpicture}

Beweis im Skript.

\subsection{Perfekte Sicherheit}
In den Definitionen, z.B. bei \hyperref[sec:DefinitionEUFCMA]{EUF-CMA} finden sich zwei Einschränkungen, die im folgenden erläutert werden:

\subsubsection{Warum müssen wir uns auf PPT-Angreifer beschränken?}
Durch Brute-Force könnte ein unbeschränkter Angreifer alle Signaturen durchprobieren und so valide Signaturen für beliebige Nachrichten finden, wodurch er beim Sicherheitsexperiment immer gewinnen würde.

\subsubsection{Warum muss die Erfolgswahrscheinlichkeit des Angreifers nur vernachlässigbar sein?}
Die Erfolgswahrscheinlichkeit kann nicht 0 sein, da der Angreifer durch zufälliges Raten eine gültige Signatur für eine beliebige Nachricht finden könnte, wodurch er das Sicherheitsexperiment gewinnt.

\subsection{Erweiterung des Nachrichtenraumes}
Wir konstruieren fast immer Signaturen mit ``kleinem'' Nachrichtenraum, z.B.
\begin{itemize}
  \item{$\mathbb{Z}_p = \{0, \dots, p-1\}$, $p$ prim}
  \item{$\{0, 1\}^{q(k)}$, $q$ Polynom, $k$ Sicherheitsparameter}
\end{itemize}
Unser Ziel ist es jedoch, beliebige Nachrichten, z.B. $\{0, 1\}^*$, zu signieren.

\subsubsection{Hashfunktionen}\label{sec:hashfunktionen}
Eine kryptographische Hashfunktion $H = (Gen_H, Eval_H)$ ist ein Tupel aus zwei PPT-Algorithmen:
\begin{itemize}
  \item{$Gen_H(1^k)$ berechnet $t$, sodass $t$ eine Funktion
              \begin{flalign*}
                H_t: \{0, 1\}^* \rightarrow \mathcal{M}_t
              \end{flalign*}
              spezifiziert}
  \item{$Eval_H(1^k, t,x)$ berechnet $H_t(x)$}
\end{itemize}

\subsubsection{Kollisionsresistenz}
Eine Hashfunktion $H = (Gen_H, Eval_H)$ ist \highlight{kollisionsresistent}, falls für alle $t \leftarrow Gen_H(1^k)$ und für alle PPT $\mathcal{A}$ gilt, dass
\begin{flalign*}
  \Pr[\mathcal{A}(1^k, t) = (x, x'): H_t(x) = H_t(x') \land x \neq x'] \leq negl(k)
\end{flalign*}
für eine im Sicherheitsparameter $k$ vernachlässigbare Funktion $negl$.

\subsubsection{Signatur mit unbeschränktem Nachrichtenraum (\highlight{Hash-then-Sign})}
Wir wollen nun Signaturen mit unbeschränktem Nachrichtenraum konstruieren. Gegeben:
\begin{itemize}
  \item{$\Sigma' = (Gen', Sign', Vfy')$ mit Nachrichtenraum $\mathcal{M}$}
  \item{kollisionsresistente Hashfunktion $H: \{0,1\}^* \rightarrow \mathcal{M}$}
\end{itemize}
Konstruiere $\Sigma = (Gen, Sign, Vfy)$ mit Nachrichtenraum $\{0,1\}^*$:
\begin{itemize}
  \item{$Gen(1^k)$ berechnet $(pk, sk) \leftarrow Gen'(1^k)$}
  \item{$Sign(sk, m)$ berechnet $\sigma \leftarrow Sign'(sk, H(m))$}
  \item{$Vfy(pk, m, \sigma)$ gibt $Vfy'(pk, H(m), \sigma)$ aus}
\end{itemize}

\section{q-mal Signaturen}
\subsection{Von EUF-naCMA-Sicherheit zu EUF-CMA-Sicherheit}
Gegeben
\begin{itemize}
  \item{ein EUF-naCMA-sicheres Signaturverfahren $\Sigma'$ und}
  \item{ein EUF-1-naCMA-sicheres Einmalsignaturverfahren $\Sigma^{(1)}$}
\end{itemize}
können wir mittels \textbf{Transformation} ein \textbf{EUF-CMA}-sicheres Signaturverfahren $\Sigma$ konstruieren.

\subsubsection{Transformation}
Gegeben:
\begin{itemize}
  \item{EUF-naCMA-sicheres Signaturverfahren $\Sigma' = (Gen', Sign', Vfy')$}
  \item{EUF-1-naCMA-sicheres Signaturverfahren $\Sigma^{(1)} = (Gen^{(1)}, Sign^{(1)}, Vfy^{(1)})$}
\end{itemize}
Konstruiere nun $\Sigma = (Gen, Sign, Vfy)$ wie folgt:
\begin{itemize}
  \item{$Gen(1^k)$: \begin{flalign*}
                (pk, sk) \coloneqq (pk', sk') \leftarrow Gen'(1^k)
              \end{flalign*} }
  \item{$Sign(sk,m)$: \begin{flalign*}
                (pk^{(1)}, sk^{(1)}) & \leftarrow Gen^{(1)}(1^k)                   \\
                \sigma'              & \leftarrow Sign'(sk, pk^{(1)})              \\
                \sigma^{(1)}         & \leftarrow Sign^{(1)} (sk^{(1)} , m)        \\
                \sigma               & \coloneqq (pk^{(1)}, \sigma^{(1)}, \sigma') \\
              \end{flalign*} }
  \item{$Vfy(pk, m, \sigma)$ gibt 1 aus, wenn\begin{flalign*}
                Vfy'(pk, pk^{(1)}, \sigma') = 1 \land Vfy^{(1)}(pk^{(1)}, m, \sigma^{(1)}) = 1
              \end{flalign*}
              sonst 0 }
\end{itemize}
Es wird also für jede Signatur ein neues Einmalschlüsselpaar erzeugt.

\subsection{Mehrmal-Signaturverfahren aus Einmalsignaturverfahren}
Einmalsignaturverfahren sind effizient und einfach zu konstruieren, daher würden wir gerne eine Variation dieser verwenden, um mehrfach signieren zu können (q-mal-Signaturverfahren).

\subsubsection{Naiver Ansatz: q Schlüsselpaare}
Der naive Ansatz ist, $q$ Schlüsselpaare zu verwenden und einen Zähler $st \in \{1, \dots, q\}$ als Zustand zu verwenden, der auch im Secret Key und in der Signatur vorkommt:
\begin{itemize}
  \item{$Gen(1^k)$: \begin{flalign*}
                 & (pk_i, sk_i) \leftarrow Gen^{(1)}(1^k) \text{ für alle } i \in \{1, \dots, q\} \\
                 & pk \coloneqq (pk_1, \dots, pk_q)                                               \\
                 & sk \coloneqq (sk_1, \dots, sk_q, st = 1)
              \end{flalign*} }
  \item{$Sign(sk,m)$: \begin{flalign*}
                 & i         \coloneqq st                   \\
                 & \sigma_i  \leftarrow Sign^{(1)}(sk_i, m) \\
                 & \sigma    \leftarrow (\sigma_i, i)       \\
                 & st        \coloneqq st + 1               \\
              \end{flalign*} }
  \item{$Vfy(pk, m, \sigma = (\sigma_i, i))$: \begin{flalign*}
                Vfy^{(1)}(pk_i, m, \sigma_i) \stackrel{?}{=} 1
              \end{flalign*}
        }
\end{itemize}

\underline{\textbf{Eigenschaften bezogen auf Signaturanzahl (q):}}
\begin{itemize}
  \item{$|pk| \in \Theta(q)$}
  \item{$|sk| \in \Theta(q)$}
  \item{$|\sigma| \in \Theta(1)$}
\end{itemize}

\subsubsection{Zwischenschritt: Hashfunktion verwenden}
Ein weiterer möglicher Ansatz ist die verwendung einer \hyperref[sec:hashfunktionen]{Hashfunktion}
\begin{itemize}
  \item{\notice{$H$ Hashfunktion}}
  \item{$Gen(1^k)$: \begin{flalign*}
                 & (pk_i, sk_i) \leftarrow Gen^{(1)}(1^k) \text{ für alle } i \in \{1, \dots, q\} \\
                 & pk \coloneqq \notice{H}(pk_1, \dots, pk_q)                                     \\
                 & sk \coloneqq (sk_1, \dots, sk_q, \notice{pk_1, \dots, pk_q}, st = 1)
              \end{flalign*} }
  \item{$Sign(sk,m)$: \begin{flalign*}
                 & i         \coloneqq st                                         \\
                 & \sigma_i  \leftarrow Sign^{(1)}(sk_i, m)                       \\
                 & \sigma    \leftarrow (\sigma_i, i, \notice{pk_1, \dots, pk_q}) \\
                 & st        \coloneqq st + 1                                     \\
              \end{flalign*} }
  \item{$Vfy(pk, m, \sigma = (\sigma_i, i))$: \begin{flalign*}
                Vfy^{(1)}(pk_i, m, \sigma_i) \stackrel{?}{=} 1 \notice{\text{ und } H(pk_1, \dots, pk_q) \stackrel{?}{=} pk}
              \end{flalign*}
        }
\end{itemize}

\underline{\textbf{Eigenschaften bezogen auf Signaturanzahl (q):}}
\begin{itemize}
  \item{\notice{$|pk| \in \Theta(1)$}}
  \item{$|sk| \in \Theta(q)$}
  \item{\notice{$|\sigma| \in \Theta(q)$}}
\end{itemize}

\newpage
\subsubsection{Merkle-Bäume}
\highlight{Merkle-Bäume} (auch \textbf{Hash-Bäume} genannt) sind (meist binäre) Bäume, bei denen die Blätter Hashwerte der Daten sind und jeder Knoten darüber aus Hashwerten seiner Kinder besteht:\par
\begin{forest}
  for tree={circle, draw, outer sep=2pt, s sep=1cm, edge={<-, shorten <=-2pt, shorten >=-2pt}, minimum size=1.2cm}
  [{$h_{0,1}$}, label=left:{$pk \coloneqq$}
  [
  {$h_{1,1}$}
    [{$h_{2,1}$}
        [{$h_{3,1}$} [{$pk_1$}, {draw=none}]]
        [{$h_{3,2}$} [{$pk_2$}, {draw=none}]]
    ]
    [{$h_{2,2}$}
        [{$h_{3,3}$} [{$pk_3$}, {draw=none}]]
        [{$h_{3,4}$} [{$pk_4$}, {draw=none}]]
    ]
  ]
  [
  {$h_{1,2}$}
    [{$h_{2,3}$}
        [{$h_{3,5}$} [{$pk_5$}, {draw=none}]]
        [{$h_{3,6}$} [{$pk_6$}, {draw=none}]]
    ]
    [{$h_{2,4}$}
        [{$h_{3,7}$} [{$pk_7$}, {draw=none}]]
        [{$h_{3,8}$} [{$pk_8$}, {draw=none}]]
    ]
  ]
  ]
\end{forest}\par
Der \notice{Co-Pfad} eines Knotens $v$ in einem Binärbaum mit Wurzel $r$ ist die Folge aller Knoten $u_1, \dots, u_n$ wobei $u_i$ der Geschwisterknoten des $i$-ten Knotens auf dem Pfad von $v$ zu $r$ ist:\par
\begin{forest}
  for tree={circle, draw, outer sep=2pt, s sep=1cm, edge={<-, dotted, shorten <=-2pt, shorten >=-2pt}, minimum size=1.2cm}
  [{$h_{0,1}$}, label=left:{$pk \coloneqq$}
  [
  {$h_{1,1}$}, edge=solid
  [{\notice{$h_{2,1}$}}, draw=noticeColor
  [{$h_{3,1}$} [{$pk_1$}, {draw=none}]]
  [{$h_{3,2}$} [{$pk_2$}, {draw=none}]]
  ]
  [{$h_{2,2}$}, edge=solid
  [{$h_{3,3}$}, edge=solid [{\notice{$pk_3$}}, {draw=none}, edge=solid]]
  [{\notice{$h_{3,4}$}}, draw=noticeColor  [{$pk_4$}, {draw=none}]]
  ]
  ]
  [
  {\notice{$h_{1,2}$}}, draw=noticeColor
  [{$h_{2,3}$}
    [{$h_{3,5}$} [{$pk_5$}, {draw=none}]]
    [{$h_{3,6}$} [{$pk_6$}, {draw=none}]]
  ]
  [{$h_{2,4}$}
    [{$h_{3,7}$} [{$pk_7$}, {draw=none}]]
    [{$h_{3,8}$} [{$pk_8$}, {draw=none}]]
  ]
  ]
  ]
\end{forest}\par
Der \notice{Co-Pfad} wird nun in die Signatur hinzugefügt, wodurch der $pk$ von $pk_3$ ausgehend (in diesem Beispiel) in $Vfy$ berechnet werden kann.\par
\begin{itemize}
  \item{$Gen(1^k)$: \begin{flalign*}
                 & (pk_i, sk_i) \leftarrow Gen^{(1)}(1^k) \text{ für alle } i \in \{1, \dots, q\} \\
                 & pk \coloneqq \notice{\text{Baum-Hash}}(pk_1, \dots, pk_q)                      \\
                 & sk \coloneqq (sk_1, \dots, sk_q, \notice{pk_1, \dots, pk_q}, st = 1)
              \end{flalign*} }
  \item{$Sign(sk,m)$: \begin{flalign*}
                 & i         \coloneqq st                                            \\
                 & \sigma_i  \leftarrow Sign^{(1)}(sk_i, m)                          \\
                 & \sigma    \leftarrow (\sigma_i, i, \notice{pk_i, \text{Co-Pfad}}) \\
                 & st        \coloneqq st + 1                                        \\
              \end{flalign*} }
  \item{$Vfy(pk, m, \sigma)$: \begin{flalign*}
                 & \notice{\text{Berechne Wurzel } h'}                                                        \\
                 & Vfy^{(1)}(pk_i, m, \sigma_i) \stackrel{?}{=} 1 \notice{\text{ und } h' \stackrel{?}{=} pk}
              \end{flalign*}
        }
\end{itemize}
\underline{\textbf{Eigenschaften bezogen auf Signaturanzahl (q):}}
\begin{itemize}
  \item{$|pk| \in \Theta(1)$}
  \item{$|sk| \in \Theta(q)$}
  \item{\notice{$|\sigma| \in \Theta(\log q)$}}
\end{itemize}

\end{document}