\documentclass[12pt,A4]{extarticle}	
\usepackage{filecontents}

\newcommand{\lectureTitle}{Parallelrechner und Parallelprogrammierung [WIP]}
\newcommand{\semester}{Sommersemester 2024}

\newcommand{\titleSize}{\LARGE}

\usepackage[a4paper,left=0.9cm,right=1cm,top=1.37cm,bottom=2.5cm]{geometry}
\usepackage[utf8]{inputenc}
\usepackage{xifthen}
\usepackage{cmbright}
\usepackage{fontawesome}
\usepackage[T1]{fontenc}
\usepackage{lastpage,lipsum}
\usepackage{hyperref}
\usepackage{transparent}
\usepackage{color}
\usepackage{fancyhdr}

\renewcommand*\familydefault{\sfdefault}
\setlength{\parindent}{0mm}

\usepackage{transparent}
\usepackage{color}
\usepackage{fancyhdr}

\definecolor{headerBg}{RGB}{11, 67, 158}
\definecolor{headerGrayColor}{RGB}{210, 210, 210}

\pagestyle{fancy}
\fancyhead[C]{
  \fcolorbox{headerBg}{headerBg}{
    \hspace{0.6cm}\begin{minipage}[c][50pt][c]{\paperwidth}
      \begin{minipage}[c]{.45\textwidth}
        \huge{\textcolor{white}{Vorlesung}}\normalsize\\
        \textcolor{headerGrayColor}{\small{Wintersemester 2022/23}}
      \end{minipage}%
      \begin{minipage}[c]{.45\textwidth}
        \raggedleft
        \textcolor{white}{
          \small{\href{https://nilslambertz.de/}{nilslambertz.de}}\\
          \href{https://github.com/nilslambertz/}{\textcolor{white}{\faicon{github}} \small{nilslambertz}}}
      \end{minipage}
    \end{minipage}}
}
\renewcommand{\headrulewidth}{0pt}
\setlength{\headheight}{40pt}

\newlength{\oddmarginwidth}
\setlength{\oddmarginwidth}{1in+\hoffset+\oddsidemargin}
\newlength{\evenmarginwidth}
\setlength{\evenmarginwidth}{\evensidemargin+1in}
\fancyhfoffset[LO,RE]{\oddmarginwidth}
\fancyhfoffset[LE,RO]{\evenmarginwidth}
\cfoot{\thepage\ $/$ \pageref*{LastPage}}


\definecolor{highlightColor}{RGB}{66, 135, 245}
\newcommand{\highlight}[1]{\textcolor{highlightColor}{\textbf{#1}}}

\def\contentsname{\empty}

\begin{document}

\disclaimer

\tableofcontents
\clearpage

\section{Einführung}
\subsection{Warum gibt es Parallelrechner?}
\begin{itemize}
    \item{Verkürzung der \textbf{Ausführungszeit} von Anwendungen}
    \item{Lösung von Problemen mit \textbf{größerer Komplexität} bzw. \textbf{feinerer Auflösung}}
    \item{bessere \textbf{Fehlertoleranz} (z.B. durch Mehrfachberechnung)}
    \item{wissenschaftliches Interesse}
\end{itemize}

\subsection{Anwendungsgebiete von Parallelrechnern}
\begin{itemize}
    \item{\textbf{Wissenschaftliche Simulationen}: Wettervorhersage, Klimamodellierung etc.}
    \item{\textbf{Datenanalyse}}
    \item{\textbf{Simulationen in der Industrie}, z.B. beim Flugzeugbau oder in der Automobilindustrie}
\end{itemize}

\subsection{Warum brauchen wir Simulationen?}
\begin{itemize}
    \item{Prozesse oft zu langsam und komplex, um Forschung zu ermöglichen}
    \item{Experimente sind oft \textbf{teuer} und \textbf{aufwändig}}
\end{itemize}

\subsection{Grundbegriffe}
\subsubsection{Parallelrechner}
Ein \highlight{Parallelrechner} besteht aus einer Menge von Verarbeitungselementen, die in einer koordinierten Weise — teilweise zeitgleich — zusammenarbeiten, um eine Aufgabe zu lösen.\par
\textbf{Verarbeitungselemente} sind z.B.
\begin{itemize}
    \item{\textbf{Gleichartige Rechenwerke}, z.B. die Verarbeitungselemente eines Feldrechners}
    \item{\textbf{Prozessorknoten} eines Multiprozessors}
    \item{\textbf{Vollständige Rechner}}
\end{itemize}

\end{document}