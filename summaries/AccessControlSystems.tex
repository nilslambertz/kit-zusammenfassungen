\documentclass[12pt,A4]{extarticle}

\begin{filecontents}{\jobname.bib}
  @phdthesis{nobi2022towards,
  title={Towards Machine Learning Based Access Control},
  author={Nobi, Mohammad Nur},
  year={2022},
  school={The University of Texas at San Antonio}
}
@ARTICLE{salzer1975protection,
  author={Saltzer, J.H. and Schroeder, M.D.},
  journal={Proceedings of the IEEE}, 
  title={The protection of information in computer systems}, 
  year={1975},
  volume={63},
  number={9},
  pages={1278-1308},
  keywords={Protection;Authorization;Permission;Access control;Terminology;Data security;Information security;Computer architecture;Modems},
  doi={10.1109/PROC.1975.9939}}
@ARTICLE{AContemporaryLookAtSaltzerAndSchroeders1975DesignPrinciples,
  author={Smith, Richard E.},
  journal={IEEE Security and Privacy}, 
  title={A Contemporary Look at Saltzer and Schroeder's 1975 Design Principles}, 
  year={2012},
  volume={10},
  number={6},
  pages={20-25},
  keywords={Information security;Standards;Cryptography;Design methodology;Privacy;Computer security;security;protection mechanisms;Saltzer;Schroeder},
  doi={10.1109/MSP.2012.85}}
\end{filecontents}

\newcommand{\lectureTitle}{Access Control Systems [WIP]}
\newcommand{\lectureSubtitle}{Models and Technology}
\newcommand{\semester}{Sommersemester 2024}

\newcommand{\titleSize}{\LARGE}

\usepackage[a4paper,left=0.9cm,right=1cm,top=1.37cm,bottom=2.5cm]{geometry}
\usepackage[utf8]{inputenc}
\usepackage{xifthen}
\usepackage{cmbright}
\usepackage{fontawesome}
\usepackage[T1]{fontenc}
\usepackage{lastpage,lipsum}
\usepackage{hyperref}
\usepackage{transparent}
\usepackage{color}
\usepackage{fancyhdr}

\renewcommand*\familydefault{\sfdefault}
\setlength{\parindent}{0mm}

\usepackage{transparent}
\usepackage{color}
\usepackage{fancyhdr}

\definecolor{headerBg}{RGB}{11, 67, 158}
\definecolor{headerGrayColor}{RGB}{210, 210, 210}

\pagestyle{fancy}
\fancyhead[C]{
  \fcolorbox{headerBg}{headerBg}{
    \hspace{0.6cm}\begin{minipage}[c][50pt][c]{\paperwidth}
      \begin{minipage}[c]{.45\textwidth}
        \huge{\textcolor{white}{Vorlesung}}\normalsize\\
        \textcolor{headerGrayColor}{\small{Wintersemester 2022/23}}
      \end{minipage}%
      \begin{minipage}[c]{.45\textwidth}
        \raggedleft
        \textcolor{white}{
          \small{\href{https://nilslambertz.de/}{nilslambertz.de}}\\
          \href{https://github.com/nilslambertz/}{\textcolor{white}{\faicon{github}} \small{nilslambertz}}}
      \end{minipage}
    \end{minipage}}
}
\renewcommand{\headrulewidth}{0pt}
\setlength{\headheight}{40pt}

\newlength{\oddmarginwidth}
\setlength{\oddmarginwidth}{1in+\hoffset+\oddsidemargin}
\newlength{\evenmarginwidth}
\setlength{\evenmarginwidth}{\evensidemargin+1in}
\fancyhfoffset[LO,RE]{\oddmarginwidth}
\fancyhfoffset[LE,RO]{\evenmarginwidth}
\cfoot{\thepage\ $/$ \pageref*{LastPage}}


\definecolor{purple}{RGB}{153,71,155}
\definecolor{highlightColor}{RGB}{66, 135, 245}
\newcommand{\highlight}[1]{\textcolor{highlightColor}{\textbf{#1}}}

\def\contentsname{\empty}

\begin{document}

\disclaimer{}

\tableofcontents
\clearpage

\section{Introduction}
\subsection{What is Access Control?}
\subsubsection{Definition according to OSI Basic Reference Model (1989)}
\begin{itemize}
  \item{\textbf{Access Control}: ``The prevention of unauthorized use of a resource''}
  \item{\textbf{Authorization}: ``The granting of rights, which includes the granting of access based on access rights''}
  \item{Listed as a \textbf{security service}: ``This service provides protection against unauthorized use of resources''}
  \item{Based on \textbf{policies}: ``The control of access will be in accordance with various security policies''}
\end{itemize}

\subsubsection{Aspects of Access Control}
\begin{itemize}
  \item{\textbf{Policy}: Focus on models, specification of security policies (What does it mean to be secure?)}
  \item{\textbf{Enforcement}: Focus on technologies, design and implementation of systems (How do we make it secure?)}
\end{itemize}

\subsection{Creating and enforcing policies}
Classification by \cite{nobi2022towards}:
\begin{itemize}
  \item{\textbf{Policy Authoring/Engineering}: Define a policy model and access control rules}
  \item{\textbf{Policy Verification and Testing}: Test for leaks or contradictory privileges}
  \item{\textbf{Policy Administration}: Maintain and update policies/configurations}
  \item{\textbf{Policy Enforcement}}
\end{itemize}

\section{Foundations}
\subsection{Design Principles}
\subsubsection{Design Principles of Saltzer and Schroeder (1975)}
Saltzer and Schroeder \cite{salzer1975protection} proposed the following design principles for secure systems:
\begin{itemize}
  \item{\textbf{Economy of mechanism}: Keep the design as simple and small as possible}
  \item{\textbf{Fail-safe defaults}: Base access decisions on permission, not exclusion (default is lack of access)}
  \item{\textbf{Complete mediation}: Every access to every object must be checked}
  \item{\textbf{Open design}: Security through obscurity is not a good idea}
  \item{\textbf{Separation of privilege}: Divide access rights into multiple parts}
  \item{\textbf{Least privilege}: Give programs and users only the permissions they need for the job}
  \item{\textbf{Least common mechanism}: Minimize shared mechanisms}
  \item{\textbf{Psychological acceptability}: Make security mechanisms easy to use and understand}
\end{itemize}

\subsubsection{A Contemporary Look at Saltzer and Schroeder's 1975 Design Principles}
Richard Smith \cite{AContemporaryLookAtSaltzerAndSchroeders1975DesignPrinciples} took a contemporary look at the design principles and found that most of them are still relevant today. He listed the following principles:
\begin{itemize}
  \item{\textbf{Continuous improvement}: Security is a process, not a product}
  \item{\textbf{Least privilege}}
  \item{\textbf{Defense in depth}: Use multiple layers of security}
  \item{\textbf{Open design}}
  \item{\textbf{Chain of control}: Ensure that trustworhy software is being executed}
  \item{\textbf{Deny by default}: Default should be lack of access}
  \item{\textbf{Transitive trust}: Trust should be transitive}
  \item{\textbf{Separation of duty}: Critical tasks should be divided among multiple individuals or entities}
\end{itemize}

\subsection{Policy Administration: Who is in charge of setting policies?}
\subsubsection{Discretionary Access Control (DAC)}
In \highlight{Discretionary Access Control (DAC)} each resource is assigned an owner. The owner can decide who has access to the resource.

\subsubsection{Mandatory Access Control (MAC)}
In \highlight{Mandatory Access Control (MAC)} access is controlled by the system, not the owner. The system enforces access control based on a system-wide policy.

\subsection{Access Control Matrix (ACM)}
The \textbf{Protection State} of a system consists of all components and their privileges over each other. The \highlight{Access Control Matrix (ACM)} encodes the protection state.

\begin{center}
  \begin{tabular}{c|c|c|c|c|c|c|c|}
    \multicolumn{1}{c}{} & \multicolumn{1}{c}{$s_1$} & \multicolumn{1}{c}{$\dots$} & \multicolumn{1}{c}{$s_n$} & \multicolumn{1}{c}{$o_1$} & \multicolumn{1}{c}{\dots} & \multicolumn{1}{c}{$o_m$} \\
    \cline{2-7}
    $s_1$                &                           &                             &                           &                           &                           &                           \\
    \cline{2-7}
    $\vdots$             &                           &                             &                           &                           &                           &                           \\
    \cline{2-7}
    $s_n$                &                           &                             &                           &                           &                           &                           \\
    \cline{2-7}
  \end{tabular}
\end{center}

\begin{itemize}
  \item{Subjects $S = \{s_1, \dots, s_n\}$}
  \item{Objects $O = S \cup \{o_1, \dots, o_m\}$}
  \item{Rights $R = \{r_1, \dots, r_k\}$}
  \item{Entries $A[s_i, o_j] \subseteq R$}
\end{itemize}

\subsubsection{Primitive Operations}
\begin{itemize}
  \item{\textbf{create subject $s$; create object} $o$ (Creates new column/row)}
  \item{\textbf{destroy subject $s$; destroy object} $o$ (Removes column/row)}
  \item{\textbf{enter $r$ into $A[s,o]$} (Adds right $r$ for subject $s$ over object $o$)}
  \item{\textbf{delete $r$ from $A[s,o]$} (Removes right $r$ for subject $s$ over object $o$)}
\end{itemize}
Modification of the ACM usually requires executing a sequence of operations.

\subsubsection{Commands}
\textbf{command} \textcolor{purple}{\textless name\textgreater}(\textcolor{green}{\textless parameters\textgreater}) \\
\textbf{if}\\
\hspace*{10mm}{\textcolor{blue}{\textless conditions\textgreater}}\\
\textbf{then}\\
\hspace*{10mm}{\textcolor{red}{\textless operations\textgreater}}\\
\textbf{end}\par
If all \textcolor{blue}{conditions} are met, all \textcolor{red}{operations} (conjunction $\land$) are executed, otherwise nothing happens.

\subsubsection{Commands: Example}
\textbf{command} \textcolor{purple}{make.file}(\textcolor{green}{p, f}) \\
\textbf{
  \hspace*{10mm}{\textcolor{blue}{create object $f$;}}\\
  \hspace*{10mm}{\textcolor{blue}{center $own$ into $A[p, f]$;}}\\
  \hspace*{10mm}{\textcolor{blue}{enter $r$ into $A[p, f]$;}}\\
  \hspace*{10mm}{\textcolor{blue}{enter $w$ into $A[p, f]$;}}\\
  end}

\subsubsection{Leak, Safety, Safety Question}
\begin{itemize}
  \item{\textbf{Leak}: Addings a right $r$ where there was not one before is called \textit{leaking}}
  \item{\textbf{Safety}: If a system $S$, beginning in initial state $s_0$ cannot leak right $r$, it is safe with respect to the right $r$}
  \item{\textbf{Safety Question}: Does there exist \textbf{one} algorithm whether an \textbf{arbitrary} protection system $S$ with initial state $s_0$ is safe with respect to a generic right $r$?}
\end{itemize}
The Safety Question is decicable for \textbf{mono-operational} commands. In general, the Safety Question is undecidable.

\newpage
\bibliographystyle{apalike}
\bibliography{\jobname}


\end{document}