\documentclass[12pt,A4]{extarticle}	

\newcommand{\lectureTitle}{Markenrecht}
\newcommand{\semester}{Wintersemester 2022/23}

\usepackage[a4paper,left=0.9cm,right=1cm,top=1.37cm,bottom=2.5cm]{geometry}
\usepackage[utf8]{inputenc}
\usepackage{xifthen}
\usepackage{cmbright}
\usepackage{fontawesome}
\usepackage[T1]{fontenc}
\usepackage{lastpage,lipsum}
\usepackage{hyperref}
\usepackage{transparent}
\usepackage{color}
\usepackage{fancyhdr}

\renewcommand*\familydefault{\sfdefault}
\setlength{\parindent}{0mm}

\definecolor{headerBg}{RGB}{11, 67, 158}
\definecolor{headerGrayColor}{RGB}{210, 210, 210}

\newcommand{\printTitle}{\textcolor{white}{\lectureTitle}\normalsize}
\newcommand{\printSubtitle}{
  \ifdefined\lectureSubtitle
    \textcolor{white}{\small{\lectureSubtitle}}\\
  \fi
}

\fancyhf{}
\pagestyle{fancy}
\fancyhead[C]{
  \fcolorbox{headerBg}{headerBg}{
    \hspace{0.6cm}\begin{minipage}[c][50pt][c]{\paperwidth}
      \begin{minipage}[c]{.7\textwidth}
        \ifdefined\titleSize
          \titleSize \printTitle\\
        \else
          \huge\printTitle\\
        \fi
        \printSubtitle
        \textcolor{headerGrayColor}{\small{\semester}}
      \end{minipage}%
      \begin{minipage}[c]{.2\textwidth}
        \raggedleft
        \textcolor{white}{
          \small{\href{mailto:mail@nilslambertz.de}{\textcolor{white}{\faicon{envelope}} mail@nilslambertz.de}}\\
          \href{https://github.com/nilslambertz/kit-zusammenfassungen}{\textcolor{white}{\faicon{github}} \small{nilslambertz}}}
      \end{minipage}
    \end{minipage}}
}
\renewcommand{\headrulewidth}{0pt}
\setlength{\headheight}{40pt}

\newlength{\oddmarginwidth}
\setlength{\oddmarginwidth}{1in+\hoffset+\oddsidemargin}
\newlength{\evenmarginwidth}
\setlength{\evenmarginwidth}{\evensidemargin+1in}
\fancyhfoffset[LO,RE]{\oddmarginwidth}
\fancyhfoffset[LE,RO]{\evenmarginwidth}
\cfoot{\thepage\ $/$ \pageref*{LastPage}}

\definecolor{greenColor}{RGB}{15, 122, 25}
\newcommand{\green}[1]{\textcolor{greenColor}{#1}}

\definecolor{highlightColor}{RGB}{66, 135, 245}
\newcommand{\highlight}[1]{\textcolor{highlightColor}{\textbf{#1}}}
\definecolor{markenGesetzLink}{RGB}{194, 74, 14}
\newcommand{\markenG}[2][]{\textbf{\textcolor{markenGesetzLink}{\href{https://www.gesetze-im-internet.de/markeng/__#2.html}{§ #2 \ifthenelse{\equal{#1}{}}{}{#1 }MarkenG}}}}
\newcommand{\markenGG}[2][]{\textbf{\textcolor{markenGesetzLink}{\href{https://www.gesetze-im-internet.de/markeng/__#2.html}{§§ #1 MarkenG}}}}
\newcommand{\bgb}[2][]{\textbf{\textcolor{markenGesetzLink}{\href{https://www.gesetze-im-internet.de/bgb/__#2.html}{§ #2 \ifthenelse{\equal{#1}{}}{}{#1 }BGB}}}}
\newcommand{\uwg}[2][]{\textbf{\textcolor{markenGesetzLink}{\href{https://www.gesetze-im-internet.de/uwg_2004/__#2.html}{§ #2 \ifthenelse{\equal{#1}{}}{}{#1 }UWG}}}}
\newcommand{\umv}[2][]{\textbf{\textcolor{markenGesetzLink}{\href{https://lxgesetze.de/umv/#2}{Art. #2 \ifthenelse{\equal{#1}{}}{}{#1 }UMV}}}}


\def\contentsname{\empty}

\begin{document}

\tableofcontents
\clearpage

\section{Grundlagen}

\begin{itemize}
  \item{\textbf{Immatrielle Güter} sind nicht körperliche Vermögensgegenstände, also nicht greifbare Dinge}
  \item{\textbf{Gewerblicher Rechtsschutz} besteht u.a. aus Patentrecht, Lauterkeitsrecht und \textbf{Markenrecht}}
  \item{es existieren viele Berührungspunkte mit anderen Rechten z.B. im \textbf{Grundgesetz} oder im \textbf{Verwaltungsrecht}}
  \item{für Markenrechte existieren nationale, EU-weite und internationale Gesetze und Abkommen}
\end{itemize}

\subsection{Materielles vs. formelles Recht}
\textbf{Materielles Recht} (``Recht haben''):
\begin{itemize}
  \item{beschäftigt sich mit den vorliegenden Tatsachen}
  \item{beschreibt, \textbf{wie eine Marke aussehen muss}, um geschützt werden zu können}
\end{itemize}
\textbf{Formelles Recht}
\begin{itemize}
  \item{Rechtsnormen, die zur \textbf{Durchsetzung} des materiellen Rechts dienen}
  \item{beschreibt die \textbf{erforderlichen Schritte}, um eine Marke schützen zu lassen}
\end{itemize}

\section{Einführung: Markenrecht}
\subsection{Was ist das Ziel von Markenrechten?}
\begin{itemize}
  \item{Marken sollen vor Benutzung durch Dritte geschützt werden}
  \item{Inhaber erhält \textit{eigentumsähnliche} Rechte (\textbf{Subjektive Privatrechte})}
  \item{durch Marken können \textbf{Produkte} unterschiedlicher Hersteller \textbf{unterschieden werden}}
  \item{\textbf{Immatrieller Wert} eines Unternehmens besteht zu großem Teil aus Marken}
\end{itemize}

\subsection{Grundlegende Funktionen von Marken}
\begin{itemize}
  \item{\highlight{Herkunftsfunktion}: Marken kennzeichnen Hersteller}
  \item{\highlight{Unterscheidungsfunktion}: Marken unterscheiden Hersteller voneinander}
  \item{Garantie-/Qualitätsfunktion: Verbraucher verlassen sich auf gleichbleibende Qualität}
  \item{Werbefunktion: Wiedererkennungseffekt der Marke}
\end{itemize}

\subsection{Grundsätze}
\subsubsection{Grundsatz der \highlight{Priorität}}
\begin{itemize}
  \item{Rechte verschiedener Rechteinhaber erzeugen einen \textbf{Interessenkonflikt} bzw. eine \textbf{Kollision}}
  \item{\textbf{Prioritätsjüngeres} Kennzeichen \textbf{muss} dem prioritätsälterem \textbf{weichen}}
  \item{``Wer zuerst den Schutz erlangt hat, der gewinnt''}
\end{itemize}

\subsubsection{Grundsatz des \highlight{Produktbezugs} der Marke}\label{sec:Produktbezug}
\begin{itemize}
  \item{Marke wird für bestimmte Produkte/Produktklassen eingetragen}
  \item{Schutz gilt \textbf{nur für diese Produkte}}
  \item{Schutzbereich kann nachträglich nicht erweitert, sondern nur eingeschränkt werden}
\end{itemize}

\section{Markenschutz}
\subsection{Wie entsteht Markenschutz?}
Die drei Möglichkeiten zur Entstehung des Markenschutzes sind in \markenG{4} geregelt:
\begin{enumerate}
  \item{\highlight{Eintragung} der Marke durch das \textbf{DPMA} (\textbf{Registermarken})}
  \item{\highlight{Benutzungsmarken} (auch Gewohnheitsmarken genannt), wenn sie Verkehrsgeltung erlangt haben (dazu wird die Bekanntheit in den zuständigen Verkehrskreisen und die Eigenschaften der Marke wie Prägnanz, Originalität und Unterscheidungskraft betrachtet)}
  \item{\textbf{Ausländische Marken}, die im Heimatland geschützt sind, aber in Deutschland noch nicht verwendet werden}
\end{enumerate}

\subsubsection{Was geschieht bei Markenkollisionen?}
Bei Kollisionen zweier Marken wird nach der \highlight{Priorität} entschieden. Derjenige, der den Markenschutz früher erlangt hat, hat Anspruch darauf.
Bei eingetragenen Marken ist das Datum eindeutig festgelegt (siehe später).
Bei Benutzungsmarken muss letztendlich ein Gericht ein \textbf{Priorirätsdatum} nennen, woraufhin ein Gutachten zur Bekanntheit der Marke an diesem Datum erstellt wird.

\subsection{Eingetragene Marke vs. Benutzungsmarke}
\bgroup
\def\arraystretch{1.5}
\begin{table}[h]
  \begin{tabular}{|l|p{0.4\linewidth}|p{0.4\linewidth}|}
    \hline \rowcolor{gray!30}
                     & Eingetragene Marke                                                                                                                                    & Benutzungsmarke                       \\ \hline
    Entstehung       & Antrag beim DPMA                                                                                                                                      & nach Erlangung von Verkehrsgeltung    \\ \hline
    Rechtssicherheit & Gewisse \textbf{Rechtssicherheit}: Richter ist an Entscheidung des DPMA gebunden und kann den Schutz nicht aufheben, auch wenn er anderer Meinung ist & Richter entscheidet über Markenschutz \\ \hline
    Priorität        & Ab Tag der Anmeldung (siehe später)                                                                                                                   & Erst nach Verwendung und Etablierung  \\ \hline
    Schutzbereich    & Bundesweit                                                                                                                                            & In bestimmten Fällen nur lokal        \\ \hline
  \end{tabular}
\end{table}
\egroup

\subsubsection{Vorteile eingetragener Marken gegenüber Benutzungsmarken}
Grundsätzliche sind eingetragene Marken \textbf{NICHT} mehr wert als Benutzungsmarken und werden (falls Markenschutz für die Benutzungsmarke besteht) genau gleich behandelt.
Dennoch ergeben sich folgende Vorteile für eingetragene Marken:
\begin{itemize}
  \item{sie sind \green{ab Tag der Anmeldung geschützt}, auch wenn sie nicht verwendet werden}
  \item{sie sorgen für \green{Rechtssicherheit} (und seltenere Rechtsstreits)}
  \item{Schutz gilt \green{bundesweit}}
  \item{zur \textbf{IR-Marke} wird eine eingetragene Basismarke benötigt}
\end{itemize}

\subsection{Materielle Voraussetzungen zur Eintragung einer Marke}
Für die Eintragung einer Marke sind \markenG{3} und \markenG{8} relevant.

\subsubsection{§ 3 MarkenG}
\markenG{3} definiert die Markenfähigkeit von Zeichen (gilt auch für Benutzungsmarken!):
\begin{itemize}
  \item{Marke muss \highlight{abstrakte Unterscheidungskraft} besitzen (d.h. sie muss Waren unterscheiden können, ohne ein spezielles Produkt zu betrachten)}
  \item{Marke muss \textbf{zeichenfähig} sein, also auf die Sinnesorgane einwirken (z.B. visuell wahrgenommen werden können)}
  \item{Marke darf \textbf{kein funktionell notwendiger Bestandteil} des Produkts sein}
\end{itemize}
Diese Voraussetzungen sind von fast jedem Zeichen erfüllt.

\subsubsection{§ 8 MarkenG (Absolute Schutzhindernisse)}
\markenG{8} definiert \highlight{absolute Schutzhindernisse}, die das Eintragen einer Marke unmöglich machen.
Diese werden \textbf{von Amts wegen} (v. A. w.) geprüft, d.h. sie werden vom DPMA vor jeder Eintragung überprüft.\par
\textbf{§ 8 MarkenG gilt nicht für Benutzungsmarken!}\par
Absatz 1 sagt aus, dass das Zeichen in irgendeiner ``sichtbaren'' und speicherbaren Form vorliegen muss (z.B. ein Bild oder Video), damit werden u.a. Gerüche ausgeschlossen.\par
Absatz 2 enthält die wichtigen Schutzhindernisse:
\begin{enumerate}
  \item{Marke muss \highlight{konkrete Unterscheidungskraft} (unter Betrachtung des Produkts) besitzen, damit die Herkunftsfunktion erfüllt werden kann (``Waschmaschine'' für Waschmaschinen ist nicht konkret unterscheidungskräftig)}
  \item{\textbf{Freihaltebedürfnis} für grundlegende Begriffe der geschäftlichen Kommunikation. Die Stärke des Freihaltebedürfnisses hängt davon ab, ob und wie die Mitbewerber das Zeichen als beschreibende Eigenschaften ihrer Produkte benötigen.
              \begin{itemize}
                \item{Je größer das Freihaltebedürfnis, desto höher die Anforderungen an Unterscheidungskraft}
                \item{Je weniger beschreibend, desto eher unterscheidungskräftig ist es}
              \end{itemize}
        }
  \item{\textbf{Gattungsbezeichnungen} können für das Produkt nicht geschützt werden (``Diesel'' kann nicht für Motoren geschützt werden, für Kleidung allerdings schon)}
  \item{\textbf{Täuschende Zeichen} können nicht geschützt werden, z.B. unberechtigte Titel (Dr. Oetker musste wirklich Doktor sein) oder falsche Altersangaben (``seit 1900'')}
  \item{\textbf{Verbotene und sittenwidrige Zeichen} können nicht geschützt werden}
  \item{\textbf{Wappen oder Flaggen} können nicht geschützt werden}
\end{enumerate}
Bei \highlight{Verkehrsdurchsetzung} (extrem hoher Bekanntheit in den beteiligten Verkehrskreisen) existieren Ausnahmen von Absatz 2 Nr. 1, 2 und 3, siehe \markenG[Abs. 3]{8}

\subsection{Markentypen/Markenfähige Zeichen}
Grundsätzlich sind alle Zeichen als Marke zugelassen, nicht nur die, die in \markenG[Abs. 1]{3} aufgezählt werden, darunter fallen
\begin{itemize}
  \item{Wortmarken (wenn sie nicht nur das Produkt beschreiben)}
  \item{Buchstaben (falls sie nicht als branchenüblichen Beschreibung von Objekten, Klassen usw. verwendet werden)}
  \item{Bildmarken wie Logos und grafisch gestaltete Schriftzüge (einfachste geomatrische Formen sind nicht unterscheidungskräftig), auch Wort-Bild-Marken möglich}
  \item{Farben haben i.d.R. keine Unterscheidungskraft (Ausnahmen bestätigen die Regel, z.B. bei Verkehrsdurchsetzung)}
  \item{Geruchs- und Geschmacksmarken sind \textbf{nicht schutzfähig}}
  \item{Bewegungsmarke (z.B. brüllender Löwe in Film-Intros)}
  \item{Positionsmarken (z.B. Knopf im Ohr bei Steiff)}
  \item{dreidimensionale Gestaltungen:
              \begin{itemize}
                \item{Formmarken (z.B. Mercedes-Stern)}
                \item{Warenformmarken (z.B. quadratische Form von Ritter Sport)}
                \item{Warenverpackungsformmarken (z.B. Coca-Cola-Flasche)}
              \end{itemize}
        }
\end{itemize}

\subsubsection{Besonderheiten für dreidimensionale Gestaltungen}
Gemäß \markenG[Abs. 2]{3} sind einige Formen nicht schutzfähig:
\begin{enumerate}
  \item{wenn es sich um die Grundform des Produkts handelt (z.B. Birnen-Form für Birnen)}
  \item{wenn die Form technisch bedingt ist (z.B. Form von Klemmbausteinen)}
  \item{wenn die Form der Waren einen wesentlichen (oft ästhetischen) Wert verleiht (z.B. die Form eines Rings)}
\end{enumerate}

\subsubsection{Rechtserhaltung bei Marken}
Die \textbf{Rechtserhaltung} (Benutzungspflicht der Marke zur Wahrung der Rechte) ist bei Wortmarken einfacher zu gewährleisten, da beliebige Schriftarten und Farben verwendet werden können.

\subsection{Anmeldung einer Marke}
\subsubsection{Marken doppelt anmelden}
In bestimmten Fällen ist es sinnvoll, die Marke mehrfach anzumelden.
Während Wortmarken von einem relativ großen Schutzbereich umfasst werden (werden ohne feste Schriftart eingetragen, alle verkehrsüblichen Schreibweisen sind umfasst),
sind Wort-Bild-Marken auf diese Schriftart begrenzt.\par
Dadurch könnten Mitbewerber den Schriftzug verwenden, wenn die Grafik weggelassen wird. Dagegen hilft die \textbf{doppelte Eintragung} als Wortmarke und Wort-Bild-Marke.\par
Es empfiehlt sich ebenfalls, Bildmarken in \textbf{mehreren Farben} (zumindest schwarz/weiß und farbig) anzumelden.

\subsubsection{Strategie der Eintragung}
Bei einem größeren Produktportfolio können verschiedene Strategien bei der Eintragung angewendet werden:
\begin{itemize}
  \item{\highlight{Einzelmarkenstrategie}:
              \begin{itemize}
                \item{jedes Produkt unter eigener Marke}
                \item{sinnvoll, wenn die Zielgruppe sich scharf abgrenzen lässt}
                \item{relativ hohe Kosten}
              \end{itemize}
        }
  \item{\highlight{Dachmarkenstrategie}:
              \begin{itemize}
                \item{alle Produkte unter einer Dachmarke}
                \item{sinnvoll, wenn Produktangebot zu groß}
                \item{wenig Aufwand, Einführung neuer Produkte wird vereinfacht (bekannte Marke für Kunden)}
              \end{itemize}
        }
  \item{\highlight{Familienmarkenstrategie}:
              \begin{itemize}
                \item{jede Produktgruppe unter eigener Marke}
                \item{praktisch Dachmarke für jede Produktgruppe mit den daraus entstehenden Vorteilen}
              \end{itemize}
        }
\end{itemize}

\subsection{Weitere Markentypen}
\subsubsection{Kollektivmarken}
Neben \highlight{Individualmarken} können auch \highlight{Kollektivmarken} (\markenG{97}) eingetragen werden:
\begin{itemize}
  \item{Kollektivmarken können nur von \textbf{rechtsfähigen Verbänden} eingetragen werden}
  \item{Der Verband muss eine Satzung beifügen, wann die Marke verwendet werden darf}
  \item{jeder (auch Unternehmen), der die Voraussetzungen erfüllt, darf die Marke verwenden}
  \item{gemäß \markenG[Abs. 1]{97} können auch geographische Marken geschützt werden, was bei Individualmarken durch die absoluten Schutzhindernisse gemäß \markenG[Abs. 2 S. 2]{8} nicht möglich ist}
  \item{Kollektivmarken können als Gütezeichen z.B. für Lebensmittel verwendet werden, wenn ein Gutachten vorliegt}
\end{itemize}

\subsubsection{Gewährleistungsmarken}
Seit 2019 können gemäß \markenGG[106a f.]{106a} auch \highlight{Gewährleistungsmarken} eingetragen werden.
\begin{itemize}
  \item{dienen als \textbf{Güte- oder Prüfsiegel}}
  \item{werden von \textbf{neutralen Zertifizierungsstellen} eingetragen}
  \item{besondere Qualität oder Eigenschaften sollen gewährleistet werden (anders als Herkunftsfunktion bei Individualmarken)
              \begin{itemize}
                \item{\textbf{gewährleistender Charakter} muss ich unmittelbar aus Marke ergeben}
                \item{Satzung muss beigefügt werden (Eigenschaften, die gesichert werden und Prüfmechanismen)}
              \end{itemize}
        }
\end{itemize}

\textbf{Wichtige Eigenschaften von Gewährleistungsmarken}
\begin{itemize}
  \item{\textbf{Neutralität}: Selbstzertifizierung nicht möglich (Hersteller oder Vertreiber nicht erlaubt)}
  \item{\textbf{Überwachung}: Prüfung des Mechanismus und Erteilung der Marke wird kontrolliert}
  \item{\textbf{Transparenz} für Verbraucher}
\end{itemize}

Bei Verstößen gegen die Satzung kann auf \textbf{Schadensersatz} geklagt werden. Klagebefugt ist der Markeninhaber und gemäß \markenG[Abs. 1]{106c} jede zur Nutzung berechtigte Person, wenn der Markeninhaber zustimmt.

\section{Formelles Markenrecht}
Für den Eintrag einer Marke muss ein Antrag beim DPMA gestellt werden, dabei wird das Zeichen auf materielle und formelle Voraussetzungen geprüft. Falls diese erfüllt sind, besteht ein Anspruch auf Eintragung gemäß \markenG[Abs. 2]{33}.

\subsection{Anwartschaft}
Nachdem alle erforderlichen Schritte unternommen wurden und der Antrag beim DPMA eingegangen ist, ist man in der Position der \textbf{Anwartschaft}. Dies bedeutet, dass alles mögliche unternommen wurde und keine weiteren Schritte unternommen werden können.

\subsection{Anmeldeverfahren §§ 32 ff. MarkenG}
Für die Anmeldung muss ein \textbf{Formblatt} ausgefüllt und eingereicht werden. \textbf{Der Einreichungstag gilt als Anmeldedatum} (für Priorität wichtig), man erhält eine Empfangsbestätigung mit Aktenzeichen, Datum und fälligen Gebühren.\par
Für die Anmeldung müssen die Unterlagen gemäß \markenG[Abs. 2 und 3]{32} vorliegen. Für das Anmeldedatum ist \textbf{nur Absatz 2} relevant (\markenG[Abs. 1]{33}).\par
Falls die Anmeldung bzgl \markenG[Abs. 2]{32} fehlerhaft ist, verschiebt sich der Priorität auf den Tag der Nachreichung. Bei fehlenden Angaben nach Absatz 3 hat dies keinen Einfluss auf die Priorität (solange die Angaben fristgerecht nachgereicht werden).\par
Für die erfolgreiche Anmeldung ist gemäß \markenG[Abs. 1 S. 3]{36} die fristgerechte Zahlung der Gebühren wichtig.

\subsubsection{Klassifikation für das Warenverezichnis}
Zu jeder Anmeldung muss ein \textbf{Waren- und Dienstleisungsverzeichnis} (\ref{sec:Produktbezug}) beigefügt werden. Dabei kann die \highlight{Klassifikation von Nizza} verwendet werden, die 34 Waren- und 11 Dienstleistungsklassen unterscheidet. Es können allerdings auch detaillierte Produkte spezifiziert werden.

\subsubsection{Kette der Vorschriften, die geprüft werden}
Die Vorschriften werden in folgender Reihenfolge geprüft:
\begin{enumerate}
  \item{\markenG[Abs. 1]{36}}
  \item{\markenG[Abs. 1]{33}}
  \item{\markenG[Abs. 2]{32}}
\end{enumerate}

\subsubsection{Sonstiges}
\begin{itemize}
  \item{je genereller die Produktgruppe, desto höher ist das Risiko für Konflikte mit älteren Marken}
  \item{wenn für einzelne Waren einer Produktgruppe/eines Oberbegriffs der Widerspruch erfolgreich ist, wird \textbf{die komplette Marke gelöscht}}
  \item{der Benutzungsnachweis (5-Jahres-Frist) muss für einen Großteil des Oberbegriffs der Produkte geliefert werden}
  \item{bei Anmeldung ausländischer Marken nach der Pariser Übereinkunft innerhalb von 6 Monaten kann der Prioritätstag gemäß \markenG{34} auf den Tag der Anmeldung im Ausland gesetzt werden. Analog verhält sich \markenG{35}}
\end{itemize}

\subsection{Prüfungsverfahren}
\begin{itemize}
  \item{\markenG{36} nennt alle  Voraussetzungen, die bei der Anmeldung geprüft werden}
  \item{die \textbf{Mindesterfordernisse} gemäß \markenG[Abs. 2]{32} entscheiden über den Prioritätstag}
  \item{die weiteren Erfordernisse in \markenG[Abs. 3]{32} müssen nur fristgerecht eingereicht werden}
  \item{falls die geforderten Dokumente nicht fristgerecht nachgereicht werden, so ergeht ein \textbf{Zurückweisungsbeschluss} gemäß \markenG[Abs. 4]{36}}
\end{itemize}

\subsection{Ablauf nach erfolgreicher Eintragung}
\begin{enumerate}
  \item{Inhaber erhält eine Eintragungsurkunde}
  \item{Die Eintragung wird im \textbf{Markenblatt} veröffentlicht}
  \item{innerhalb von 3 Monaten können Dritte der Eintragung beim DPMA widersprechen (siehe folgende Kapitel), danach ist nur noch eine Klage möglich}
\end{enumerate}

\section{Rechtsmittel}
Durch \textbf{Rechtsmittel} kann man sich gegen Beschlüsse/Anordnungen von Behörden wehren.

\subsection{Erinnerung § 64 MarkenG}
Bei der \highlight{Erinnerung} gemäß \markenG{64} handelt es sich um eine \textbf{gebührenfreie interne Überprüfung}. Sie kann gefordert werden, wenn die Entscheidung durch einen Beamten des \textbf{gehobenes Dienstes} (oder vergleichbar) getroffen wurde. Die Erinnerung wird von einem Beamten des höheren Dienstes überprüft und entweder akzeptiert oder zurückgewiesen.

\subsection{Beschwerde § 66 MarkenG}
Nach einem Zurückweisungsbeschluss oder Ablehnung durch Beamten höheren Dienstes kann Beschwerde gemäß \markenG{66} eingelegt werden. Diese muss ebenfalls beim DPMA eingereicht werden und kostet circa 300€. Entweder wird daraufhin die Eintragung verfügt oder der Fall dem Bundespatentgericht vorgelegt.

\subsection{Bundespatentgericht}
\begin{itemize}
  \item{\textbf{kein Rechtsanwaltszwang}}
  \item{kann das DPMA anweisen, die Marke einzutragen oder eine neue Überprüfung durchzuführen}
  \item{gegen den Beschluss kann beim BGH geklagt werden}
\end{itemize}

\subsection{Bundesgerichtshof}
Grundsätzlich benötigt man gemäß \markenG[Abs. 1]{83} die Zulassung des Gerichts für eine Revision beim BGH, Ausnahmen sind in Absatz 2 und 3 zu finden.
\begin{itemize}
  \item{Sachverhalt steht nach den Entscheidungen der unteren Gerichte fest}
  \item{BGH entscheidet nicht über Tatsachen, sondern Rechtsfragen/Rechtsverletzungen (z.B. ob eine Marke eintragungsfähig ist, ob Verwechslungsgefahr besteht)}
  \item{es können keine neuen Tatsachen hervorgebracht werden}
  \item{\textbf{Rechtsanwaltszwang}, nur 38 ausgewählte Anwälte sind beim BGH zugelassen}
  \item{wirksame Verweigerung der Revision vom Bundespatentgericht kann nicht angefochten werden (anders als in anderen Rechtsgebieten)}
\end{itemize}

\section{Beendigung des Markenschutzes}
\subsection{Schutzdauer}
Die Schutzdauer beträgt gemäß \markenG[Abs. 1]{47} 10 Jahre und kann gegen Gebühr unbegrenzt oft um weitere 10 Jahre verlängert werden. Nach Ablauf der Schutzdauer wird die Marke \textbf{gelöscht}.

\subsection{Verzicht}
Der Inhaber kann seine Marke gemäß \markenG{48} jederzeit löschen lassen.

\subsection{Verfall}
Unter verschiedenen Umständen kann eine Marke gemäß \markenG{49} verfallen:
\begin{itemize}
  \item{Absatz 1: Bei \textbf{5-jähriger Nichtbenutzung}, Beweislast liegt beim Kläger}
  \item{Absatz 2 Satz 1: Wenn sich die Marke \textbf{nachträglich} in eine Gattungsbezeichnung umgewandelt hat, kommt fast nie vor}
  \item{Absatz 2 Satz 2: Bei \textbf{Täuschung}}
  \item{Absatz 2 Satz 3: Wenn der Inhaber \textbf{nicht mehr fähig ist, Inhaber zu sein} (z.B. Insolvenz oder Auflösung des Unternehmens), um herrenlose Marken zu vermeiden}
\end{itemize}

\subsection{Nichtigkeit}
Eine Marke wird gemäß \markenG{50} unter bestimmten Voraussetzungen für nichtig erklärt und gelöscht, z.B. wenn sie entgegen der absoluten Schutzhindernisse eingetragen wurde.

\subsection{Vor Gericht}
Marken können ebenfalls durch \textbf{Verfalls- oder Nichtigkeitsverfahren} gemäß \markenG{55} gelöscht werden. Bei Verfallsklagen handelt es sich um \highlight{Popularklagen}, d.h. jeder kann Klage erheben.

\section{Ansprüche und Vor-/Nachteile von Klagen}
\subsection{Ansprüche}\label{sec:MarkenAnsprueche}
\textit{Wiederholung}: Markenrechte sind \textit{subjektive Rechte}:
\begin{itemize}
  \item{\textbf{Positives Benutzungsrecht}: Inhaber kann die Marke verwenden, wie er will}
  \item{\textbf{Negatives Verbietungsrecht}: Inhaber kann andere ausschließen, wie er will}
\end{itemize}
Der Inhaber hat Anspruch auf Löschung gegen andere Marken, wenn diese in Kollision mit der eigenen stehen. Zudem dürfen Dritte gemäß \markenG[Abs. 2]{14} die Marke ohne Erlaubnis nicht im geschäftlichen Verkehr verwenden.

\subsubsection{Unterlassung}
Bei Wiederholungsgefahr (i.d.R. immer gegeben) besteht gemäß \markenG[Abs. 4]{15} Anspruch auf \highlight{Unterlassung}. Dieser wird über Drohung eines Ordnungsgeldes (bis 250.000 €) oder Ordnungshaft (bis 6 Monate) durchgesetzt.

\subsubsection{Schadensersatz}
Bei \textbf{Vorsatz} oder \textbf{Fahrlässigkeit} besteht gemäß \markenG[Abs. 5]{15} Anspruch auf \highlight{Schadensersatz}. Die Dauer und Intensität der Verletzung ist hier jedoch schwierig zu beweisen und die Durchsetzung läuft nach einem mühsamen dreistufigen Verfahren ab:

\begin{enumerate}
  \item{Feststellung der Schadensersatzpflicht}
  \item{Erteilung der Auskünfte über Gewinn etc.}
  \item{Schadensersatz einklagen, dafür zwei Möglichkeiten zur Berechnung:
              \begin{enumerate}
                \item{\textbf{Konkrete Schadensberechnung}: Wie viel Gewinn ist entgangen? Problem: Kausalität, der Schaden muss aus der Verletzung entstanden sein}
                \item{\textbf{Verletzergewinn einfordern}}
              \end{enumerate}
        }
\end{enumerate}
In der Realität wird eine großzügige Lizenzgebühr als Berechnungsgrundlage genommen.

\subsubsection{Auskunftsanspruch}
Gemäß \markenG{19} besteht ein \textbf{Auskunftsanspruch} gegen den Verletzer, damit die Quellen und Vertriebswege veranschaulicht und die Größe der Verletzung berechnet werden kann. \textbf{Das Verschulden ist hierfür nicht erforderlich}.

\subsubsection{Weitere Ansprüche}
\begin{itemize}
  \item{Vernichtungsanspruch gemäß \markenG{18}}
  \item{Beschlagnahmeanspruch gemäß \markenG{146}}
\end{itemize}

\subsection{Vorteile einer Klage}
\begin{itemize}
  \item{\textbf{Wertverlust} der Marke \textbf{vermeiden}}
  \item{\textbf{Schadensersatzansprüche} geltend machen}
  \item{Marke wird \textbf{in rechtlicher Sicht gestärkt}, da sich sonst ähnliche Marken etablieren können. Dadurch würde die Unterscheidungskraft geschwächt werden und der Schutzumfang der Marke verringert werden.}
\end{itemize}

\subsection{Risiken einer Klage}
\begin{itemize}
  \item{Verwechslungsgefahr wird subjektiv bewertet, daher Risiko auf den Prozesskosten sitzen zu bleiben}
  \item{Prioritätslage muss sicher sein, sonst führt man ggf. die Löschung der eigenen Marke herbei}
\end{itemize}

\subsection{Sonstiges}
\begin{itemize}
  \item{Ansprüche verjähren gemäß \markenG{20} (und \bgb{195}) nach drei Jahren}
  \item{Wahl des Gerichtsstandes ist wichtig, da sich unterschiedliche Gerichte unterschiedlich gut mit den verschiedenen Teilbereichen auskennen}
  \item{gemäß \markenG{140} sind Landgerichte zuständig}
  \item{es kann vor dem Landgericht im Wohnsitz des Beklagten oder vor einem beliebigen Kennzeichengericht (bundesweit) geklagt werden}
  \item{der Streitwert wird vom Gericht festgelegt, liegt im Regelfall zwischen 50.000 € und 200.000 €}
\end{itemize}

\section{Gerichtsverfahren}
\subsection{Vorgerichtliche Verfahren/Abmahnverfahren}
Vor einem Gerichtsverfahren sollte der Gegenüber \textbf{über die Verletzung aufgeklärt} und die Unterlassung gefordert werden, andernfalls \textbf{mit dem Rechtsweg gedroht} werden.\par
Dies ist sinnvoll, da der Verletzer sich der Verletzung häufig nicht bewusst ist und die Prozesskosten von dem getragen werden, \textbf{der Anlass zur Klage gegeben hat}. Wenn der Beklagte den Fehler vor Gericht sofort einräumt und unterlässt, muss evtl. der Kläger die Kosten bezahlen.

\subsection{Einstweiliges Verfügungsverfahren}
\textbf{Einstweilige Verfügungen} werden i.d.R. am gleichen oder nächsten Tag ausgestellt. Dazu muss der Anspruch und die Dringlichkeit (nicht länger als 4 Wochen seit Kenntnis) glaubhaft gemacht (nicht bewiesen) werden.\par
Der Antragsgegner kann diese akzeptieren oder widersprechen. Er kann entweder einen Antrag auf Aufhebung der Einstweiligen Verfügung oder einen Antrag auf Verpflichtung zur Klageerhebung stellen. Wenn der Kläger dann bis zum festgelegten Datum keine Klage stellt, wird die Einstweilige Verfügung aufgehoben.

\subsection{Prüfungsschema des Gerichts}
\textbf{I. Zulässigkeit}
\begin{enumerate}
  \item{Zuständigkeit
              \begin{enumerate}
                \item{sachlich, gemäß \markenG{140} Landgerichte}
                \item{örtlich, gemäß ZPO mehrere Gerichtsstände}
              \end{enumerate}
        }
  \item{\textbf{Prozessführungsbefugnis} (Befugnis, Prozess in eigenem Namen zu führen)
              \begin{itemize}
                \item{Markeninhaber gemäß \markenG{28}}
                \item{Lizenznehmer mit Zustimmung des Inhabers gemäß \markenG[Abs. 3]{30}}
              \end{itemize}
        }
  \item{Allgemeine Prozessvoraussetzungen
              \begin{itemize}
                \item{nicht gleichzeitig bei zwei Gerichten}
                \item{nicht nach rechtskräftiger Entscheidung im gleichen Fall}
              \end{itemize}
        }
\end{enumerate}

\textbf{II. Begründetheit} (besteht der Anspruch wirklich)
\begin{enumerate}
  \item{Bestehens der Marke (\markenG{4})}
  \item{Verletzung (\markenG[Abs. 2]{14})
              \begin{enumerate}
                \item{inländische Verletzung}
                \item{im geschäftlichen Verkehr}
                \item{ohne Zustimmung des Inhabers}
                \item{kennzeichenmäßige Nutzung}
              \end{enumerate}
        }
  \item{\highlight{Einreden} des Beklagten (\markenGG[20-24]{20})
              \begin{enumerate}
                \item{Verjährung (\markenG{20})}
                \item{Verwirkung (\markenG{21})}
                \item{Zulässiger Drittgebrauch (\markenG[Abs. 1]{23}, Abwägung von Interessen):
                            \begin{itemize}
                              \item{Eigener Name darf verwendet werden, aber prioritätsjüngere muss zumutbares tun, um Verwechslung zu vermeiden}
                              \item{z.B. ``besonders geeignet für Microsoft Windows'' ist erlaubt}
                            \end{itemize}
                      }
                \item{Erschöpfung (\markenG{24})}
                \item{Mangelnde Benutzung (\markenG{25})}
                \item{Bestehen eines eigenen \textbf{prioritätsälteren} Rechts}
              \end{enumerate}
        }
  \item{Wiederholungsgefahr (bei Unterlassung) bzw. Verschulden (bei Schadensersatz)}
\end{enumerate}
Wenn alle Voraussetzungen vorliegen, ist der \textbf{Anspruch gegeben}.

\section{Rechtsübergang und Lizenzen}
\subsection{Rechtsübergang}
Eine Marke kann gemäß \markenG{27} durch ein Rechtsgeschäft ``aktiv übertragen werden'' oder kraft Gesetzes (z.B. bei Vererbung) übergehen.

\subsection{Lizenzierung}
Gemäß \markenG{30} können Lizenzen für Marken vergeben werden:
\begin{itemize}
  \item{Nutzung wird für bestimmte Zeit erlaubt}
  \item{Vertrag \textbf{sui generis}, formfrei und stellt ein Dauerschuldverhältnis dar}
  \item{die Lizenz wird nicht ins Markenregister eingetragen}
\end{itemize}

\subsubsection{Arten von Lizenzen}
\bgroup
\def\arraystretch{1.5}
\begin{table}[h]
  \begin{tabular}{|l|p{0.2\linewidth}|p{0.4\linewidth}|}
    \hline \rowcolor{gray!30}
    Art der Lizenz                  & Anzahl der Lizenzen & Inhaber darf Marke selbst weiter benutzen \\ \hline
    \textbf{Einfache Lizenz}      & beliebig viele      & Ja                                        \\ \hline
    \textbf{Alleinige Lizenz}       & eine                & Ja                                        \\ \hline
    \textbf{Ausschließliche Lizenz} & eine                & Nein                                      \\ \hline
  \end{tabular}
\end{table}

\section{Geschäftliche Bezeichnungen}
Gemäß \markenG{1} werden neben Marken auch \highlight{geschäftliche Bezeichnungen} geschützt.

\subsection{Voraussetzungen und Schutzbereich}
\begin{itemize}
  \item{Schutz entsteht durch Benutzung im geschäftlichen Verkehr, im Gegensatz zu Benutzungsmarken ist hier \textbf{keine Bekanntheit erforderlich}, falls Kennzeichnungskraft gegeben ist}
  \item{Schutz kann bundesweit oder lokal begrenzt gelten}
  \item{Priorität gilt wie bei Marken}
\end{itemize}

\subsection{Ansprüche}\label{sec:GeschaeftlicheBezeichnungenAnsprueche}
Ansprüche für geschäftliche Bezeichnungen sind in \markenGG[15, 18 und 19]{15} zu finden, wobei \markenG{15} den wichtigsten Anspruch darstellt und ähnlich zu \markenG{14} formuliert ist.
\begin{itemize}
  \item{\textbf{Unterlassung} (\markenG[Abs. 4]{15}, ohne Verschulden)}
  \item{\textbf{Schadensersatz} (\markenG[Abs. 5]{15})}
\end{itemize}

\subsection{Arten von geschäftlichen Bezeichnungen (§ 5 MarkenG)}
\subsubsection{Unternehmenskennzeichen}
Unternehmenskennzeichen (\markenG[Abs. 2]{5}) sind beispielsweise Uniformen oder besondere Gebäudemerkmale einer (Fast-Food-)Kette

\subsubsection{Werktitel}
Werktitel (\markenG[Abs. 3]{5}) sind beispielsweise Namen von Büchern, Fernsehsendungen und auch Titelblätter von Zeitschriften. Je nach Medium unterscheiden sich die Ansprüche an die Unterscheidungskraft, die für einen Schutz nötig sind.\par
Als Prioritätsdatum wird das Datum der erstmaligen Benutzung, z.B. der Erstaufführung oder Veröffentlichung angenommen.

\section{Geographische Herkunftsangaben}
Die dritte in \markenG{1} geschützten Kennzeichen sind \highlight{geographische Herkunftsangaben}.
\begin{itemize}
  \item{oft als Adjektive verwendet}
  \item{erhöht \textbf{Markttransparenz} für Verbraucher}
  \item{Herkunft deutet auf \textbf{Qualität/besondere Merkmale} hin}
  \item{Kennzeichnungsfunktion (aber keine individuellen Schutzrechte, jeder an dem Ort kann sie verwenden)}
  \item{kann als Kollektivmarke eingetragen werden, ist aber auch ohne Eintragung geschützt}
\end{itemize}

\subsection{Schutzebene: National}
National sind geographische Herkunftsangaben gemäß \markenG{126} geschützt. Darunter fallen \textbf{mittelbare} aber auch \textbf{unmittelbare} Herkunftsangaben, sodass kein ausdrücklicher Hinweis notwendig ist (z.B. Bocksbeutel-Wein aus Franken).\par
Allerdings werden nur Kennzeichen geschützt, die vom Verkehr \textbf{als wirkliche Herkunftsangabe verstanden} werden, so würde ``Lampe Passau'' keinen Schutz genießen.\par
Unter \markenG[Abs. 2]{126} (Gattungsbezeichnungen) fallen Zeichen, die diese Eigenschaften verloren haben, z.B. ``Wiener Schnitzel''.

\subsubsection{Schutzbereich}
Gemäß \markenG{127} umfasst der Schutzbereich von geographischen Herkunftsangaben
\begin{enumerate}
  \item{Schutz gegen Waren anderer Herkunft}
  \item{Schutz gegen qualitätsmindere Waren gleicher Herkunft}
  \item{Schutz von geographischen Herkunftsangaben mit besonderem Ruf gegen unlautere/nicht gerechtfertigte Benutzung oder Verwässerung der Herkunftsangabe}
  \item{Schutz gegen Irreführung}
\end{enumerate}

\subsubsection{Ansprüche}
Ähnliche wie bei \hyperref[sec:MarkenAnsprueche]{Marken} und \hyperref[sec:GeschaeftlicheBezeichnungenAnsprueche]{geschäftlichen Bezeichnungen} bestehen gemäß \markenG{128} Ansprüche:
\begin{itemize}
  \item{Unterlassung}
  \item{Schadensersatz (bei Verschulden)}
\end{itemize}
Gemäß \uwg[Abs. 3]{8} kann jeder klagen (\textbf{Popularklage}), nicht nur Betroffene. Ebenfalls drohen gemäß \markenG{144} strafrechtliche Konsequenzen.

\subsubsection{Prüfungsreihenfolge}
\begin{enumerate}
  \item{\markenG{126}:
              \begin{itemize}
                \item{Abs. 1: mittelbare oder unmittelbare Herkunftsangabe}
                \item{Abs. 2: keine Gattungsbezeichnung}
              \end{itemize}
        }
  \item{\markenG{127}:
              \begin{itemize}
                \item{Gefahr der Irreführung}
                \item{Gefahr Ruf zu verletzen oder auszunutzen}
                \item{mindere Qualität aus gleicher Herkunft}
              \end{itemize}
        }
\end{enumerate}

\subsection{Schutzebene: EU}
Auf \textbf{EU-Ebene} existieren drei Gütezeichen \textbf{für Agrarprodukte und Lebensmittel}, der Schutz entsteht durch Eintragung ins europäische Register.

\subsubsection{Geschützte Ursprungsbezeichnung}
\begin{itemize}
  \item{\textbf{alle} Produktionsschritte und Rohstoffe kommen aus dem Gebiet}
  \item{z.B. Käse, Wein, Champagner}
\end{itemize}

\subsubsection{Geschützte geografische Angabe}
\begin{itemize}
  \item{\textbf{mindestens ein} Produktionsschritt in dem Gebiet}
  \item{kann beim Verbraucher falsche Erwartungen wecken, da Rohstoffe aus anderem Gebiet kommen können}
\end{itemize}

\subsubsection{Garantiert traditionelle Spezialität}
\begin{itemize}
  \item{beliebige Herkunft}
  \item{\textbf{traditionelles Rezept/Herstellungsweise}}
\end{itemize}

\subsection{Schutzebene: International}
\subsubsection{Pariser Verbandsübereinkunft (PVÜ, 1883)}
\begin{itemize}
  \item{einer der ersten Verträge zum gewerblichen Rechtsschutz}
  \item{enthält Regelungen zum Markenrecht, Patentrecht und zum unlauteren Wettbewerb}
  \item{Sonderabkommen erlaubt, solange sie PVÜ nicht widersprechen}
\end{itemize}
\textbf{Schwerpunkte}:
\begin{itemize}
  \item{\highlight{Unionspriorität}: Wurde eine Marke in einem Land angemeldet, gilt das Prioritätsdatum auch in anderen Ländern, solange die Marke dort ebenfalls innerhalb von 6 Monaten angemeldet wird}
  \item{\highlight{Inländerbehandlung}: Ausländer werden wie Inländer behandelt, haben die gleichen Rechte und Ansprüche}
  \item{\textbf{Mindeststandards} für Marken festgelegt}
\end{itemize}

\subsubsection{Madrider Abkommen (MMA, 1891) und Protokoll zum Madrider Abkommen (PMMA, 1989)}
Im \textbf{Madrider Abkommen} und dem zugehörigen \textbf{Protokoll} wird die (international registrierte) \highlight{IR-Marke} eingeführt. National ist die IR-Marke in \markenGG[107 ff.]{107} festgehalten.
\begin{itemize}
  \item{Vss.: Nationale Marke im Ursprungsland}
  \item{gilt anfangs für 5 Jahre, danach unabhängig von der Basismarke}
  \item{beim Antrag müssen die \textbf{Erstreckungsländer} benannt werden, in denen der Schutz gelten soll}
  \item{Marke wird \textbf{ohne Prüfung veröffentlicht} und an die nationalen Ämter weitergeleitet}
  \item{die nationalen Ämter können widersprechen, ansonsten gilt der Schutz}
\end{itemize}
Der Vorteil der IR-Marke ist, dass nicht in jedem Land ein Antrag gestellt werden muss. Die IR-Marke ist \textbf{ein Bündel aus nationalen Rechten}!

\subsubsection{TRIPS-Abkommen (1994)}
\begin{itemize}
  \item{Abkommen zum Schutz des geistigen Eigentums}
  \item{legt Mindeststandards für z.B. Urheberrecht fest}
  \item{bei Verstoß Streitschlichtungsmechanismus}
\end{itemize}

\subsubsection{Unionsmarke (2016)}
Die \highlight{Unionsmarke} gilt innerhalb der EU, die gesetzliche Grundlage ist die \textbf{Unionsmarkenverordnung}.
\begin{itemize}
  \item{Unionsmarke entsteht nur durch Eintragung, keine Benutzungsmarken möglich}
  \item{zuständiges ist das Amt der Europäischen Union für geistiges Eigentum (EUIPO)}
  \item{Prioritätsprinzip gilt}
  \item{\textbf{Unionsmarkengerichte} werden von den Mitgliedsstaaten ernannt, z.B. Landgericht Stuttgart}
\end{itemize}
\textbf{Grundsätze}:
\begin{itemize}
  \item{\textbf{Autonomie}: Wirkung richtet sich ausschließlich nach Unionsmarkenverordnung}
  \item{\textbf{Einheitlichkeit}: Gilt für die gesamte EU, bei Zurückweisung in einem Land wird die Marke komplett gelöscht (es gibt Ausnahmen)}
  \item{\textbf{Koexistenz}: Existiert neben nationalen Marken}
\end{itemize}
Die Unionsmarkenverordnung ist in vielen Punkten fast identisch zum Markengesetz:\par
\bgroup
\def\arraystretch{1.5}
\begin{table}[h]
  \begin{tabular}{|l|p{0.3\linewidth}|p{0.3\linewidth}|}
    \hline \rowcolor{gray!30}
    Inhalt                        & Unionsmarkenverordnung & Äquivalent im MarkenG \\ \hline
    Schutzfähige Zeichen          & \umv{4}                & \markenG{3}           \\ \hline
    Entstehung des Markenschutzes & \umv{6}                & \markenG{4}           \\ \hline
    Absolute Schutzhindernisse    & \umv{7}                & \markenG{8}           \\ \hline
    Relative Schutzhindernisse    & \umv{8}                & \markenG{9}           \\ \hline
    Rechte des Inhabers           & \umv{9}                & \markenG{14}          \\ \hline
    Zulässiger Drittgebrauch      & \umv{14}               & \markenG{23}          \\ \hline
    Erschöpfung                   & \umv{15}               & \markenG{24}          \\ \hline
    Benutzungspflicht             & \umv{18}               & \markenG{26}          \\ \hline
    Rechtsübergang                & \umv{21}               & \markenG{27}          \\ \hline
  \end{tabular}
\end{table}
Beachte \umv[Abs. 1d und Abs. 2]{7}. Für \umv{18} genügt die ernsthafte Benutzung in einem Mitgliedsstaat.

\end{document}