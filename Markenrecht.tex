\documentclass[12pt,A4]{extarticle}	

\newcommand{\lectureTitle}{Markenrecht}
\newcommand{\semester}{Wintersemester 2022/23}

\usepackage[a4paper,left=0.9cm,right=1cm,top=1.37cm,bottom=2.5cm]{geometry}
\usepackage[utf8]{inputenc}
\usepackage{xifthen}
\usepackage{cmbright}
\usepackage{fontawesome}
\usepackage[T1]{fontenc}
\usepackage{lastpage,lipsum}
\usepackage{hyperref}
\usepackage{transparent}
\usepackage{color}
\usepackage{fancyhdr}

\renewcommand*\familydefault{\sfdefault}
\setlength{\parindent}{0mm}

\definecolor{headerBg}{RGB}{11, 67, 158}
\definecolor{headerGrayColor}{RGB}{210, 210, 210}

\newcommand{\printTitle}{\textcolor{white}{\lectureTitle}\normalsize}
\newcommand{\printSubtitle}{
  \ifdefined\lectureSubtitle
    \textcolor{white}{\small{\lectureSubtitle}}\\
  \fi
}

\fancyhf{}
\pagestyle{fancy}
\fancyhead[C]{
  \fcolorbox{headerBg}{headerBg}{
    \hspace{0.6cm}\begin{minipage}[c][50pt][c]{\paperwidth}
      \begin{minipage}[c]{.7\textwidth}
        \ifdefined\titleSize
          \titleSize \printTitle\\
        \else
          \huge\printTitle\\
        \fi
        \printSubtitle
        \textcolor{headerGrayColor}{\small{\semester}}
      \end{minipage}%
      \begin{minipage}[c]{.2\textwidth}
        \raggedleft
        \textcolor{white}{
          \small{\href{mailto:mail@nilslambertz.de}{\textcolor{white}{\faicon{envelope}} mail@nilslambertz.de}}\\
          \href{https://github.com/nilslambertz/kit-zusammenfassungen}{\textcolor{white}{\faicon{github}} \small{nilslambertz}}}
      \end{minipage}
    \end{minipage}}
}
\renewcommand{\headrulewidth}{0pt}
\setlength{\headheight}{40pt}

\newlength{\oddmarginwidth}
\setlength{\oddmarginwidth}{1in+\hoffset+\oddsidemargin}
\newlength{\evenmarginwidth}
\setlength{\evenmarginwidth}{\evensidemargin+1in}
\fancyhfoffset[LO,RE]{\oddmarginwidth}
\fancyhfoffset[LE,RO]{\evenmarginwidth}
\cfoot{\thepage\ $/$ \pageref*{LastPage}}

\definecolor{greenColor}{RGB}{15, 122, 25}
\newcommand{\green}[1]{\textcolor{greenColor}{#1}}
\definecolor{redColor}{RGB}{209, 8, 15}
\newcommand{\red}[1]{\textcolor{redColor}{#1}}

\definecolor{highlightColor}{RGB}{66, 135, 245}
\newcommand{\highlight}[1]{\textcolor{highlightColor}{\textbf{#1}}}
\definecolor{markenGesetzLink}{RGB}{194, 74, 14}
\newcommand{\markenG}[2][]{\textbf{\textcolor{markenGesetzLink}{\href{https://www.gesetze-im-internet.de/markeng/__#2.html}{§ #2 \ifthenelse{\equal{#1}{}}{}{#1 }MarkenG}}}}
\newcommand{\markenGG}[2][]{\textbf{\textcolor{markenGesetzLink}{\href{https://www.gesetze-im-internet.de/markeng/__#2.html}{§§ #1 MarkenG}}}}

\begin{document}

\section{Grundlagen}

\begin{itemize}
  \item{\textbf{Immatrielle Güter} sind nicht körperliche Vermögensgegenstände, also nicht greifbare Dinge}
  \item{\textbf{Gewerblicher Rechtsschutz} besteht u.a. aus Patentrecht, Lauterkeitsrecht und \textbf{Markenrecht}}
  \item{es existieren viele Berührungspunkte mit anderen Rechten z.B. im \textbf{Grundgesetz} oder im \textbf{Verwaltungsrecht}}
  \item{für Markenrechte existieren nationale, EU-weite und internationale Gesetze und Abkommen}
\end{itemize}

\subsection{Materielles vs. formelles Recht}
\textbf{Materielles Recht} (``Recht haben''):
\begin{itemize}
  \item{beschäftigt sich mit den vorliegenden Tatsachen}
  \item{beschreibt, \textbf{wie eine Marke aussehen muss}, um geschützt werden zu können}
\end{itemize}
\textbf{Formelles Recht}
\begin{itemize}
  \item{Rechtsnormen, die zur \textbf{Durchsetzung} des materiellen Rechts dienen}
  \item{beschreibt die \textbf{erforderlichen Schritte}, um eine Marke schützen zu lassen}
\end{itemize}

\section{Einführung: Markenrecht}
\subsection{Was ist das Ziel von Markenrechten?}
\begin{itemize}
  \item{Marken sollen vor Benutzung durch Dritte geschützt werden}
  \item{Inhaber erhält \textit{eigentumsähnliche} Rechte (\textbf{Subjektive Privatrechte})}
  \item{durch Marken können \textbf{Produkte} unterschiedlicher Hersteller \textbf{unterschieden werden}}
  \item{\textbf{Immatrieller Wert} eines Unternehmens besteht zu großem Teil aus Marken}
\end{itemize}

\subsection{Grundlegende Funktionen von Marken}
\begin{itemize}
  \item{\highlight{Herkunftsfunktion}: Marken kennzeichnen Hersteller}
  \item{\highlight{Unterscheidungsfunktion}: Marken unterscheiden Hersteller voneinander}
  \item{Garantie-/Qualitätsfunktion: Verbraucher verlassen sich auf gleichbleibende Qualität}
  \item{Werbefunktion: Wiedererkennungseffekt der Marke}
\end{itemize}

\subsection{Grundsätze}
\subsubsection{Grundsatz der \highlight{Priorität}}
\begin{itemize}
  \item{Rechte verschiedener Rechteinhaber erzeugen einen \textbf{Interessenkonflikt} bzw. eine \textbf{Kollision}}
  \item{\textbf{Prioritätsjüngeres} Kennzeichen \textbf{muss} dem prioritätsälterem \textbf{weichen}}
  \item{``Wer zuerst den Schutz erlangt hat, der gewinnt''}
\end{itemize}

\subsubsection{Grundsatz des \highlight{Produktbezugs} der Marke}\label{sec:Produktbezug}
\begin{itemize}
  \item{Marke wird für bestimmte Produkte/Produktklassen eingetragen}
  \item{Schutz gilt \textbf{nur für diese Produkte}}
  \item{Schutzbereich kann nachträglich nicht erweitert, sondern nur eingeschränkt werden}
\end{itemize}

\section{Markenschutz}
\subsection{Wie entsteht Markenschutz?}
Die drei Möglichkeiten zur Entstehung des Markenschutzes sind in \markenG{4} geregelt:
\begin{enumerate}
  \item{\highlight{Eintragung} der Marke durch das \textbf{DPMA} (\textbf{Registermarken})}
  \item{\highlight{Benutzungsmarken} (auch Gewohnheitsmarken genannt), wenn sie Verkehrsgeltung erlangt haben (dazu wird die Bekanntheit in den zuständigen Verkehrskreisen und die Eigenschaften der Marke wie Prägnanz, Originalität und Unterscheidungskraft betrachtet)}
  \item{\textbf{Ausländische Marken}, die im Heimatland geschützt sind, aber in Deutschland noch nicht verwendet werden}
\end{enumerate}

\subsubsection{Was geschieht bei Markenkollisionen?}
Bei Kollisionen zweier Marken wird nach der \highlight{Priorität} entschieden. Derjenige, der den Markenschutz früher erlangt hat, hat Anspruch darauf.
Bei eingetragenen Marken ist das Datum eindeutig festgelegt (siehe später).
Bei Benutzungsmarken muss letztendlich ein Gericht ein \textbf{Priorirätsdatum} nennen, woraufhin ein Gutachten zur Bekanntheit der Marke an diesem Datum erstellt wird.

\subsection{Eingetragene Marke vs. Benutzungsmarke}
\bgroup
\def\arraystretch{1.5}
\begin{table}[h]
  \begin{tabular}{|l|p{0.4\linewidth}|p{0.4\linewidth}|}
    \hline
                     & Eingetragene Marke                                                                                                                                    & Benutzungsmarke                       \\ \hline
    Eintragung       & Antrag beim DPMA                                                                                                                                      & \textit{nicht nötig}                  \\ \hline
    Rechtssicherheit & Gewisse \textbf{Rechtssicherheit}: Richter ist an Entscheidung des DPMA gebunden und kann den Schutz nicht aufheben, auch wenn er anderer Meinung ist & Richter entscheidet über Markenschutz \\ \hline
    Priorität        & Ab Tag der Anmeldung (siehe später)                                                                                                                   & Erst nach Verwendung und Etablierung  \\ \hline
    Schutzbereich    & Bundesweit                                                                                                                                            & In bestimmten Fällen nur lokal        \\ \hline
  \end{tabular}
\end{table}
\egroup

\subsubsection{Vorteile eingetragener Marken gegenüber Benutzungsmarken}
Grundsätzliche sind eingetragene Marken \textbf{NICHT} mehr wert als Benutzungsmarken und werden (falls Markenschutz für die Benutzungsmarke besteht) genau gleich behandelt.
Dennoch ergeben sich folgende Vorteile für eingetragene Marken:
\begin{itemize}
  \item{sie sind \green{ab Tag der Anmeldung geschützt}, auch wenn sie nicht verwendet werden}
  \item{sie sorgen für \green{Rechtssicherheit} (und seltenere Rechtsstreits)}
  \item{Schutz gilt \green{bundesweit}}
  \item{zur \textbf{IR-Marke} wird eine eingetragene Basismarke benötigt}
\end{itemize}

\subsection{Materielle Voraussetzungen zur Eintragung einer Marke}
Für die Eintragung einer Marke sind \markenG{3} und \markenG{8} relevant.

\subsubsection{§ 3 MarkenG}
\markenG{3} definiert die Markenfähigkeit von Zeichen (gilt auch für Benutzungsmarken!):
\begin{itemize}
  \item{Marke muss \highlight{abstrakte Unterscheidungskraft} besitzen (d.h. sie muss Waren unterscheiden können, ohne ein spezielles Produkt zu betrachten)}
  \item{Marke muss \textbf{zeichenfähig} sein, also auf die Sinnesorgane einwirken (z.B. visuell wahrgenommen werden können)}
  \item{Marke darf \textbf{kein funktionell notwendiger Bestandteil} des Produkts sein}
\end{itemize}
Diese Voraussetzungen sind von fast jedem Zeichen erfüllt.

\subsubsection{§ 8 MarkenG (Absolute Schutzhindernisse)}
\markenG{8} definiert \highlight{absolute Schutzhindernisse}, die das Eintragen einer Marke unmöglich machen.
Diese werden \textbf{von Amts wegen} (v. A. w.) geprüft, d.h. sie werden vom DPMA vor jeder Eintragung überprüft.\par
\textbf{§ 8 MarkenG gilt nicht für Benutzungsmarken!}\par
Absatz 1 sagt aus, dass das Zeichen in irgendeiner ``sichtbaren'' und speicherbaren Form vorliegen muss (z.B. ein Bild oder Video), damit werden u.a. Gerüche ausgeschlossen.\par
Absatz 2 enthält die wichtigen Schutzhindernisse:
\begin{enumerate}
  \item{Marke muss \highlight{konkrete Unterscheidungskraft} (unter Betrachtung des Produkts) besitzen, damit die Herkunftsfunktion erfüllt werden kann (``Waschmaschine'' für Waschmaschinen ist nicht konkret unterscheidungskräftig)}
  \item{\textbf{Freihaltebedürfnis} für grundlegende Begriffe der geschäftlichen Kommunikation. Die Stärke des Freihaltebedürfnisses hängt davon ab, ob und wie die Mitbewerber das Zeichen als beschreibende Eigenschaften ihrer Produkte benötigen.
              \begin{itemize}
                \item{Je größer das Freihaltebedürfnis, desto höher die Anforderungen an Unterscheidungskraft}
                \item{Je weniger beschreibend, desto eher unterscheidungskräftig ist es}
              \end{itemize}
        }
  \item{\textbf{Gattungsbezeichnungen} können für das Produkt nicht geschützt werden (``Diesel'' kann nicht für Motoren geschützt werden, für Kleidung allerdings schon)}
  \item{\textbf{Täuschende Zeichen} können nicht geschützt werden, z.B. unberechtigte Titel (Dr. Oetker musste wirklich Doktor sein) oder falsche Altersangaben (``seit 1900'')}
  \item{\textbf{Verbotene und sittenwidrige Zeichen} können nicht geschützt werden}
  \item{\textbf{Wappen oder Flaggen} können nicht geschützt werden}
\end{enumerate}
Bei \highlight{Verkehrsdurchsetzung} (extrem hoher Bekanntheit in den beteiligten Verkehrskreisen) existieren Ausnahmen von Absatz 2 Nr. 1, 2 und 3, siehe \markenG[Abs. 3]{8}

\subsection{Markentypen/Markenfähige Zeichen}
Grundsätzlich sind alle Zeichen als Marke zugelassen, nicht nur die, die in \markenG[Abs. 1]{3} aufgezählt werden, darunter fallen
\begin{itemize}
  \item{Wortmarken (wenn sie nicht nur das Produkt beschreiben)}
  \item{Buchstaben (falls sie nicht als branchenüblichen Beschreibung von Objekten, Klassen usw. verwendet werden)}
  \item{Bildmarken wie Logos und grafisch gestaltete Schriftzüge (einfachste geomatrische Formen sind nicht unterscheidungskräftig), auch Wort-Bild-Marken möglich}
  \item{Farben haben i.d.R. keine Unterscheidungskraft (Ausnahmen bestätigen die Regel, z.B. bei Verkehrsdurchsetzung)}
  \item{Geruchs- und Geschmacksmarken sind \textbf{nicht schutzfähig}}
  \item{Bewegungsmarke (z.B. brüllender Löwe in Film-Intros)}
  \item{Positionsmarken (z.B. Knopf im Ohr bei Steiff)}
  \item{dreidimensionale Gestaltungen:
              \begin{itemize}
                \item{Formmarken (z.B. Mercedes-Stern)}
                \item{Warenformmarken (z.B. quadratische Form von Ritter Sport)}
                \item{Warenverpackungsformmarken (z.B. Coca-Cola-Flasche)}
              \end{itemize}
        }
\end{itemize}

\subsubsection{Besonderheiten für dreidimensionale Gestaltungen}
Gemäß \markenG[Abs. 2]{3} sind einige Formen nicht schutzfähig:
\begin{enumerate}
  \item{wenn es sich um die Grundform des Produkts handelt (z.B. Birnen-Form für Birnen)}
  \item{wenn die Form technisch bedingt ist (z.B. Form von Klemmbausteinen)}
  \item{wenn die Form der Waren einen wesentlichen (oft ästhetischen) Wert verleiht (z.B. die Form eines Rings)}
\end{enumerate}

\subsubsection{Rechtserhaltung bei Marken}
Die \textbf{Rechtserhaltung} (Benutzungspflicht der Marke zur Wahrung der Rechte) ist bei Wortmarken einfacher zu gewährleisten, da beliebige Schriftarten und Farben verwendet werden können.

\subsection{Anmeldung einer Marke}
\subsubsection{Marken doppelt anmelden}
In bestimmten Fällen ist es sinnvoll, die Marke mehrfach anzumelden.
Während Wortmarken von einem relativ großen Schutzbereich umfasst werden (werden ohne feste Schriftart eingetragen, alle verkehrsüblichen Schreibweisen sind umfasst),
sind Wort-Bild-Marken auf diese Schriftart begrenzt.\par
Dadurch könnten Mitbewerber den Schriftzug verwenden, wenn die Grafik weggelassen wird. Dagegen hilft die \textbf{doppelte Eintragung} als Wortmarke und Wort-Bild-Marke.\par
Es empfiehlt sich ebenfalls, Bildmarken in \textbf{mehreren Farben} (zumindest schwarz/weiß und farbig) anzumelden.

\subsubsection{Strategie der Eintragung}
Bei einem größeren Produktportfolio können verschiedene Strategien bei der Eintragung angewendet werden:
\begin{itemize}
  \item{\highlight{Einzelmarkenstrategie}:
              \begin{itemize}
                \item{jedes Produkt unter eigener Marke}
                \item{sinnvoll, wenn die Zielgruppe sich scharf abgrenzen lässt}
                \item{relativ hohe Kosten}
              \end{itemize}
        }
  \item{\highlight{Dachmarkenstrategie}:
              \begin{itemize}
                \item{alle Produkte unter einer Dachmarke}
                \item{sinnvoll, wenn Produktangebot zu groß}
                \item{wenig Aufwand, Einführung neuer Produkte wird vereinfacht (bekannte Marke für Kunden)}
              \end{itemize}
        }
  \item{\highlight{Familienmarkenstrategie}:
              \begin{itemize}
                \item{jede Produktgruppe unter eigener Marke}
                \item{praktisch Dachmarke für jede Produktgruppe mit den daraus entstehenden Vorteilen}
              \end{itemize}
        }
\end{itemize}

\section{Formelles Markenrecht}
Für den Eintrag einer Marke muss ein Antrag beim DPMA gestellt werden, dabei wird das Zeichen auf materielle und formelle Voraussetzungen geprüft. Falls diese erfüllt sind, besteht ein Anspruch auf Eintragung gemäß \markenG[Abs. 2]{33}.

\subsection{Anwartschaft}
Nachdem alle erforderlichen Schritte unternommen wurden und der Antrag beim DPMA eingegangen ist, ist man in der Position der \textbf{Anwartschaft}. Dies bedeutet, dass alles mögliche unternommen wurde und keine weiteren Schritte unternommen werden können.

\subsection{Anmeldeverfahren §§ 32 ff. MarkenG}
Für die Anmeldung muss ein \textbf{Formblatt} ausgefüllt und eingereicht werden. \textbf{Der Einreichungstag gilt als Anmeldedatum} (für Priorität wichtig), man erhält eine Empfangsbestätigung mit Aktenzeichen, Datum und fälligen Gebühren.\par
Für die Anmeldung müssen die Unterlagen gemäß \markenG[Abs. 2 und 3]{32} vorliegen. Für das Anmeldedatum ist \textbf{nur Absatz 2} relevant (\markenG[Abs. 1]{33}).\par
Falls die Anmeldung bzgl \markenG[Abs. 2]{32} fehlerhaft ist, verschiebt sich der Priorität auf den Tag der Nachreichung. Bei fehlenden Angaben nach Absatz 3 hat dies keinen Einfluss auf die Priorität (solange die Angaben fristgerecht nachgereicht werden).\par
Für die erfolgreiche Anmeldung ist gemäß \markenG[Abs. 1 S. 3]{36} die fristgerechte Zahlung der Gebühren wichtig.

\subsubsection{Klassifikation für das Warenverezichnis}
Zu jeder Anmeldung muss ein \textbf{Waren- und Dienstleisungsverzeichnis} (\ref{sec:Produktbezug}) beigefügt werden. Dabei kann die \highlight{Klassifikation von Nizza} verwendet werden, die 34 Waren- und 11 Dienstleistungsklassen unterscheidet. Es können allerdings auch detaillierte Produkte spezifiziert werden.

\subsubsection{Kette der Vorschriften, die geprüft werden}
Die Vorschriften werden in folgender Reihenfolge geprüft:
\begin{enumerate}
  \item{\markenG[Abs. 1]{36}}
  \item{\markenG[Abs. 1]{33}}
  \item{\markenG[Abs. 2]{32}}
\end{enumerate}

\subsubsection{Sonstiges}
\begin{itemize}
  \item{je genereller die Produktgruppe, desto höher ist das Risiko für Konflikte mit älteren Marken}
  \item{wenn für einzelne Waren einer Produktgruppe/eines Oberbegriffs der Widerspruch erfolgreich ist, wird \textbf{die komplette Marke gelöscht}}
  \item{der Benutzungsnachweis (5-Jahres-Frist) muss für einen Großteil des Oberbegriffs der Produkte geliefert werden}
  \item{bei Anmeldung ausländischer Marken nach der Pariser Übereinkunft innerhalb von 6 Monaten kann der Prioritätstag gemäß \markenG{34} auf den Tag der Anmeldung im Ausland gesetzt werden. Analog verhält sich \markenG{35}}
\end{itemize}

\subsection{Prüfungsverfahren}
\begin{itemize}
  \item{\markenG{36} nennt alle  Voraussetzungen, die bei der Anmeldung geprüft werden}
  \item{die \textbf{Mindesterfordernisse} gemäß \markenG[Abs. 2]{32} entscheiden über den Prioritätstag}
  \item{die weiteren Erfordernisse in \markenG[Abs. 3]{32} müssen nur fristgerecht eingereicht werden}
  \item{falls die geforderten Dokumente nicht fristgerecht nachgereicht werden, so ergeht ein \textbf{Zurückweisungsbeschluss} gemäß \markenG[Abs. 4]{36}}
\end{itemize}

\subsection{Ablauf nach erfolgreicher Eintragung}
\begin{enumerate}
  \item{Inhaber erhält eine Eintragungsurkunde}
  \item{Die Eintragung wird im \textbf{Markenblatt} veröffentlicht}
  \item{innerhalb von 3 Monaten können Dritte der Eintragung beim DPMA widersprechen (siehe folgende Kapitel), danach ist nur noch eine Klage möglich}
\end{enumerate}

\subsection{Einschub: Kollektivmarken}
Neben \highlight{Individualmarken} können auch \highlight{Kollektivmarken} (\markenG{97}) eingetragen werden:
\begin{itemize}
  \item{Kollektivmarken können nur von \textbf{rechtsfähigen Verbänden} eingetragen werden}
  \item{Der Verband muss eine Satzung beifügen, wann die Marke verwendet werden darf}
  \item{jeder (auch Unternehmen), der die Voraussetzungen erfüllt, darf die Marke verwenden}
  \item{gemäß \markenG[Abs. 1]{97} können auch geographische Marken geschützt werden, was bei Individualmarken durch die absoluten Schutzhindernisse gemäß \markenG[Abs. 2 S. 2]{8} nicht möglich ist}
  \item{Kollektivmarken können als Gütezeichen z.B. für Lebensmittel verwendet werden, wenn ein Gutachten vorliegt}
\end{itemize}

\end{document}