\documentclass[12pt,A4]{extarticle}	

\newcommand{\lectureTitle}{Markenrecht}
\newcommand{\semester}{Wintersemester 2022/23}

\usepackage[a4paper,left=0.9cm,right=1cm,top=1.37cm,bottom=2.5cm]{geometry}
\usepackage[utf8]{inputenc}
\usepackage{xifthen}
\usepackage{cmbright}
\usepackage{fontawesome}
\usepackage[T1]{fontenc}
\usepackage{lastpage,lipsum}
\usepackage{hyperref}
\usepackage{transparent}
\usepackage{color}
\usepackage{fancyhdr}

\renewcommand*\familydefault{\sfdefault}
\setlength{\parindent}{0mm}

\usepackage{transparent}
\usepackage{color}
\usepackage{fancyhdr}

\definecolor{headerBg}{RGB}{11, 67, 158}
\definecolor{headerGrayColor}{RGB}{210, 210, 210}

\pagestyle{fancy}
\fancyhead[C]{
  \fcolorbox{headerBg}{headerBg}{
    \hspace{0.6cm}\begin{minipage}[c][50pt][c]{\paperwidth}
      \begin{minipage}[c]{.45\textwidth}
        \huge{\textcolor{white}{Vorlesung}}\normalsize\\
        \textcolor{headerGrayColor}{\small{Wintersemester 2022/23}}
      \end{minipage}%
      \begin{minipage}[c]{.45\textwidth}
        \raggedleft
        \textcolor{white}{
          \small{\href{https://nilslambertz.de/}{nilslambertz.de}}\\
          \href{https://github.com/nilslambertz/}{\textcolor{white}{\faicon{github}} \small{nilslambertz}}}
      \end{minipage}
    \end{minipage}}
}
\renewcommand{\headrulewidth}{0pt}
\setlength{\headheight}{40pt}

\newlength{\oddmarginwidth}
\setlength{\oddmarginwidth}{1in+\hoffset+\oddsidemargin}
\newlength{\evenmarginwidth}
\setlength{\evenmarginwidth}{\evensidemargin+1in}
\fancyhfoffset[LO,RE]{\oddmarginwidth}
\fancyhfoffset[LE,RO]{\evenmarginwidth}
\cfoot{\thepage\ $/$ \pageref*{LastPage}}

\definecolor{greenColor}{RGB}{15, 122, 25}
\newcommand{\green}[1]{\textcolor{greenColor}{#1}}
\definecolor{redColor}{RGB}{209, 8, 15}
\newcommand{\red}[1]{\textcolor{redColor}{#1}}

\definecolor{highlightColor}{RGB}{66, 135, 245}
\newcommand{\highlight}[1]{\textcolor{highlightColor}{\textbf{#1}}}
\definecolor{markenGesetzLink}{RGB}{194, 74, 14}
\newcommand{\markenG}[2][]{\textbf{\textcolor{markenGesetzLink}{\href{https://www.gesetze-im-internet.de/markeng/__#2.html}{§ #2 \ifthenelse{\equal{#1}{}}{}{#1 }MarkenG}}}}

\begin{document}

\section{Grundlagen}

\begin{itemize}
  \item{\textbf{Immatrielle Güter} sind nicht körperliche Vermögensgegenstände, also nicht greifbare Dinge}
  \item{\textbf{Gewerblicher Rechtsschutz} besteht u.a. aus Patentrecht, Lauterkeitsrecht und \textbf{Markenrecht}}
  \item{es existieren viele Berührungspunkte mit anderen Rechten z.B. im \textbf{Grundgesetz} oder im \textbf{Verwaltungsrecht}}
  \item{für Markenrechte existieren nationale, EU-weite und internationale Gesetze und Abkommen}
\end{itemize}

\subsection{Materielles vs. formelles Recht}
\textbf{Materielles Recht} (``Recht haben''):
\begin{itemize}
  \item{beschäftigt sich mit den vorliegenden Tatsachen}
  \item{beschreibt, \textbf{wie eine Marke aussehen muss}, um geschützt werden zu können}
\end{itemize}
\textbf{Formelles Recht}
\begin{itemize}
  \item{Rechtsnormen, die zur \textbf{Durchsetzung} des materiellen Rechts dienen}
  \item{beschreibt die \textbf{erforderlichen Schritte}, um eine Marke schützen zu lassen}
\end{itemize}

\section{Einführung: Markenrecht}
\subsection{Was ist das Ziel von Markenrechten?}
\begin{itemize}
  \item{Marken sollen vor Benutzung durch Dritte geschützt werden}
  \item{Inhaber erhält \textit{eigentumsähnliche} Rechte (\textbf{Subjektive Privatrechte})}
  \item{durch Marken können \textbf{Produkte} unterschiedlicher Hersteller \textbf{unterschieden werden}}
  \item{\textbf{Immatrieller Wert} eines Unternehmens besteht zu großem Teil aus Marken}
\end{itemize}

\subsection{Grundlegende Funktionen von Marken}
\begin{itemize}
  \item{\highlight{Herkunftsfunktion}: Marken kennzeichnen Hersteller}
  \item{\highlight{Unterscheidungsfunktion}: Marken unterscheiden Hersteller voneinander}
  \item{Garantie-/Qualitätsfunktion: Verbraucher verlassen sich auf gleichbleibende Qualität}
  \item{Werbefunktion: Wiedererkennungseffekt der Marke}
\end{itemize}

\subsection{Grundsätze}
\subsubsection{Grundsatz der \highlight{Priorität}}
\begin{itemize}
  \item{Rechte verschiedener Rechteinhaber erzeugen einen \textbf{Interessenkonflikt} bzw. eine \textbf{Kollision}}
  \item{\textbf{Prioritätsjüngeres} Kennzeichen \textbf{muss} dem prioritätsälterem \textbf{weichen}}
  \item{``Wer zuerst den Schutz erlangt hat, der gewinnt''}
\end{itemize}

\subsubsection{Grundsatz des \highlight{Produktbezugs} der Marke}
\begin{itemize}
  \item{Marke wird für bestimmte Produkte/Produktklassen eingetragen}
  \item{Schutz gilt \textbf{nur für diese Produkte}}
\end{itemize}

\section{Markenschutz}
\subsection{Wie entsteht Markenschutz?}
Die drei Möglichkeiten zur Entstehung des Markenschutzes sind in \markenG{4} geregelt:
\begin{enumerate}
  \item{\highlight{Eintragung} der Marke durch das \textbf{DPMA}}
  \item{\highlight{Benutzungsmarken} (auch Gewohnheitsmarken genannt), wenn sie Verkehrsgeltung erlangt haben (dazu wird die Bekanntheit in den zuständigen Verkehrskreisen und die Eigenschaften der Marke wie Prägnanz, Originalität und Unterscheidungskraft betrachtet)}
  \item{\textbf{Ausländische Marken}, die im Heimatland geschützt sind, aber in Deutschland noch nicht verwendet werden}
\end{enumerate}

\subsubsection{Was geschieht bei Markenkollisionen?}
Bei Kollisionen zweier Marken wird nach der \highlight{Priorität} entschieden. Derjenige, der den Markenschutz früher erlangt hat, hat Anspruch darauf.
Bei eingetragenen Marken ist das Datum eindeutig festgelegt (siehe später).
Bei Benutzungsmarken muss letztendlich ein Gericht ein \textbf{Priorirätsdatum} nennen, woraufhin ein Gutachten zur Bekanntheit der Marke an diesem Datum erstellt wird.

\subsection{Eingetragene Marke vs. Benutzungsmarke}
\bgroup
\def\arraystretch{1.5}
\begin{table}[h]
  \begin{tabular}{|l|p{0.4\linewidth}|p{0.4\linewidth}|}
    \hline
                     & Eingetragene Marke                                                                                                                                    & Benutzungsmarke                       \\ \hline
    Eintragung       & Antrag beim DPMA                                                                                                                                      & \textit{nicht nötig}                  \\ \hline
    Rechtssicherheit & Gewisse \textbf{Rechtssicherheit}: Richter ist an Entscheidung des DPMA gebunden und kann den Schutz nicht aufheben, auch wenn er anderer Meinung ist & Richter entscheidet über Markenschutz \\ \hline
    Priorität        & Ab Tag der Anmeldung (siehe später)                                                                                                                   & Erst nach Verwendung und Etablierung  \\ \hline
    Schutzbereich    & Bundesweit                                                                                                                                            & In bestimmten Fällen nur lokal        \\ \hline
  \end{tabular}
\end{table}
\egroup

\subsubsection{Vorteile eingetragener Marken gegenüber Benutzungsmarken}
Grundsätzliche sind eingetragene Marken \textbf{NICHT} mehr wert als Benutzungsmarken und werden (falls Markenschutz für die Benutzungsmarke besteht) genau gleich behandelt.
Dennoch ergeben sich folgende Vorteile für eingetragene Marken:
\begin{itemize}
  \item{sie sind \green{ab Tag der Anmeldung geschützt}, auch wenn sie nicht verwendet werden}
  \item{sie sorgen für \green{Rechtssicherheit} (und seltenere Rechtsstreits)}
  \item{Schutz gilt \green{bundesweit}}
  \item{zur \textbf{IR-Marke} wird eine eingetragene Basismarke benötigt}
\end{itemize}

\subsection{Materielle Voraussetzungen zur Eintragung einer Marke}
Für die Eintragung einer Marke sind \markenG{3} und \markenG{8} relevant.

\subsubsection{§ 3 MarkenG}
\markenG{3} definiert die Markenfähigkeit von Zeichen (gilt auch für Benutzungsmarken!):
\begin{itemize}
  \item{Marke muss \highlight{abstrakte Unterscheidungskraft} besitzen (d.h. sie muss Waren unterscheiden können, ohne ein spezielles Produkt zu betrachten)}
  \item{Marke muss \textbf{zeichenfähig} sein, also auf die Sinnesorgane einwirken (z.B. visuell wahrgenommen werden können)}
  \item{Marke darf \textbf{kein funktionell notwendiger Bestandteil} des Produkts sein}
\end{itemize}
Diese Voraussetzungen sind von fast jedem Zeichen erfüllt.

\subsubsection{§ 8 MarkenG}
\markenG{8} definiert \highlight{absolute Schutzhindernisse}, die das Eintragen einer Marke unmöglich machen.
Sie werden \textbf{von Amts wegen} (v. A. w.) geprüft, d.h. sie werden vom DPMA vor jeder Eintragung überprüft.\par
\textbf{§ 8 MarkenG gilt nicht für Benutzungsmarken!}

\end{document}