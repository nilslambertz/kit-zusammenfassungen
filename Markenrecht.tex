\documentclass[12pt,A4]{extarticle}	

\newcommand{\lectureTitle}{Markenrecht}
\newcommand{\semester}{Wintersemester 2022/23}

\usepackage[a4paper,left=0.9cm,right=1cm,top=1.37cm,bottom=2.5cm]{geometry}
\usepackage[utf8]{inputenc}
\usepackage{xifthen}
\usepackage{cmbright}
\usepackage{fontawesome}
\usepackage[T1]{fontenc}
\usepackage{lastpage,lipsum}
\usepackage{hyperref}
\usepackage{transparent}
\usepackage{color}
\usepackage{fancyhdr}

\renewcommand*\familydefault{\sfdefault}
\setlength{\parindent}{0mm}

\definecolor{headerBg}{RGB}{11, 67, 158}
\definecolor{headerGrayColor}{RGB}{210, 210, 210}

\newcommand{\printTitle}{\textcolor{white}{\lectureTitle}\normalsize}
\newcommand{\printSubtitle}{
  \ifdefined\lectureSubtitle
    \textcolor{white}{\small{\lectureSubtitle}}\\
  \fi
}

\fancyhf{}
\pagestyle{fancy}
\fancyhead[C]{
  \fcolorbox{headerBg}{headerBg}{
    \hspace{0.6cm}\begin{minipage}[c][50pt][c]{\paperwidth}
      \begin{minipage}[c]{.7\textwidth}
        \ifdefined\titleSize
          \titleSize \printTitle\\
        \else
          \huge\printTitle\\
        \fi
        \printSubtitle
        \textcolor{headerGrayColor}{\small{\semester}}
      \end{minipage}%
      \begin{minipage}[c]{.2\textwidth}
        \raggedleft
        \textcolor{white}{
          \small{\href{mailto:mail@nilslambertz.de}{\textcolor{white}{\faicon{envelope}} mail@nilslambertz.de}}\\
          \href{https://github.com/nilslambertz/kit-zusammenfassungen}{\textcolor{white}{\faicon{github}} \small{nilslambertz}}}
      \end{minipage}
    \end{minipage}}
}
\renewcommand{\headrulewidth}{0pt}
\setlength{\headheight}{40pt}

\newlength{\oddmarginwidth}
\setlength{\oddmarginwidth}{1in+\hoffset+\oddsidemargin}
\newlength{\evenmarginwidth}
\setlength{\evenmarginwidth}{\evensidemargin+1in}
\fancyhfoffset[LO,RE]{\oddmarginwidth}
\fancyhfoffset[LE,RO]{\evenmarginwidth}
\cfoot{\thepage\ $/$ \pageref*{LastPage}}

\definecolor{markenGesetzLink}{RGB}{194, 74, 14}
\newcommand{\markenG}[2][]{\textbf{\textcolor{markenGesetzLink}{\href{https://www.gesetze-im-internet.de/markeng/__#2.html}{§ #2 \ifthenelse{\equal{#1}{}}{}{#1 }MarkenG}}}}

\begin{document}

\section{Grundlagen}

\begin{itemize}
  \item{\textbf{Immatrielle Güter} sind nicht körperliche Vermögensgegenstände, also nicht greifbare Dinge}
  \item{\textbf{Gewerblicher Rechtsschutz} besteht u.a. aus Patentrecht, Lauterkeitsrecht und \textbf{Markenrecht}}
  \item{es existieren viele Berührungspunkte mit anderen Rechten z.B. im \textbf{Grundgesetz} oder im \textbf{Verwaltungsrecht}}
  \item{für Markenrechte existieren nationale, EU-weite und internationale Gesetze und Abkommen}
\end{itemize}

\section{Einführung: Markenrecht}
\subsection{Was ist das Ziel von Markenrechten?}
\begin{itemize}
  \item{Marken sollen vor Benutzung durch Dritte geschützt werden}
  \item{Inhaber erhält \textit{eigentumsähnliche} Rechte (\textbf{Subjektive Privatrechte})}
  \item{durch Marken können \textbf{Produkte} unterschiedlicher Hersteller \textbf{unterschieden werden}}
  \item{\textbf{Immatrieller Wert} eines Unternehmens besteht zu großem Teil aus Marken}
\end{itemize}

\end{document}